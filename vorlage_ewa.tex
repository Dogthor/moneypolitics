\documentclass[
        onecolumn,
        a4paper,
        abstracton,
        parskip=half
        %,draft
        ,final
        ]{scrartcl}

        \usepackage[pdftex
        ,draft
        ]{graphicx}

        \usepackage{booktabs}

        \usepackage[cmex10]{amsmath}

        \interdisplaylinepenalty=2500
        \usepackage{url}

        \usepackage[breaklinks]{hyperref}
        \hyphenation{nothing} % correct bad hyphenation here

        \usepackage{eurosym}

        \usepackage{listings}
        \lstset{basicstyle=\small\ttfamily,breaklines=true}
        \emergencystretch 1000pt

        \usepackage{subcaption}

        \usepackage{mathtools}

        % deutsche Silbentrennung
        \usepackage[ngerman]{babel}

        \usepackage[printonlyused, withpage]{acronym}

        \usepackage[a4paper]{geometry}

        % wegen deutschen Umlauten
        \usepackage[ansinew]{inputenc}

        % fuer Zitate
        \usepackage[round]{natbib}

        \usepackage{setspace}
        \usepackage{tabulary}
        \usepackage{units}
        \usepackage{cite}
        %%% Deutsche Verzeichnis-ueberschriften

        \renewcommand{\contentsname}{Inhalt}
        \renewcommand{\listtablename}{Tabellenverzeichnis}
        \renewcommand{\listfigurename}{Abbildungsverzeichnis}

        %%% Kommentarfunktion %%%
        \usepackage[textwidth=2.2cm
        ,obeyFinal
        ]{todonotes}

\begin{document}

    %%%%%% CREATING THE TITLE %%%%%
    \titlehead{
              \centering
              \includegraphics[width = 0.3\textwidth ]{logo_wip.jpg} \\
              \small Technische Universit{\"a}t Berlin \\
              Fakult{\"a}t VII (Wirtschaft \& Management) \\
              Fachgebiet Wirtschafts- und Infrastrukturpolitik (WIP)
              }
              \title{Geldpolitische Str{\"o}mungen und Instrumente}
              %\subtitle{Untertitel}
              \author{
              Gregor May
             \footnotesize \textit{(Matr: 357150)}
              \and
              Marius Hanniske
             \footnotesize \textit{(Matr: 311263)}
              }

              \date{\today}

              \maketitle
              \newpage

       %%%% ABSTRACT %%%%%

        \selectlanguage{ngerman}

        \begin{abstract}

Der Begriff Geldpolitik f{\"u}hrt immer wieder zu Hilflosigkeit, {\"U}berforderung oder falscher Benutzung. Diese Arbeit soll das Verst{\"a}ndnis von Geldpolitik kl{\"a}ren. Erg{\"a}nzt wird dies durch die Vorstellung verschiedener geldpolitischer Mittel im Euroraum. Anschlie{"s}end werden die Grundgedanken der {\"O}konomen Milton Friedman, Maynard Keynes und Friedrich von Hayek vorgestellt. Eine  Case Study am Ende soll die Frage erl{\"a}utern, ob ein Eingriff des Staates in die Geldpolitik sinnvoll ist.

        \end{abstract}
           \newpage

          \begin{flushleft}
          %\nointend \textbf{JEL Codes:} C61, H54, L94 \\
          %\nointend \textbf{Keywords:} Money, Money Politics, Econimics
          \end{flushleft}

            \tableofcontents
            %\listoffigures
            %\listoftables

            \newpage
            \onehalfspacing

      %%%%% TEXTK{\"o}RPER %%%%%

\section{Einleitung}
    \label{sec1:einleitung}
 %% Autor: Marius Hanniske, Gregor May %%

Das Leben des 1671 in Schottland geborenen John Law liest sich wie ein M{\"a}rchen\footnote[1]{\citep*[vgl.][Kap.2]{strathern2006schumpeters}}. Ein Draufg{\"a}nger im Kasino und mit den Frauen. In England zum Tode verurteilt, weil er als Sieger aus einem Duell hervorging, floh er nach Europa und bekam in den dortigen Kasinos und der fortschrittlicheren Finanzpolitik anderer L{\"a}nder, wie z.B. die Niederlande\footnote[2]{Belebung der Schifffahrt dank Windm{\"u}hlen, erster B{\"o}rsencrash auf dem Tulpenmarkt} ein f{\"u}r diese Zeit, neues Verst{\"a}ndnis von Geld: "`Ungenutztes Geld war nichts - nichts als das Potential f{\"u}r Aktion."'\footnote[3]{\citep*[vgl.][Kap.3]{strathern2006schumpeters}} Law erkannte rasch, dass mehr Bargeld zu mehr Handelsaktivit{\"a}t f{\"u}hrt und verbreitete seine Ideen u. a. in Schottland und Turin. Die Staaten konnten Noten auf sich selbst ausstellen, d. h. Kredite gew{\"a}hren im Tausch mit der F{\"a}higkeit, in Zukunft Geld aufzubringen, mit Hilfe von Steuern. Dieses Geld wird in England noch heute fiat money\footnote[4]{fiduzi{\"a}r: nicht durch Gold gedeckt, lat.: fiducia: Vertrauen} genannt und ist in Europa und auch den USA\footnote[5]{"`in god we trust"' steht u. a. auf der 1-Dollar-Note und Gottesvertrauen braucht man auch} gel{\"a}ufig. Durch Freunde bekam Law Kontakt zum Regenten von Frankreich\footnote[6]{Philippe, Herzog von Orl\'{e}ans, nachdem der Sonnenk{\"o}nig Ludwig 14. 1715 starb}, der ihm vollstes Vertrauen entgegen brachte. Die franz{\"o}sische Staatskasse, die zu der Zeit 3 Milliarden Livres\footnote[7]{Livre frz.: Pfund war vom 9. bis zum 18. Jh. die franz{\"o}sische W{\"a}hrung, 1795 abgel{\"o}st vom Franc} an Schulden angeh{\"a}uft hat, ben{\"o}tigte neue Ideen und Law war der Mann der sie hatte. Er gr{\"u}ndete 1716 die Banque National, die erste Bank Frankreichs, ausgestattet mit 6 Millionen Livres Aktienkapital. Bareinzahlungen waren in Form von M{\"u}nzen auf Konten m{\"o}glich, auch {\"U}berweisungen durch Schecks auf andere Konten. Was aber neu und besonders war stellte "raison d'etre"\footnote[8]{gedeckt durch die 6 Millionen Aktienkapital, Versprechen(!) auf Auszahlung, allerdings Barreserven von 350 tausend Livres} dar, d. h. die Ausgabe von Papiergeld. Das war eine grandiose Idee und sie funktionierte. Leute hatten mehr Geld zur Verf{\"u}gung und gaben es auch aus, G{\"u}ternachfrage stieg an und damit auch die G{\"u}terproduktion. Auch einen weiteren Plan setzte Law um: die Gr{\"u}ndung der Mississippi-Gesellschaft. Philippe {\"u}bereignete ihm dazu Louisiana, mit der Ausgabe von Aktien sollte eine Expedition\footnote[9]{\citep*[vgl.][Kap.3]{strathern2006schumpeters}} finanziert werden. Law und seine Familie bewegten sich in den h{\"o}chsten Kreisen, ja befanden sich f{\"o}rmlich an der Spitze, der High Society. Der p{\"a}pstliche Nimbus war zur Geburtstagsfeier von Laws Tochter eingeladen, gab der Tochter einen Kuss, ihr {\"a}lterer Bruder ging mit Ludwig 15. jagen.
War alles perfekt? Konnte es so weitergehen? Das "`System"', wie Law es nannte, brach in sich zusammen. Die Gewinne wurden nicht wie versprochen f{\"u}r eine Expedition ausgegeben, sondern zur Begleichung der immensen Staatsschulden. Bei einer Abwertung der Papiernoten verlor die Bev{\"o}lkerung endg{\"u}ltig das Vertrauen in jene.
Wie erw{\"a}hnt funktioniert die heutige Geldpolitik wie in dem von Law durchgef{\"u}hrten Feldexperiment, doch Sicherheitsmechanismen verhindern meist eine derartige Eskalation.

Wir befassen uns in dieser Hausarbeit mit den Instrumentarien der Geldpolitik, ihrer Zielsetzung und Akteure. Dabei gehen wir auch auf Denkanst{\"o}{"s}e der Str{\"o}mungen des Keynesianismus, der Monetaristen und der Gruppe der neuen politischen {\"O}konomie, der {\"o}sterreichischen Schule ein. Abschlie{"s}end widmen wir uns der Frage, ob ein staatliches Intervenieren in der Geldpolitik sinnvoll ist.





\subsection{Benennung der Fragestellung}

%% Autor: Marius Hanniske, Gregor May %%

Am Anfang steht die Forschung. Dabei sind zwei Motive herausgearbeitet, die Triebfedern der wissenschaftlichen Forschung darstellen:
1.) die Neugierde nach Wissen {\"u}ber die Welt. Sie ist regelm{\"a}{"s}ig Motivation zur Forschung, so versuchte auch John Law  seine Ideen umzusetzen um herauszufinden ob sie funktionieren.
Eine 2.) treibende Kraft ist die unvollst{\"a}ndige Information, welche daf{\"u}r sorgt, dass Menschen nur eine vage oder gar keine Kenntnis von der Zukunft besitzen. Die Angst vor dem Unbekannten veranlasst Menschen dazu diese Unkenntnis zu beseitigen. Der Drang zur Ver{\"a}nderung der Situation schwindet jedoch, wenn die Zukunft vielversprechend aussieht.\footnote[10]{ \citep*[S.11]{bombach1981theorie}} 

Die M{\"o}glichkeiten zur Ver{\"a}nderung ist nicht zu vernachl{\"a}ssigen als Reaktion auf die entt{\"a}uschenden Ergebnisse wissenschaftlicher Theorien. Was tun, wenn sich eine Diskrepanz zwischen der Theorie und Realit{\"a}t aufzeigt?
 Eine M{\"o}glichkeit: Evolution, Anpassung der Theorie an die Realit{\"a}t. Dies ist um so schwieriger sobald die Theorie fest gefahren ist, auf Probleme mit den immer gleichen Argumentationen antworten will. 
 Dann verbleibt nur eine zweite M{\"o}glichkeit: Revolution! Eine v{\"o}llig neue Konstruktion der Analytik muss angefertigt werden, welche die Realit{\"a}t besser wiedergibt als die vergangene Theorie. 
 Nicht zu vergessen das Kriterium der Falsifizierbarkeit: Widerlegung einer Theorie hat ein st{\"a}rkeres Gewicht als ihre Best{\"a}tigung.\footnote[11]{\citep*[S.161]{bombach1981theorie}}Verdeutlichen wir uns das an den {\"o}konomischen Theorien der Klassiker und Neoklassiker. 
     
Ihre Theorien haben weit {\"u}ber ein Jahrhundert das {\"o}konomische Geschehen gepr{\"a}gt. Und auch heute noch gepr{\"a}gen ihre Wertvorstellungen und Ideen in einem gewissen Rahmen die von uns zu betrachtenden Str{\"o}mungen.
Die Klassiker beschw{\"o}den die Kr{\"a}fte des freien Marktes.  Adam Smith setzte dabei mit dem 1776 ver{\"o}ffentlichten Werk "`Wohlstand der Nationen"' den Anfang.

Eine unsichtbaren Hand\footnote[12]{\citep*[S.137]{bombach1981theorie}} erkl{\"a}rt die Wirkkr{\"a}fte der freien Marktwirtschaft. Wirtschaftssubjekte passen sich rasch ver{\"a}nderden Bedingungen wie  Zinsen, Preisen und L{\"o}hnen an. Aus einer Vielzahl von Angeboten filtern rational handelnde Individuen das Beste heraus.

In einem Ungleichgewicht geratene Preise reagieren schneller als realisierte Angebots- und Nachfragemengen und f{\"u}hren rasch zu einem Gleichgewicht zur{\"u}ck.\footnote[13]{\citep*[S.291]{Basseler2010}} Neoklassiker befassten sich vornehmlich damit, mit Realpreisen\footnote[14]{Die Preise von G{\"u}tern kommen nie unabh{\"a}ngig von Preisen anderer G{\"u}ter zustande} zu argumentieren.\footnote[15]{\citep*[S.138]{bombach1981theorie}} Die Preise k{\"o}nnen bequemer zugeteilt, identifiziert und bewertet werden da sie eine (be-)rechenbare Einheit erhalten. Diese Einheit, ob nun US-Dollar, Yen oder Euro stellt das Geld ganz allgemein dar. Dabei l{\"a}sst sich das Geld in drei Funktionen unterteilen: Tauschmittel, Recheneinheit und Wertaufbewahrungsfunktion.\footnote[16]{\citep*[S.417]{Basseler2010}}

Die Klassiker haben die Wertaufbewahrungsfunktion des Geldes jedoch gr{\"o}{"s}tenteils vernachl{\"a}ssigt, weil es der {\"o}konomischen Rationalit{\"a}t widerspricht.\footnote[17]{\citep*[S.53]{bombach1981theorie}} Geld als Tauschmittel und Recheneinheit macht aus einer geschlo{"s}enen Gesellschaft\footnote[18]{In dem Fall ist mit "`geschlo{"s}ener Gesellschaft"' eine nur aus Selbstversorgungseinheiten bestehende Wirtschaft, in der kein G{\"u}tertausch stattfindet, gemeint.} eine Tauschwirtschaft.
Doch in Anbetracht seiner Wertaufberahrungsfunktion hat Geld eine Wirkung, die Geldmenge wird zu einem analytischen Konzept, die Wirkung des Geldes wird als Zielgr{\"o}{"s}e der Geldpolitik herangezogen.\footnote[19]{\citep*[S.421]{Basseler2010}} 
      
Die Forderung der Klassiker war individuelle Selbstst{\"a}ndigkeit und Freiheit. Der marktwirtschaftlichen Wettbewerbsmechanismus erschien ihnen am besten geeignet um die Interessen der Produzenten mit denen der Konsumenten in Einklang zu bringen. Dabei ordneten sie dem Staat eine untergeordnete Rolle zu. Seine Aufgabe bestand "`lediglich"' aus der Verwaltung und der Aufrechterhaltung der Rechtsordnung, f{\"u}r die innere und {\"a}u{"s}ere Sicherheit, sowie das Verkehrswesen und das Bildungs- und Gesundheitswesen zu sorgen.\footnote[20]{\citep*[S.60f]{Basseler2010}}
           
           
             


\subsection{Beschreibung des eigenen Verst{\"a}ndnis von Geldpolitik}
 %% Autor: Gregor May %%

Nach Ba{"s}eler hat Geldpolitik die Hauptaufgabe, eine optimale Geldversorgung der Wirtschaft zu gew{\"a}hrleisten.\footnote[21]{\citep*[vgl.][S. 551]{Basseler2010}} Geldpolitik umfasst dabei die Steuerung, Regulierung und Gew{\"a}hrleistung von Gesch{\"a}ften zwischen Kreditbanken, Gesch{\"a}ftsbanken und den Zentralbanken.
 "`Die Akteure im Finanzbereich werden allgemein Finanzintermedi{\"a}re genannt. [Diese] vermitteln Finanzprodukte zwischen den Anbietern und Nachfragern. (\ldots) Dies sind vor allem Banken und Kapitalanlagegesellschaften, die selbst Finanzprodukte kreieren sowie institutionelle Anleger."'\footnote[22]{\citep*[vgl.][S.511f]{Basseler2010}}

In der Realit{\"a}t wird die Aufgabe von einer gr{\"o}{"s}tenteils staatlichen, jedoch unabh{\"a}ngigen Zentralbank {\"u}bernommen. Weit verbreitet herrscht Konsens dar{\"u}ber, dass Geldpolitik staatliche Aufgabe bleiben mu{"s}, obwohl Ideen einer dezentralen, dem Wettbewerb unterliegenden Geldversorgung durch private Gesch{\"a}ftsbanken und dar{\"u}ber hinaus ein System konkurierender Parallelw{\"a}hrungen kursieren.

Die Zentralbank hat konkret die Aufgabe f�r die Durchf{\"u}hrung einer Geldpolitik mit dem Ziel einer angemessenen Begrenzung der Geldmenge zu sorgen, die Ausgabe eines gesetzlichen Zahlungsmittels (Staatliches Emissionsmonopol) zu steuern, einen reibungslosen Zahlungs- und Kreditverkehrs als "`Bank der Banken"' zu gew{\"a}hrleisten, sowie die Geldwertstabilit{\"a}t zu wahren und eine ausreichende Menge an Geld f{\"u}r Krisenzeiten bereitzustellen.

"`Zentrale Zielgr{\"o}{"s}e der Geldpolitik ist die Geldmenge M3. (\ldots)Dabei ist zu beachten, dass das herk{\"o}mmliche Konzept von Banken zum Konzept der \ac{MFIs}\footnote[23]{MFIs sind im Wesentlichen Zentralbanken, Kreditinstitute, sowie Geldmarktfonds.\citep*[S.508]{Basseler2010}} erweitert worden ist."'\footnote[24]{\citep*[S. 507]{Basseler2010}}

Die gr{\"o}{"s}te Rolle innerhalb der Geldmenge M3 spielen Bargeldumlauf und t{\"a}glich f{\"a}llige Einlagen. Ebenfalls quantitativ bedeutsam sind Einlagen mit vereinbarterter K{\"u}ndigungsfrist. Sonstige Komponenten machen insgesamt nur knapp 20 Prozent der Geldmenge M3 aus.\footnote[25]{\citep*[S.508]{Basseler2010}}

"`Gesch{\"a}ftsbanken (oder auch Kreditinstitute) sind die zentralen Akteure im Finanzbereich einer Volkswirtschaft. Die erste zentrale Funktion von Gesch{\"a}ftsbanken (kurz: Banken) ist die Abwicklung des Zahlungsverkehrs einer Volkswirtschaft. (\ldots)Die zweite zentrale Funktion von Banken ist die Organisation und Durchf{\"u}hrung des Kreditverkehrs einer Volkswirtschaft."'\footnote[26]{\citep*[vgl.][S.512-13]{Basseler2010}}

"`Die Organisation des Kreditverkehrs ist ein klassisches Gesch{\"a}ft der Banken: Sie beschaffen Geld, sie verleihen Geld und sie versuchen, dieses Geld mit Gewinn wieder zur{\"u}ckzubekommen.(\ldots)Die Kreditgew{\"a}hrung war neben der Abwicklung des Zahlungsverkehrs die klassische Aufgabe der Gesch{\"a}ftsbank."'
 "`Daneben gibt es weitere Aufgaben der Banken, vor allem die Verm{\"o}gensverwaltung der Kunden, die Ausgabe und den Handel mit Wertpapieren, die Beratung und Unterst{\"u}tzung bei Unternehmenszusammenschl{\"u}ssen oder die Unterst{\"u}tzung von Unternehmen bei ihrer Kapitalaufnahme, etwa bei B{\"o}rseng{\"a}ngen. Dies wird zusammenfassend Investmentbanking bezeichnet. (\ldots) In diesem Segment des Bankengesch{\"a}fts  werden auf Zertifikate entwickelt und verkauft oder Fonds aufgelegt und Fondsanteile verkauft. In Kontinentaleuropa ist (\ldots)das Universalbankensystem etabliert(\ldots). Sparkassen {\"u}bernehmen (\ldots)die Abwicklung des Zahlungsverkehrs und des "`kleinen"' Kreditverkehrs, kleine Privatbanken {\"u}bernehmen eher die Funktionen des Investmentbankings und gro{"s}e Universalbanken {\"u}bernehmen alle Gesch{\"a}ftssparten."'\footnote[27]{\citep*[vgl.][S.512-13]{Basseler2010}}

"`Das Eigenkapital [einer Bank] setzt sich konkret zusammen aus dem Grundkapital, den Kapitalr{\"u}cklagen und den Gewinnr{\"u}cklagen (einbehaltene Gewinne) sowie einer stillen Einlage des Finanzmarktstabilisierungsfonds (\ldots). Grunds{\"a}tzlich ist das Eigenkapital der Banken von zentraler Bedeutung. Es ist letztlich das Kapital, das die Bank zum Ausgleich von Verlusten aus ihrem Kredit- und Investmentgesch{\"a}ft einsetzen kann.  Daher sind in der Bankenaufsicht bestimmte Mindestanforderungen an die H{\"o}he des haftenden Eigenkapitals (\ldots)vorgesehen"'\footnote[28]{\citep*[vgl.][S.515]{Basseler2010}}
 
"`Mit dem(\ldots) 01.01.1999 ist (\ldots)die Europ{\"a}ische Zentralbank (EZB) die zentrale Institution, also die Zentralbank, f{\"u}r die Festlegung und Ausf{\"u}hrung der Geldpolik. Daneben existiert weiterhin die Deutsche Bundesbank, die als Zentralbank der Bundesrepublik Deutschland Teil des Europ{\"a}ischen Systems der Zentralbanken ist. Sie ist (\ldots)ausf{\"u}hrendes Organ der geldpolitischen Entscheidungen der (\ldots)EZB."'\footnote[29]{\citep*[vgl.][S.512]{Basseler2010} }
\clearpage


\section{Das technisches System, Organisationen, Ziele und Instrumente der Europ{\"a}ischen Geldpolitk}
  \label{sec2:technischesSystem}

\subsection{Organisationen der Europ{\"a}ischen Geldpolitk}
%%Autor:Gregor May %%

In der Europ{\"a}ischen Union (EU) wird die Geldpolitik durch die Europ{\"a}ische Zentralbank (EZB) und das (\ac{ESZB}) organisiert. Dabei umfasst das ESZB alle 28 nationalen Zentralbanken (NZBen) der Mitgliedsstaaten der\ac{EU}sowie die Europ{\"a}ische Zentralbank. Zuletzt wurden Estland und Lettland in die Eurozone aufgenommen.\footnote[30]{Die Eurozone besteht derzeit aus 18 EU-Staaten und hat daher auch den Beinamen "`Euro-18"' erhalten.} Sonderstatus im ESZB haben dabei die sogenannten "`Outs"', jene Mitgliedsstaaten der EU, die den Euro nicht eingef{\"u}hrt haben.
Dies umfasst derzeit: D{\"a}nemark, Gro{"s}britanien, Schweden sowie die meisten neuen EU-Mitgliedsstaaten nach 2001. Sie sind vom Entscheidungsprozess der ESZB ausgeschlossen und vollziehen eine eigenst{\"a}ndige nationale Geldpolitik.
Formal unterscheidet der EG-Vertrag\footnote[31]{\ac{EGV}, Art. 105-109 d} zwischen EZB und ESZB, faktisch entscheidet jedoch nur eine Institution: die EZB mit ihren Beschlussorganen.\footnote[32]{\citep*[vgl.][S.553]{Basseler2010}}

Geldarten und Geldsch{\"o}pfung sind durch die EZB und das ESZB bei einer supranationalen Institution innerhalb des Eurosystems monopolisiert.

\subsection{Die Europ{\"a}ische Zentralbank}
 %% Autor: Gregor May %%
Als Beschlussorgane leiten der EZB-Rat und das Direktorium die Europ{\"a}ischen Zentralbank. Das Direktorium der \ac{EZB} wird aus dem Pr{\"a}sidenten und dem Vizepr{\"a}sidenten der EZB, sowie weiteren vier Mitgliedern gebildet. Diese vier Mitglieder werden einvernehmlich von den Regierungen der Mitgliedsstaaten auf Ebene von Staats- und Regierungschefs ernannt. Dem EU-Rat steht hierbei ein Empfehlungsrecht zu. Erweitert besteht beim Europ{\"a}ischem Parlament und beim EZB-Rat ein Anh{\"o}rungsrecht.\footnote[33]{\citep*[vgl.][S.553]{Basseler2010}}

Der EZB-Rat besteht aus dem Direktorium und den Pr{\"a}sidenten aller nationalen Zentralbanken, die den Euro gemeinsam eingef{\"u}hrt haben. Innerhalb der Europ{\"a}ischen Zentralbank liegt die exekutive Gewalt beim Direktorium, welches "`f{\"u}r die Durchf{\"u}hrung der Geldpolitik nach den Leitlinien und Beschl{\"u}ssen des EZB-Rates verantwortlich ist."'\footnote[34]{\citep*[S.553]{Basseler2010}}

Das Direktorium der \ac{EZB} ist gegen{\"u}ber den nationalen Zentralbanken des Eurosystems weisungsbefugt. Somit entsteht ein duales System, bestehend aus dem Exekutivorgan der EZB in Form des Direktoriums und ein Beschlussorgan in Form des EZB-Rates.

Beschl{\"u}sse der gemeinschaftlichen europ{\"a}ischen Geldpolitik des Euroraumes werden vom EZB-Rat erarbeitet und erlassen um die Ausgabe von M{\"u}nzen und Banknoten zu regeln oder um die vom \ac{ESZB} {\"u}bertragenen Aufgaben zu erf{\"u}llen.\footnote[35]{\citep*[vgl.][S.553]{Basseler2010}}
Derzeit umfasst der EZB-Rat 18 L{\"a}nder und das Direktorium. Abgestimmt wird mit einfacher Mehrheit der Anwesenden, bei Stimmengleichheit entscheidet die Stimme des Pr{\"a}sidenten.


\subsection{Ziele und Aufgaben von ESZB und EZB}  %% Autor: Gregor May %%
Die EG-Vertr{\"a}ge legen als vorrangiges Ziel des \ac{ESZB} und damit der EZB "`die Gew{\"a}hrleistung der Preisstabilit{\"a}t(\ldots)[fest]. (\ldots)Soweit dies ohne Beeintr{\"a}chtigung (\ldots)m{\"o}glich ist, unterst{\"u}tzt das ESZB die allgemeine Wirtschaftspolitik der Gemeinschaft, um zur Verwirklichung der in Artikel 2 festgelegten Ziele der Gemeinschaft beizutragen. Das ESZB handelt im Einklang mit dem Grundsatz einer offenen Marktwirtschaft und mit freien Wettbewerb\ldots"'\footnote[36]{\citep*[vgl.][S.554]{Basseler2010}} Hier wird von der EZB eine Priorisierung der Preisstabilit{\"a}t gegen{\"u}ber anderen Zielen wie Vollbesch{\"a}ftigung und Wachstum festgeschrieben.

Ideologische Grundlage f{\"u}r diese Priorisierung ist die von monetaristischen Str{\"o}mungen ausgehende Vorstellung, dass eine Zentralbank zu allererst die Verantwortung der Preisstabilit{\"a}tssicherung besitzt. Andere Akteure, wie Gewerkschaften und Tarifparteien, sind f{\"u}r die Vollbesch{\"a}ftigung verantwortlich. Wirtschaftswachstum ergibt sich aus technischem Fortschritt und Bev{\"o}lkerungswachstum.\footnote[37]{ebd.}


Die Aufgaben des \ac{ESZB} werden in Art. 105, Abs. 2\ac{EGV} wie folgt festgelegt: die Geldpolitik der Gemeinschaft festzulegen und umsetzen, Divisengesch{\"a}fte im Einklang mit Artikel 111\footnote[38]{Dieser Artikel legt fest, dass alle Entscheidungen {\"u}ber Wechselkurssysteme dem Ministerrat vorbehalten sind und dementsprechend nicht durch die EZB gelenkt werden kann.} durchzuf{\"u}hren, die offiziellen W{\"a}hrungsreserven der Mitgliedstaaten zu halten und zu verwalten, das reibungslose Funktionieren des Zahlungssystems zu f{\"o}rdern\footnote[39]{\citep*[vgl.][S.555]{Basseler2010}}





\subsubsection{Unabh{\"a}ngigkeit der EZB}  %% Autor: Gregor May %%

Gew{\"a}hrleistung der Preisstabilit{\"a}t ist als vorrangiges Ziel der EZB festgeschrieben. Da die EZB eine gesamteurop{\"a}ische Geldpolitik organisieren soll, steht eine Unabh{\"a}ngigkeit von nationalen Interessen einzelner Regierungen als unabdingbar im Raum. Einfachheitshalber sei festgestellt, dass die EZB auf verschiedene Arten als unabh{\"a}ngig gelten darf.\footnote[40]{\citep*[vgl.][S.555-557]{Basseler2010}}

Zum Ersten ist die EZB funktional relativ unabh{\"a}ngig, da sie Weisungen nicht entgegennehmen darf.\footnote[41]{vgl. Artikel 107, \ac{EGV} (Unabh{\"a}ngigkeit der EZB) } Die Unabh{\"a}ngigkeit, keinerlei Kontrollen durch Regierungen und Parlamente zu unterliegen, ist im europ{\"a}ischen Raum einzigartig. Sie wird nur durch die Verpflichtung beschr{\"a}nkt eine allgemeine Wirtschaftspolitik der Gemeinschaft zu unterst{\"u}tzen, soferndas Ziel der Preisstabilit{\"a}t nicht beeintr{\"a}chtigt wird.

Zum Zweiten ist die EZB dar{\"u}ber hinaus personell unabh{\"a}ngig, allein durch die Ernennung von Pr{\"a}sidenten der Nationalbanken k{\"o}nnen einzelne Regierungen Einfluss aus{\"u}ben. Zum Dritten ist die EZB finanziell unabh{\"a}ngig --- sie verf{\"u}gt {\"u}ber eigene Einnahmen und einen eigenen Haushalt --- und besitzt die Kontrolle {\"u}ber alle Instrumente der Geldpolitik.

\subsection{Instrumente der Europ{\"a}ischen Geldpolitik}  %% Autor: Gregor May %%

Im weiteren Verlauf wollen wir kurz die geldpolitischen Instrumente erl{\"a}ufern, die der Europ{\"a}ischen Zentralbank zur Verf{\"u}gung stehen.



\subsubsection{Offenmarktpolitik der EZB}  %% Autor: Gregor May %%

Zentrales Instrument f{\"u}r die Geldversorgung einer Wirtschaft ist die Offenmarktpolitik. Man versteht hierunter den An- und Verkauf von Wertpapieren gegen Zentralbankgeld durch die Zentralbank.
Die Zentralbank kann durch Offenmarktk{\"a}ufe bzw. -verk{\"a}ufe die Zentralbankgeldsch{\"o}pfung steuern.

F{\"u}r die EZB im \ac{ESZB} sind nur finanziell solide \ac{MFIs}, sprich Finanzinstitute, die in das Mindestreservesystem einbezogen sind, als Gesch{\"a}ftspartner der \ac{EZB} zugelassen. Es wird allgemein zwischen der expansiven (Kauf von Wertpapieren) und kontraktiven Offenmarktpolitik (Verkauf von Wertpapieren) unterschieden.\footnote[42]{\citep*[S.556]{Basseler2010}}

Nach einem Kauf von Wertpapieren hat sich der Bestand an Zentralbankgeld der Gesch{\"a}ftsbanken erh{\"o}ht. Diese erh{\"o}hte Geldbasis erm{\"o}glicht nun den Gesch{\"a}ftsbanken Kredite an Nichtbanken weiterzugeben.\footnote[43]{\citep*[S.557]{Basseler2010}}

Die Zentralbank kann Gesch{\"a}ftsbanken nicht zur Offenmarktpolitik zwingen, sondern muss entsprechende Anreize bieten. Bei expansiver Offenmarktpolitik bestehen diese in niedrigen Zinss{\"a}tzen f{\"u}r die Zentralbankgeldkreditgew{\"a}hrung unterhalb des Geldmarktzinses, bei kontraktiver Offenmarktpolitik in Zinss{\"a}tzen {\"u}ber dem Geldmarktzins.
Zeitlich befristete Offenmarktgesch{\"a}fte werden auch Wertpapierpensionsgesch{\"a}fte oder Repo-Gesch{\"a}fte genannt.\footnote[44]{Repo-Gesch{\"a}fte leitet sich vom englischen "`Repurchase"', zu deutsch R{\"u}ckkauf, ab} Der Zinssatz wird hier Pensionssatz genannt.\footnote[45]{Entsprechend hei{"s}t der Pensionssatz auch Repo-Rate}
Durch Repo-Gesch{\"a}ften l{\"a}sst sich die Zentralbankgeldmenge durch die Zentralbank recht einfach steuern. Wenn zeitlich begrenzte Gesch{\"a}fte auslaufen und nicht erneuert werden, treten automatisch kontraktive Effekte auf.


Das Bankensystem im Eurogebiet ist auf die Bereitstellung von Zentralbankgeld durch die EZB angewiesen, weshalb die Offenmarktpolitik der EZB am Interbankengeldmarkt\footnote[46]{\citep*[vgl.][S.558f]{Basseler2010}} ansetzt. Dementsprechend greift die EZB auf folgende Instrumente zur Steuerung der Offenmarktpolitik zur{\"u}ck:
\begin{enumerate}
 \item{Hauptrefinanzierungsinstrument}
 \item{l{\"a}ngerfristige Refinanzierungsgesch{\"a}fte}
 \item{Feinsteuerungsoption}
 \item{Strukturelle Operationen}
 \end{enumerate}

Das Hauptrefinanzierungsinstrument und das l{\"a}ngerfristige Refinanzierungsgesch{\"a}ft sind regelm{\"a}{"s}ig stattfindende, dem Interbankenmarkt Liquidit{\"a}t zuf{\"u}hrende Repo-Gesch{\"a}fte.
F{\"u}r das Hauptrefinanzierungsinstrument sind die w{\"o}chentlichen Transaktionen auf eine Woche G{\"u}ltigkeit begrenzt. Das Hauptrefinanzierungsinstrument steuert die Zinss{\"a}tze (Zentraler Leitzins der EZB) sowie die Liquidit{\"a}t am europ{\"a}ischen Geldmarkt. Die Refinanzierungsgesch{\"a}fte finden mit monatlichem Abstand statt und haben  Laufzeiten von mehreren Monaten bis hin zu einem Jahr.

Sollte es zu unerwarteten marktm{\"a}{"s}igen Liquidit{\"a}tsschwankungen auf die Zinss{\"a}tze kommen, stehen immer noch die Feinsteuerungsoperationen zur Verf{\"u}gung. Diese werden entweder mit befristeten Transaktionen oder in Form von definitiven  K{\"a}ufen oder Verk{\"a}ufen von Wertpapieren ausgef{\"u}hrt.

Der EZB stehen als letztes Instrument der Offenmarktpolitik die Strukturellen Operationen zur Verf{\"u}gung. Sie haben das Ziel die grundlegenden Liquidit{\"a}tspositionen des Finanzsektors zu beeinflussen. Dies passiert {\"u}ber die Emission von Schuldverschreibungen, definitive K{\"a}ufen oder Verk{\"a}ufe und befristete Transaktionen.\footnote[47]{\citep*[S.560]{Basseler2010}}

\subsubsection{St{\"a}ndige Fazilit{\"a}ten}  %% Autor: Gregor May %%

Will die EZB eine genaue Steuerung des Geldmarktzinssatzes, so greift sie auf das Instrument der st{\"a}ndigen Fazilit{\"a}t zur{\"u}ck. Hierunter versteht man die Bereitstellung bzw. Absch{\"o}pfung von Liquidit{\"a}t jeweils bis zum n{\"a}chsten Gesch{\"a}ftstag in Form von Tageskrediten oder t{\"a}glichen Anlagen {\"u}bersch{\"u}ssiger Liqudit{\"a}ten. Im Unterschied zur Offenmarktpolitik erfolgt die Inanspruchnahme der st{\"a}ndigen Fazilit{\"a}ten auf Initiative der Banken und ist grunds{\"a}tzlich unbeschr{\"a}nkt m{\"o}glich. Bislang werden sie jedoch nur in geringem Umfang genutzt, da die Konditionen im Vergleich zu denen am Interbankenmarkt ung{\"u}nstiger sind.\footnote[48]{\citep*[vgl.][S.560ff]{Basseler2010}}

Gesch{\"a}ftsbanken k{\"o}nnen zur Deckung eines vor{\"u}bergehenden Liquidit{\"a}tsbedarfs unbegrenzt die Spitzenrefinanzierungsfazilit{\"a}t in Anspruch nehmen, so sie von von der EZB als Gesch{\"a}ftspartner zugelassen sind. Dieser Bedarf wird mit einem im Voraus bekanntgegebenen Kreditzinssatz verzinst, welcher die Obergrenze des allgemeinen Tagesgeldsatzes am Geldmarkt ist.

Gesch{\"a}ftsbanken mit {\"u}bersch{\"u}ssiger Liquidit{\"a}t  k{\"o}nnen auf Einlagefazilit{\"a}ten zur{\"u}ckgreifen und die Liquidit{\"a}t bis zum n{\"a}chsten Gesch{\"a}ftstag bei den nationalen Zentralbanken anlegen. Diese Einlage wird mit einem im Voraus bekanntgegebenen Zinssatz verzinst, welcher die Untergrenze des allgemeinen Tagesgeldsatzes am Geldmarkt ist.

Gew{\"o}hnlicherweise bewegt sich der Leitzins der EZB zwischen dem Zinssatz f{\"u}r die Einlagefazilit{\"a}t und der Spitzenrefinanzierungsfazilit{\"a}t. Ein Zinssatz von weniger als 1 Prozent f{\"u}r die Hauptrefinanzierung stellt einen bisher unerreichten historischen Tiefpunkt f{\"u}r die EZB dar.
Dass die EZB Wert darauf legt, den Geldmarktzinssatz genau zu steuern, zeigt, dass auch keynesianische Elemente in die Geldpolitik der EZB einflie{"s}en.\footnote[49]{\citep*[vgl.][S.562f]{Basseler2010}}

\subsubsection{Mindestreservepolitik}   %% Autor: Gregor May %%
In der Europ{\"a}ischen Geldpolitik ist den Gesch{\"a}ftsbanken vorgeschrieben bei der Zentralbank einen bestimmten Prozentsatz ihrer Einlagen, den Mindestreservesatz, als Sichtguthaben vor zu halten. Die Mindestreservepolitik\footnote[50]{\citep*[vgl.][S.562f]{Basseler2010}} soll einen stark wirkenden Einfluss auf die Geldsch{\"o}pfung bzw. das Geldsch{\"o}pfungspotential der EZB erhalten. Gibt die EZB eine Erh{\"o}hung des Mindestreservesatz vor, nimmt das Geldsch{\"o}pfungspotential zu, bei Senkung nimmt es ab.
Ein gewollter Nebeneffekt ist, dass bei einer {\"a}nderung des Mindestreservesatzes auch die Liqudit{\"a}tsreserven der Gesch{\"a}ftsbanken sich ver{\"a}ndern.
Die Mindestreservepolitik schafft so einen stabilen zus{\"a}tzlichen Zentralbankgeldbedarf. Dies stellt die direkte Verbindung zwischen Mindestreserve und Geldsch{\"o}pfung her.

Mindestreserven m{\"u}ssen durch die Kreditinstitute nur im Monatsdurchschnitt gehalten werden. Der Zinssatz f{\"u}r das Hauptrefinanzierungsgesch{\"a}ft wird f{\"u}r angelegte Mindestreserven verwendet.

\subsubsection{Geldpolitische Strategien in Europa}
%% Autor: Gregor May %%

Die geldpolitische Strategie des Eurosystems wurde vom Rat der Europ{\"a}ischen Zentralbanken entwickelt und am 13.10.1998 der {\"o}ffentlichkeit pr{\"a}sentiert worden.
Zentrale Elemente sind seitdem:
\begin{itemize}
 \item Das Hauptziel der Preisstabilit{\"a}t, definiert als Anstieg des Harmonisierten Verbraucherpreisindex (HVPI) f{\"u}r den Euroraum von unter 2 Prozent.\footnote[51]{Seit Einf{\"u}hrung des Euros ist diese Marke  jedes Jahr {\"u}berschritten worden, gerade in der Finanzkrise mit mehr als 1 Prozent j{\"a}hrlich.} Allerdings muss die Preisstabilit{\"a}t nur mittelfristig gew{\"a}hrleistet werden.
 \item Eine Geldmengenpolitik mit der Verk{\"u}ndung eines Referenzwertes f{\"u}r das Wachstum der Geldmenge M3.
 \item Beobachtung und Einsch{\"a}tzung der k{\"u}nftigen Preisentwicklung sowie der Preisstabilit{\"a}t des Euroraumes insgesamt.
\end{itemize}

Das Ziel der Preisstabili{\"a}t ist 2003 dahingehend pr{\"a}zisiert worden, dass mittelfristig eine Preissteigerungsrate unter, aber ann{\"a}hernd, 2 Prozent gegen{\"u}ber dem Vorjahr eingehalten werden muss. Dadurch wird versucht, deflation{\"a}re Geldpolitik seitens der Nationalen Zentralbanken zu unterbinden.\footnote[52]{\citep*[vgl.][S.564-568]{Basseler2010}}

Analyse und Beurteilung zuk{\"u}nftiger Preisentwicklungen nimmt eine zentrale Rolle in der Arbeit der EZB ein. Diese Arbeit wird als  "`Zwei-S{\"a}ulen-Strategie"' umschrieben.\footnote[53]{\citep*[S.568]{Basseler2010}}

Sie umfasst eine Analyse kurz- und mittelfristiger Preisentwicklungen durch realwirtschaftliche Bestimmungsfaktoren, sowie eine monet{\"a}re Analyse der Entwicklung von Geldmengen, vor allem der Geldmenge M3, als mittel- bis langfristiger Faktor der Preisentwicklung.

W{\"a}hrend sich die wirtschaftliche Analyse mit dem Zusammenspiel von Angebot- und Nachfrageentwicklung an den G{\"u}ter-, Dienstleistungs- und Faktorm{\"a}rkten zur Vermeidung von Inflation sehr keynesianisch gepr{\"a}gt ist, so wird die Geldmengenanalyse aus monetaristischer Sichtweise betrieben\.footnote[54]{\citep*[S.568]{Basseler2010}} Somit l{\"a}sst sich die europ{\"a}ische Geldpolitik als ein Kompromiss, eine Mischung aus unterschiedlichen geldpolitischen Ansichten und geldtheorietischen Positionen verstehen. Eine vereinigte Geldpolitik Europas.\footnote[55]{\citep*[vgl.][S.558f]{Basseler2010}}

\clearpage







\section{Einblick in die Grundgedanken der {\"O}konomen des 19. und 20. Jahrhunderts}
  \label{sec3:stroemungen}
\subsection{{\"U}bersicht der zu behandelnden Str{\"o}mungen} %% Autor: Gregor May %%

Im n{\"a}chsten Teil m{\"o}chten wir uns nun mehr 3 {\"o}konomischen Str{\"o}mungen und ihren Ideen zur Geldpolitik widmen. Beginnen wollen wir mit John Maynard Keynes, welcher eine Steuerung der Geldpolitik durch Eingreifen des Staates vorsah. Friedrich August von Hayek widerspricht dieser keynessianischen Str{\"o}mung und fordert eine g{\"a}nzliche Entstaatlichung der Geldpolitik bis hin zur Privatisierung der Umlaufmittel und begr{\"u}ndet die {\"o}sterreichische Schule. Milton Friedman setzt auf die optimale Geldmenge zur L{\"o}sing der wirtschaftlichen Probleme. Er gilt als ein Mitbegr{\"u}nder der Monetaristen.


\subsection{Einblick in die Grundgedanken der keynessianischen Schule (John Maynard Keynes)} %% Autor: Marius Hanniske %%

Auf den Hintergrund der Weltwirtschaftskrise der Jahre 1929 bis 1932 entwickelte Lord John Maynard Keynes\footnote[56]{geboren: 5. Juni 1883 in Cambridge; gestorben: 21. April 1946 in Tilton, im weiteren Keynes} seine Konzeption zu seinem Hauptwerk \frqq The General Theory of Employment, Interest and Money\flqq. Nahezu s{\"a}mtliche Investitionst{\"a}tigkeiten kamen in dieser Krise zum Erliegen. Das f{\"u}hrte zu einer bis dahin nicht gekannten Massenarbeitslosigkeit.

Fehler die gemacht wurden\footnote[57]{z.B. vergr{\"o}{"s}erte die Landwirtschaft w{\"a}hrend des 1. Weltkriegs seine Kapazit{\"a}ten nach dem Abbruch der internationalen Handelsbeziehungen und litt nach Beendigung des Krieges an {\"u}berproduktion die sie nicht abbaute. So fielen in den Jahren vor der Krise die Preise f{\"u}r Anbauprodukte ins bodenlose. Weiterer Preis- und Lohnverfall waren die Folge und damit ein Anstieg der Arbeitslosigkeit. \citep*[vgl.][S.14ff]{bombach1981theorie}} sind eindeutig den falschen Annahmen der Klassiker zuzuschreiben:\footnote[58]{\citep*[S.36]{Keynes2011}} "`das sich bei freier Konkurrenz einpendelnde Preis-, Lohn- und Zinsniveau f{\"u}hre stets zu Vollbesch{\"a}ftigung der Produktionsfaktoren."'

Das Say'sche Theorem besagt, dass das Angebot seine Nachfrage schafft, indem alle produzierten G{\"u}ter mit dem im Produktionsprozess verdienten Einkommen aufgekauft werden. Nach Keynes nimmt jedoch der Hang zum Verbrauch bei zunehmenden Einkommen relativ ab, sodass Say's Theorem mehr oder weniger unerf{\"u}llt bleibt. Es wird gespart, wodurch ein Nachfrager{\"u}ckgang entsteht. Die sinkende Nachfrage l{\"a}sst die Absatzerwartungen zur{\"u}ck gehen und dies kann dann zu Arbeitslosigkeit f{\"u}hren.\footnote[59]{\citep*[S.203]{peters2000}} 

Keynes bem{\"a}ngelte an der klassischen Theorie, dass sie die Vollbesch{\"a}ftigung aller Produktionsfaktoren voraussetze, was ja offensichtlich in der Weltwirtschaftskrise nicht zutraf. Den Hauptgrund f{\"u}r die hohe Arbeitslosigkeit in der Volkswirtschaft sieht Keynes in der unzureichenden Nachfrage die sich aus {\"u}bersch{\"u}ssiger Ersparnis ergibt. Somit wird die Nachfrage zur strategischen Gr{\"o}{"s}e, die durch staatliche Nachfrageimpulse stimuliert werden soll, wenn die Wirtschaftst{\"a}tigkeit zu erlahmen droht. Obwohl Keynes in bestimmten Situationen die Investition positiv vom Zins beeinflusst sieht, h{\"a}lt er den Einfluss des Zinssatzes bei weitem nicht f{\"u}r ausreichend um eine optimale Investitionsrate zu erzielen.\footnote[60]{\citep*[S.208]{peters2000}}
Keynes argumentiert, dass die Investitionsnachfrage praktisch nicht auf eine {\"A}nderung des Zinssatzes reagiert und wenn es doch so w{\"a}re, w{\"u}rde das Systemgleichgewicht
ein Zinsniveau, dass n{\"o}tige Investitionen zur Vollbesch{\"a}ftigung lohnend machen w{\"u}rde,
nicht zu.\footnote[61]{ \citep*[S.174]{bombach1981theorie}}

Man kann also zusammenfassen, dass in Keynes Wirtschaftspolitik die Besch{\"a}ftigung die prim{\"a}re Zielgr{\"o}{"s}e ist, die angestrebt werden soll und demnach die Fiskalpolitik
eine starke Wirkung hat, st{\"a}rker als die Geldpolitik, die nur indirekt wirke.\footnote[62]{\citep*[S.181]{bombach1981theorie}}
Das wird auch nochmal in dem folgenden Zitat von John Maynard Keynes deutlich: "`Die Bedeutung des Geldes liegt allein in seiner Kaufkraft. Eine Ver{\"a}nderung in der M{\"u}nzeinheit(\ldots), hat (\ldots)keine Auswirkungen."'\footnote[63]{\citep*[S.1]{Keynes1997}}

\subsection{Einblick in die Grundgedanken der {\"o}sterreichischen Schule (Friedrich August von Hayek)} %% Autor: Gregor May %%

Die Ideen von Hayek zur Geldpolitik setzen an einem ganz anderen Punkt an. Ihm geht es darum, das staatliche Monopol zu brechen, welches rund um das Thema Geld und Geldpolitik existiert. Denn
"`es kann eine wirkliche Gefahr f{\"u}r die Freiheit werden, wenn ein zu gro{"s}er Teil der Wirtschaftst{\"a}tigkeit direkt in die H{\"a}nde des Staates ger{\"a}t. Doch was hier abzulehnen ist, ist nicht das Staatsunternehmen als solches, sondern das Staatsmonopol."'\footnote[64]{\citep*[S.290]{hayek1971}}

Zwar macht Hayek das Zugest{\"a}ndnis, dass es "`einer Regierung (\ldots) nat{\"u}rlich freistehen [muss], dar{\"u}ber zu entscheiden, in welchem Zahlungsmittel die Steuern zu entrichten sind, und sie muss Vertr{\"a}ge in jedem beliebigen Zahlungsmittel abschlie{"s}en k{\"o}nnen (wodurch sie ein von ihr ausgegebenes Zahlungsmittel beg{\"u}nstigen kann)."'\footnote[65]{\citep*[S.23]{Hayek1977}}

Aber wie genau diese Zahlungsmittel aussehen, dar{\"u}ber soll der Staat keine Entscheidungsgewalt haben.

Denn "`dass es grunds{\"a}tzlich m{\"o}glich ist (\ldots) in einer Marktwirtschaft, dass der Staat die Verantwortung f{\"u}r die Kapitalbildung und ihre Lenkung {\"u}bernimmt, ist unbestreitbar. Aber ist es auch zweckm{\"a}{"s}ig und w{\"u}nschenswert?"'\footnote[66]{\citep*[S.22]{Hayek1969}}
Auch h{\"a}lt Hayek es f{\"u}r wichtig, zu erw{\"a}hnen, dass "`Die Regierungen haben in der Handhabung der W{\"a}hrung eine viel aktivere Rolle {\"u}bernommen und das war ebenso sehr die Ursache wie die Folge der Instabilit{\"a}t."'\footnote[67]{ \citep*[S.409]{hayek1971}}

Hayek geht es auch weniger um die Geldpolitik an sich, sondern um die Geldarten und W{\"a}hrungen, was schnell zu Verwirrung f{\"u}hren kann: "`Wenn wir von unterschiedlichen Geldarten sprechen, denken wir an unterschiedlich bezeichnete Einheiten, die in ihrem relativen Wert zueinander schwanken k{\"o}nnen. (\ldots)Ich hielt es immer f{\"u}r n{\"u}tzlich, (\ldots) dass es f{\"u}r die Erkl{\"a}rung monet{\"a}rer Ph{\"a}nomene viel hilfreicher w{\"a}re, wenn "`Geld"' als Adjektiv eine Eigenschaft beschriebe, die unterschiedliche Dinge (\ldots)besitzen k{\"o}nnen. "`Umlaufmittel"'ist aus diesem Grund passender, da Dinge in unterschiedlichem ma{"s} in verschiedenen Regionen oder Bev{\"o}lkerungsgruppen "`in Umlauf sein"' k{\"o}nnen."'\footnote[68]{\citep*[S.40f]{Hayek1977}}

Doch "`Wir werden "`Umlaufmittel"' au{"s}erdem, vielleicht etwas im Widerspruch zur urspr{\"u}nglichen Bedeutung des Begriffes, in dem Sinn verwenden, dass nicht nur Papier und andere Sorten eines von "`Hand-zu-Hand-gehenden Geldes"' eingeschlossen sind, sondern auch Scheckkonten und andere Tauschmittel, die f{\"u}r die meisten Zwecke genutzt werden k{\"o}nnen, f{\"u}r die auch Schecks in Frage kommen."'\footnote[69]{\citep*[S.43]{Hayek1977}}

F{\"u}r die W{\"a}hrung sieht Hayek ein Marktmodell vor, wobei "`der Verkauf (am Schalter oder durch Versteigerung) (\ldots) anf{\"a}nglich die wichtigste Emissionsform der neuen W{\"a}hrung [w{\"a}re]. Nachdem sich jedoch ein regul{\"a}rer Markt herausgebildet h{\"a}tte, w{\"u}rde sie normalerweise im Wege des {\"u}blichen Bankgesch{\"a}fts, d. h. durch kurzfristige Kreditvergabe in Umlauf gebracht."'\footnote[70]{\citep*[S.31]{Hayek1977}}

Die Wertstabilit{\"a}t dieser W{\"a}hrung kommt f{\"u}r Hayek durch den Wettbewerb: "`Wettbewerb w{\"u}rde sicherlich die emittierenden Institutionen weit wirksamer dazu zwingen, den Wert ihres Geldes (in Bezug auf ein festgesetztes G{\"u}terb{\"u}ndel) konstant zu halten, als es irgendeine Verpflichtung zur Einl{\"o}sung des Geldes in diese G{\"u}ter (oder in Gold) k{\"o}nnte."'\footnote[71]{\citep*[S.32]{Hayek1977}}

Jedoch "`es ist der "`Grad ihrer Akzeptierbarkeit (oder Liquidit{\"a}t, d. h. in der Eigenschaft), die sie zu Geld macht."'\footnote[72]{\citep*[S.40]{Hayek1977}}

Zur genaueren Beschreibung seines Systems, soll Hayek wieder selbst zu Wort kommen: "`Der emittierenden Bank werden zwei Methoden zur {\"a}nderung des Volumens ihrer zirkulierenden Umlaufmittel zur Verf{\"u}gung stehen: Sie kann ihr Umlaufmittel gegen andere (oder gegen Wertpapiere und m{\"o}glicherweise einige Waren) verkaufen oder kaufen; und sie kann ihre Kreditgew{\"a}hrungst{\"a}tigkeit einschr{\"a}nken oder ausdehnen. Um die ausstehende Menge ihres Umlaufmittels unter Kontrolle zu halten, wird sie sich im ganzen auf die Einr{\"a}umung relativ kurzfristiger Kredite beschr{\"a}nken, so dass bei Reduktion oder zeitweisem Einstellen neuer Kreditvergabe die laufenden R{\"u}ckzahlungen ausstehender Forderungen eine rasche Verminderung ihres gesamten Geldumlaufes mit sich bringen w{\"u}rden."'\footnote[73]{\citep*[S.45]{Hayek1977}}

Wenn diese Forderungen nach einer entstaatlichten W{\"a}hrung nicht funktionieren, so bleiben von Hayek wenigstens einige Anforderungen an die staatliche Geldpolitik, so meint Hayek
"`Es gibt starke und wahrscheinlich immer noch g{\"u}ltige Gr{\"u}nde, die es w{\"u}nschenswert machen, dass diese Institutionen [Zentralbanken] von der Regierung und ihrer Finanzpolitik so weit wie m{\"o}glich unabh{\"a}ngig sind."'\footnote[74]{\citep*[S.412]{hayek1971}}

Dar{\"u}ber hinaus erhebt er die Anforderung "`(\ldots)dass die W{\"a}hrungspolitik m{\"o}glichst voraussagbar sein [soll]"'\footnote[75]{\citep*[S.420]{hayek1971}}, dass f{\"u}r Preisstabilit{\"a}t gesorgt werden m{\"u}sse und "`(\ldots)die Ma{"s}nahmen zur Einwirkung auf Preise und Besch{\"a}ftigung (\ldots) m{\"u}ssen (\ldots)darauf gerichtet sein, Ver{\"a}nderungen [an diesen] zuvorzukommen, bevor sie eintreten, als sie zu korrigieren, nachdem sie eingetreten sind."'\footnote[76]{\citep*[S.422]{hayek1971}}

"`In der Praxis kommt es wahrscheinlich nicht so sehr darauf an, wie dieses Preisniveau definiert ist, au{"s}er, dass es sich nicht ausschlie{"s}lich auf Endprodukte beziehen soll(\ldots)."'\footnote[77]{\citep*[S.423]{hayek1971}}

\subsubsection{Auseinandersetzung mit Keynes}

Gerade eine Kritik an Keynes "`General Theorie"' zu schreiben, wollte Hayek nicht mehr machen, denn "`obwohl er [Keynes] das Werk eine "`allgemeine"' Theorie genannt hatte, war sie f{\"u}r mich zu offensichtlich wieder nur eine zeitbedingte Abhandlung, zugeschnitten auf die augenblicklichen politischen Notwendigkeiten, wie er sie sah."'\footnote[78]{\citep*[S.91]{Hayek1969}}

"`Es ist leicht zu sehen, wie die Anschauung, nach der eine zus{\"a}tzliche Geldsch{\"o}pfung zur Erzeugung einer entsprechenden G{\"u}termenge f{\"u}hren wird, zu einem Wiederaufleben der eher naiven inflationistischen Trugschl{\"u}sse f{\"u}hren musste(\ldots). Ich habe wenig Zweifel, dass wir die Nachkriegsinflation zum Gro{"s}teil dem starken Einfluss eines solchen {\"u}bervereinfachten Keynesianismus verdanken."'\footnote[79]{\citep*[S.93]{Hayek1969}}

Weiter kommentiert Hayek die "`General Theorie"':
"`Ich wage vorauszusagen, dass wenn diese Frage der Methode einmal entschieden ist, die "`Keynessche Revolution"' als eine Episode erscheinen wird, in der irrt{\"u}mliche Auffassungen {\"u}ber die geeignete wissenschaftliche Methode zu einem zeitweiligen Vergessen vieler wichtiger Einsichten f{\"u}hrten, die wir schon gewonnen hatten und die wir dann m{\"u}hevoll wiedergewinnen m{\"u}ssen."'\footnote[80]{\citep*[S.96]{Hayek1969}}

\subsection{Einblick in die Grundgedanken der Monetaristen (Milton Friedman)} %% Autor: Marius Hanniske %%

"`Die Monetaristen"', an der Spitze Milton Friedman\footnote[16]{geboren: 31. Juli 1912 in Brooklyn, New York City; gestorben: 16. November 2006 in San Francisco, im Weiteren Friedman} und die Chicagoer Schule, gehen von der Stabilit{\"a}t des privaten Sektors aus und sehen im
funktionsf{\"a}higen Marktmechanismus die Garantie, dass die Pl{\"a}ne der Wirtschaftssubjekte optimal koordiniert werden, mit der Folge eines permanenten Trends zur Vollbesch{\"a}ftigung der Produktionsfaktoren."'\footnote[81]{\citep*[vgl.][S.210]{peters2000}}

Sie hatten die Auffassung, das starke Impulse von der Geldpolitik ausgehen und
dass sie diese stabilisierende Wirkung aus{\"u}bt.\footnote[82]{\citep*[S.181]{bombach1981theorie}}
Friedman folgert aber, dass die stabilisierende Wirkung der Geldpolitik nur bei konsequenter Verfolgung der Ziele gehalten werden kann. Das ruft dann einen Effekt des Vertrauens
hervor und zwar schreibt Friedman: "`Unser Wirtschaftssystem wird dann am
besten funktionieren, wenn Produzenten und Konsumenten, Arbeitgeber und
Arbeitnehmer(\ldots) darauf vertrauen k{\"o}nnen, dass das durchschnittliche
Preisniveau sich zuk{\"u}nftig nach bekannten Regeln gestaltet - und zwar, dass
es vorzugsweise sehr stabil sein wird."'\footnote[83]{\citep*[vgl.][S.150]{friedman1970die}}

St{\"o}rungen des Wirtschaftsablaufs bzw. extreme Besch{\"a}ftigungsschwankungen werden nach Meinungen der Monetaristen durch Staatseingriffe verursacht. Wenn z. B. die Staatsausgaben erh{\"o}ht werden tritt der Staat als zus{\"a}tzlicher Nachfrager auf dem Geld- und Kreditmarkt auf. Doch wird dadurch keine Mehrnachfrage erzeugt, sondern lediglich die private Nachfrage aus diesem Sektor ersetzt. Nach monetaristischer Lehre bleibt das Preisniveau stabil, wenn die Geldmenge dem Produktionspotential entspricht. Demzufolge ist die Ursache f{\"u}r eine Inflation eine zu hohe Geldmengenausweitung im Verh{\"a}ltnis zum Wachstum der G{\"u}terproduktion.\footnote[84]{\citep*[vgl.][S.213]{peters2000}}
Friedman schl{\"a}gt also vor: "`Geldmengenzuwachs mit j{\"a}hrlich konstanten Raten, wobei die Wachstumsrate so zu bemessen ist, dass ann{\"a}hernd stabile Endproduktpreise resultieren."'\footnote[85]{\citep*[vgl.][S.132]{friedman1970die}} Somit kann mit dieser
Stabilit{\"a}t durch eine "`gem{\"a}{"s}igte Geldmengenausweitung"' eine Inflation und auch
Deflation verhindert werden.\footnote[86]{\citep*[vgl.][S.155]{friedman1970die}}

\subsubsection{Friedman {\"u}ber Keynes} %% Autor: Gregor May %%
Friedmans Sichtweise auf Keynes ist verst{\"a}ndlicherweise durch einen zeitlichen Abstand und den sich damit aufgezeigten Problemen der Keynes'schen Fiskalpolitik gepr{\"a}gt. "`Das Komplement zur Kenes'schen Vernachl{\"a}ssigung des Geldes war die Betonung der
Fiskalpolitik\ldots Besonders in den USA haben sich die Staatsausgaben als das wohl instabilste Element in der Wirtschaft der Nachkriegszeit erwiesen\ldots Dies f{\"u}hrte zu einer Wiederbetonung des flexiblen Instruments der Geldpolitik."'\footnote[87]{\citep*[S.105]{friedman1970die}}

Eine Ursache der fiskalpolitischen Probleme sieht Friedman dabei in der Betrachtung kurzfristiger Entwicklungen zur Hinf{\"u}hrung in ein Gleichgewicht indem die Marktanpassung durch staatliche Eingriffe ersetzt werden.\footnote[88]{\citep*[S.127]{friedman1970die}}

Doch empfindet Friedman dabei auch Bewunderung f{\"u}r Keynes. Obwohl er (Keynes) in Anbetracht der Krise von 1929 den besten Weg,
um aus ihr heraus zu kommen, in Fiskalpolitischen Empfehlungen sah, habe er doch ein gutes
Verst{\"a}ndnis von Geldtheorie."`Bei Keynes finde ich die Geldtheorie scharfsinnig und
modern, jedoch seine politischen Empfehlungen inakzeptabel."' Tats{\"a}chlich verteidigt
Friedman sogar Keynes' "`Fehlentscheidungen"' als Fehlinterpretation der Periode von 1929
bis 1933\footnote[89]{\citep*[S.121]{friedman1970die}}:"`Fr{\"u}her war Keynes ein eifriger Verfechter der Ansicht gewesen, dass man sich
prim{\"a}r auf die orthodoxe Geldpolitik verlassen m{\"u}sse, wolle man die Stabilit{\"a}t der
Wirtschaft f{\"o}rdern. Er gab diese Meinung auf, als er entdeckte, dass die Liquidit{\"a}ts-
Pr{\"a}ferenz Versuche, die langfristigen Zinss{\"a}tze zu {\"a}ndern durchkreuzen konnte\ldots wandte er
sich stattdessen der Fiskalpolitik zu.\footnote[90]{\citep*[S.126]{friedman1970die}}"'


\clearpage
\section{Case Study} %% Autor: Gregor May %%
\label{sec4:CaseStudy}
2014, Jahr 7 nach der wohl gr{\"o}{"s}ten Finanzkrise seit dem 2. Weltkrieg. Die Regierungen der gr{\"o}{"s}ten Weltwirtschaftsnationen planen oder f{\"u}hren gro{"s}e Ausgaben zur Bankenrettung oder wieder in Schwung Bringung der Wirtschaften durch, sehr zum Wohlwollen vieler {\"O}konomen. Und doch sei angemerkt, dass es eine nicht geringe Anzahl an Kommentaren gab und gibt, die diesen Ma{"s}nahmen kritisch bis ablehnend gegen{\"u}ber stehen. Sie tun es aus verschiedenen Gr{\"u}nden, manche aus praktischen, manche aus analytischen und manche aus ideologischen Gr{\"u}nden.

Der Fokus der Debatte geht zur{\"u}ck auf die von Hayek mit begr{\"u}ndete, sogenannte "`{\"o}sterreichische Schule,"' welche nicht selten das Argument mangelnder Effektivit{\"a}t gegen das Eingreifen des Staates in die Wirtschaft geschwungen haben. Von ihrem Standpunkt aus, sollten Individuen ihre eigenen Entscheidungen treffen mit minimaler Beeintr{\"a}chtigung durch den Staat.

Dieser Standpunkt ist derzeit nicht repr{\"a}sentativ f{\"u}r die vorherrschende Hauptstr{\"o}mungen neomonetaristischer und neokeynesianischer Wirtschafts- und Geldpolitik.

Haupts{\"a}chlich handelt es sich dabei um eine These, die mit wenig empirischer Belegbarkeit belastbar ist, allerdings kann diese These helfen die Aufmerksamkeit darauf zu lenken, wo das Eingreifen der Regierungen katastrophal gescheitert ist; jedoch auch ihr Gegenteil: die Beispiele, wo diese Interventionen funktionierten.

\begin{quote}
Ist das alte, liberale Prinzip "`Lasst das Kapital in den H{\"a}nden der Einzelnen Fr{\"u}chte tragen"' immer noch der richtige Leitsatz,
oder ist wirklich der Staat kompetenter zu entscheiden, wo und in welcher Form das verf{\"u}gbare Kapital am zweckm{\"a}{"s}igsten zu verwenden ist? \citep*[S.23f]{Hayek1969}
\end{quote}

Schaut man sich die Geschichte der Entwicklung der (Finanz-)M{\"a}rkte seit dem 16. Jahrhundert an, so kann man wieder und wieder ein, ob nun k{\"u}rzlich oder in Vergangenheit eines beobachten: Unregulierte M{\"a}rkte boomen und kollabieren schlie{"s}lich. Die aller ersten M{\"a}rkte reagierten damals mit erzwungener Standarisierung von Gewichten und Ma{"s}einheiten, eine Regulierung, der sich auch Hayek nicht verschloss.

Ironischerweise pl{\"a}dieren gerade die Vertreter einer Nicht-Regulierungs-Sichtweise im Namen des Allgemeinwohls f{\"u}r die Maximierung der individuellen Freiheit. Die Kosten des individuellen Scheiterns werden jedoch nicht berechnet.

Zwei Hauptargumente lassen sich gegen ein nicht-Eingreifen in den Markt zu Felde f{\"u}hren; zum einen existiert bereits ein Ma{"s} an Eingriffen, weit {\"u}ber das von den {\"O}sterreichern vorgeschlagene Mittel hinausgehend, in Form von lang zeitlich etablierten industriellen Normen. Gleichzeitig sehen wir {\"u}ber die letzten 3 Jahrzehnte einen beispiellosen Anstieg gesellschaftlichen und individuellen Wohlstands. Und dar{\"u}ber hinaus, sei erg{\"a}nzend festgestellt, dass die Vorteile des gesamtgesellschaftlichen Wohlstandanstiegs deutlich mehr geteilt werden, als jemals zuvor. Vereinfacht gesagt: Das derzeitige Modell angemessener Eingriffe funktioniert.

Zum anderen scheinen Verfechter einer Nicht-Regulierungs-Sichtweise davon {\"u}berzeugt zu sein, dass individuelle Entscheidungen in der Regel von Ignoranz, Falschinformation oder betr{\"u}gerischer Absicht beeinflusst sind. //Dies nimmt Hayek als Begr{\"u}ndung, dass der Staat nicht handeln d{\"u}rfe //
Ja, Individuen sind in der Lage, Entscheidungen zu treffen, die ihnen selbst, oder anderen Schaden zuf{\"u}gen k{\"o}nnen. Augenscheinlich ist jedoch, dass wir vermeintlich von Fehlern lernen k{\"o}nnen und wir f{\"u}r das n{\"a}chste mal bessere Entscheidungen treffen k{\"o}nnen. Und ja, manche Entscheidungen sind katastrophal, wie jene, die die Umwelt betreffen. Und: es gibt Fehler, von deren Wiedergutmachung unm{\"o}glich ist.

Selbstverst{\"a}ndlich k{\"o}nnen Regulierungen durch Regierungen taktisch unklug, ja, sogar t{\"o}richt sein. Auch sind desastr{\"o}se Resultate m{\"o}glich. Eingriffe k{\"o}nnen korrumpiert sein, oder schlimmer: strategischen Moral Hazard belohnen. Oder gelegentlich schlicht und einfach nicht funktionieren. Doch k{\"o}nnen diese Einw{\"a}nde nicht als generelles Argument gegen Regulierungen gelten. Jedoch ist es ein Argument daf{\"u}r, dass jede Regulierung mit bedacht ausgew{\"a}hlt und sehr sorgf{\"a}ltig geplant werden muss, sowie, dass aus Erfahrungen gelernt werden muss. Die ist, wie auch der Realit{\"a}t mit die man meistern muss, ein vertracktes unterfangen. Nicht-Regulierungs-Bef{\"u}rworter scheinen eine simplere Antwort, das "`Nein"' zu w{\"a}hlen, eine magische Beschw{\"o}rungsformel um die Realit{\"a}t weniger real werden zu Laben

Wieder und wieder, Streitpunkt um Streitpunkt, so n{\"a}hert man sich Antwort um Antwort den Fragestellungen unserer Hochkomplexen Welt.

Schauen wir uns die j{\"u}ngere Vergangenheit an, so ist r{\"u}ckblickend die mehrheitliche {\"o}ffentliche Meinung dahingehend best{\"a}tigt worden, dass Eingriffe im gro{"s}en Ma{"s}stab in das Bankensystem sinnvoll waren und sind, nachdem dieses stark beeintr{\"a}chtigt war. Es w{\"a}re katastrophal gewesen, einfach darauf zu warten, dass der Bankensystem sich wieder rekonstituirt und h{\"a}tte noch gr{\"o}{"s}eren gesellschaftlichen Schaden angerichtet, als durch die Finanzkrise schon so angerichtet wurde. Dar{\"u}ber hinaus: Wenn Konsumenten und Unternehmen ihre Ausgaben senken, m{\"u}ssen die Regierungen sich gegen den Trend stemmen und seine Ausgaben erh{\"o}hen. Auf die grundlegende Richtigkeit dieser Logik wies schon Keynes hin.

Der wissenschaftliche Einspruch gegen die Erh{\"o}hung in Deutschland ist vor allem, dass diese Erh{\"o}hung der Ausgaben Unternehmensausgaben verdr{\"a}ngen. Doch es kann nichts verdr{\"a}ngt werden, was bereits weggefallen ist. Selbstverst{\"a}ndlich wird diese Verdr{\"a}ngung von Unternehmensausgaben durch den Staat eine ernsthafte Bedrohung, sobald die Wirtschaft sich erneuert. Und ein inflation{\"a}rer Druck steht als zuk{\"u}nftige Gefahr am Himmel. Doch hei{"s}t dies blo{"s}, dass diese stimulierenden Eingriffe schnell zur{\"u}ck gefahren werden m{\"u}ssen, wenn die Umst{\"a}nde, die zu ihrer Einsetzung f{\"u}hrten, sich ver{\"a}ndern.

Auf Grund der Eile und Gr{\"o}{"s}e der in die Wege geleiteten Ausgaben (begleitet von anderen finanziellen Hilfen f{\"u}r Gesch{\"a}fts- und Kreditbanken), war von vornherein klar: Geld wird verschwendet, manches auch ohne Einsatzzweck verbrannt. Doch das war scheinbar der Preis, den man f{\"u}r die "`Kernschmelze des Finanzsektors"' bereit war zu zahlen.

Die Analogie mag vielleicht {\"u}berstrapaziert sein, doch sie trifft immer noch: Wenn das Haus in Flammen steht, sch{\"u}tte Wasser {\"u}berall hin, um die Flammen auszul{\"o}schen. Danach beginnt der schwierige Prozess: Wir m{\"u}ssen den Keller auspumpen und Bausch{\"a}den der Konstruktion reparieren.

\clearpage

\section{Abk{\"u}rzungsverzeichnis}
  \label{sec5:Abkuerzungsverzeichnis}

\begin{acronym}[ESZB]

   \acro{EG}{Europ{\"a}ische Gemeinschaft}
  \acro{EGV}{Europ{\"a}ische Gemeinschaftsvertr{\"a}ge}
    \acro{ESZB}{Europ{\"a}isches System der Zentralbanken}
    \acro{EU}{Europ{\"a}ische Union}
  \acro{EZB}{Europ{\"a}ische Zentralbank}
   \acrodefplural{EZB}{Europ{\"a}ische Zentralbanken}
  \acro{MFIs}{Monet{\"a}re Finanzinstitute}

\end{acronym}

%%%% LITERATURE %%%%%
\vspace{10pt}
  \newpage
    \singlespacing

% Literaturliste endgueltig anzeigen
           %\bibliographystyle{diss_fk}

\bibliographystyle{authordate1}
\section{Literaturverzeichnis}
  \label{sec6:Literaturverzeichnis}
  \bibliography{literatur_EWA}	% Sie benoetigen eine *.bib-Datei
  \newpage

%%%% APPENDIX %%%%%
  %\section*{Anhang}
  %\label{sec:anhang}

\end{document}
