\documentclass[
        onecolumn,
        a4paper,
        abstracton,
        parskip=half
        %,draft
        ,final
        ]{scrartcl}

        \usepackage[pdftex
        ,draft
        ]{graphicx}

        \usepackage{booktabs}

        \usepackage[cmex10]{amsmath}

        \interdisplaylinepenalty=2500
        \usepackage{url}

        \usepackage[breaklinks]{hyperref}
        \hyphenation{nothing} % correct bad hyphenation here

        \usepackage{eurosym}

        \usepackage{listings}
        \lstset{basicstyle=\small\ttfamily,breaklines=true}
        \emergencystretch 1000pt

        \usepackage{subcaption}

        \usepackage{mathtools}

        % deutsche Silbentrennung
        \usepackage[ngerman]{babel}

        \usepackage[printonlyused, withpage]{acronym}

        \usepackage[a4paper]{geometry}

        % wegen deutschen Umlauten
        \usepackage[ansinew]{inputenc}

        % fuer Zitate
        \usepackage[round]{natbib}

        \usepackage{setspace}
        \usepackage{tabulary}
        \usepackage{units}
        \usepackage{cite}
        %%% Deutsche Verzeichnis-ueberschriften

        \renewcommand{\contentsname}{Inhalt}
        \renewcommand{\listtablename}{Tabellenverzeichnis}
        \renewcommand{\listfigurename}{Abbildungsverzeichnis}

        %%% Kommentarfunktion %%%
        \usepackage[textwidth=2.2cm
        ,obeyFinal
        ]{todonotes}

\begin{document}

%%%%%% CREATING THE TITLE %%%%%
\titlehead{
          \centering
          \includegraphics[width = 0.3\textwidth ]{logo_wip.jpg} \\
          \small Technische Universit{\"a}t Berlin \\
          Fakult{\"a}t VII (Wirtschaft \& Management) \\
          Fachgebiet Wirtschafts- und Infrastrukturpolitik (WIP)
          }
          \title{Geldpolitische Str{\"o}mungen und Instrumente}
          %\subtitle{Untertitel}
          \author{
          Gregor May
          \footnotesize \textit{(Matr: 357150)}
          \and
          Marius Hanniske
          \footnotesize \textit{(Matr: 311263)}
          }

          \date{\today}

          \maketitle
   \newpage
%%%% ABSTRACT %%%%%

\selectlanguage{ngerman}

\begin{abstract}
ZUSAMMENFASSUNG
\ldots

\end{abstract}
   \newpage

  \begin{flushleft}
  %\nointend \textbf{JEL Codes:} C61, H54, L94 \\
  %\nointend \textbf{Keywords:} Money, Money Politics, Econimics
  \end{flushleft}

    \tableofcontents
    \listoffigures
    \listoftables

    \newpage
    \onehalfspacing

  %%%%% TEXTK{\"O}RPER %%%%%

\section{Einleitung}
    \label{sec1:einleitung}

    Es war Einmal\ldots
    Das Leben des 1671 in Schottland geborenen John Law liest sich wie ein M{\"a}rchen \citep[vgl.][Kap. 2]{strathern2006schumpeters}. Ein Draufg{\"a}nger im Kasino und mit den Frauen. In England zum Tode verurteilt, weil er als Sieger aus einem Duell hervorging, floh er nach Europa und bekam in den dortigen Kasinos und der fortschrittlicheren Finanzpolitik anderer L{\"a}nder, wie z.B. die Niederlande \footnote[2]{Belebung der Schifffart dank Windm{\"u}hlen, erster B{\"o}rsencrash auf dem TulpenmarktText} ein f{\"u}r diese Zeit, neues Verst{\"a}ndnis von Geld: "Ungenutztes Geld war nichts - nichts als das Potential f{\"u}r Aktion." \citep[vgl.][Kap. 2]{strathern2006schumpeters}Law erkannte rasch, dass mehr Bargeld zu mehr Handelsaktivit{\"a}t f{\"u}hrt und unterbreitete seine Ideen u.a. in Schottland und Turin. Die Staaten k{\"o}nnen Noten auf sich selbst ausstellen,d.h. Kredite gew{\"a}hren im Tausch mit der F{\"a}higkeit, in Zukunft Geld aufzubringen, mit Hilfe von Steuern. Dieses Geld wird in England noch heute fiat money [4] genannt und ist heute in Europa und auch den USA\footnote[5]{"In God we trust" steht u.a. auf der 1-Dollar-Note und Gottesvertrauen braucht man auch} gel{\"a}ufig.

    Durch Freunde bekam Law Kontakt zum Regenten von Frankreich\footnote[6]{Philippe , Herzog von OrlÈans, nachdem der Sonnenk{\"o}nig Ludwig 14. 1715 starb}, der ihm vollstes Vertrauen entgegen brachte. Die franz{\"o}sische Staatskasse, die zu der Zeit 3 milliarden Livres \footnote[7]{Livre frz.: Pfund war vom 9. bis zum 18. Jh. die franz{\"o}sische W{\"a}hrung, 1795 abgel{\"o}st vom Franc} an Schulden angeh{\"a}uft hat, ben{\"o}tigte neue Ideen und Law war der Mann der sie hatte. Er gr{\"u}ndete 1716 die Banque National, die erste Bank Frankreichs, ausgestattet mit 6 Millionen Livres Aktienkapital. Bareinzahlungen waren in Form von M{\"u}nzen auf Konten m{\"o}glich, auch {\"U}berweisungen durch Schecks auf andere Konten. Was aber neu und besonders war stellte "raison d'etre" \footnote[8]{gedeckt durch die 6 Millionen Aktienkapital, Versprechen(!) auf Auszahlung, allerdings Barreserven von 350 tausend Livres} dar, d.h. die Ausgabe von Papiergeld. Das war die grandiose Idee und sie funktionierte. Die Leute hatten mehr Geld zur Verf{\"u}gung und gaben es auch aus, die G{\"u}ternachfrage stieg an und damit auch die G{\"u}terproduktion. Und einen weiteren Plan setzte Law um, die Gr{\"u}ndung der Mississippi-Gesellschaft. Philippe {\"u}bereignete ihm dazu Louisana und mit der Ausgabe von Aktien sollte damit eine Expedition \footnote[9]{ebenda} finanziert werden. Law und seine Familie bewegten sich in den h{\"o}chsten Kreisen, ja befanden sich f{\"o}rmlich an der Spitze, der High Societie. Der p{\"a}bstliche Nimbus, der bewegt war zur Geburtstagsfeier von Laws Tochter eingeladen zu sein, gab der Tochter einen Kuss und ihr {\"a}lterer Bruder ging mit Ludwig 15. jagen. Wenn das wirklich ein M{\"a}rchen gewesen w{\"a}re w{\"u}rde es jetzt mit den Worten: "\ldots und wenn sie nicht gestorben sind, Leben sie noch Heute", enden. Doch es war wie schon erw{\"a}hnt kein M{\"a}rchen und das SystemË, wie Law es nannte, brach in sich zusammen. Die Gewinne wurden nicht wie versprochen f{\"u}r eine Expedition ausgegeben, sondern zur Begleichung der imensen Staatsschuld. Bei einer Abwertung der Noten verloren die Menschen endg{\"u}ltig das Vertrauen in diese.
    Wie schon erw{\"a}hnt funktioniert die heutige Geldpolitik wie bei dem von Law durchgef{\"u}hrten Feldexperiment, doch Sicherheitsmechanismen verhindern meist eine derartige Eskalation.
    Wir befassen uns in dieser Hausarbeit mit der Ausarbeitung dieser Sicherheitsmechanismen durch verschiedene Denkans{\"a}tze, Funktionen die die Instabilit{\"a}t des Systems vertr{\"a}glicher f{\"u}r seine Benutzer gemacht haben. Dabei gehen wir auf die Denkanst{\"o}{\ss}e der Str{\"o}mungen des Keynesianismus, der Monetaristen und der Gruppe der neuen politischen  {\"O}konomie, der {\"O}sterreichischen Schule ein.


\subsection{Benennung der Fragestellung}

    -> Einblick in die Grundgedanken der {\"O}konomen des 19.und 20.Jahrhunderts

    \ldots Doch am Anfang steht die Forschung. Dabei sind zwei Motive herausgearbeitet \footnote[9]{der Keynesianismus 1},
    die die Triebfedern der wissenschaftlichen Forschung darstellen:
    1.) die Neugierde nach Wissen {\"u}ber die Welt, sie ist eine regelm{\"a}{\ss}ige Motivation zur Forschung, wie auch John Law versucht hat seine Ideen umzusetzen um Herauszufinden
     ob sie funktionieren. Eine 2.) unabh{\"a}ngig davon treibende Kraft ist die Unvollst{\"a}ndige Information, die daf{\"u}r sorgt, dass die Menschen nur eine vage oder gar keine Kenntnis von der Zukunft besitzen. Die Angst der Menschen vor dem Unbekannten veranlasst sie dazu diese Unkenntnis zu beseitigen. Dieser Drang zur Ver{\"a}nderung der Situation schwindet jedoch, wenn die Zukunft vielversprechend aussieht.

    Nicht zu vernachl{\"a}ssigen sind die M{\"o}glichkeiten einer Ver{\"a}nderung, als Reaktion auf
    die entt{\"a}uschenden Ergebnisse wissenschaftlicher Theorien. Was tun, wenn sich eine
    Diskrepanz zwischen der Theorie und Realit{\"a}t aufzeigt? Die erste M{\"o}glichekeit:
    Evolution -> Anpassung der Theorie an die Realit{\"a}t. Schwierig, wenn die Theorie fest
    gefahren ist und auf Probleme mit den immer gleichen Argumentationen antworten will.
    Dann bleibt nur die zweite M{\"o}glichkeit der Revolution -> bei der eine v{\"o}llig neue
    Konstruktion der Analytik angefertigt wird, die die Realit{\"a}t besser wiedergibt als ihre
    Vorg{\"a}ngertheorie\ldots
    (Nicht zu vergessen das Falsifizierbarkeitskriterium: Wiederlegung einer Theorie hat ein
    st{\"a}rkeres Gewicht als ihre Best{\"a}tigun)
    Verdeutlichen wir uns das mal an den {\"o}konomischen Theorien des 19. und 20. Jahrhunderts,
    den Klassiker und Neoklassikern.
    Auf Kr{\"a}fte des freien Marktes berufen sich die Klassiker\footnote[10]{in dem Fall sind Klassiker und Neoklassiker zusammengefasst}, den Anfang machte dabei
     Adam Smith mit seinem 1776 ver{\"o}ffentlichten Werk "`Wohlstand der Nationen"'.
    "`Die freie Marktwirtschaft wird erkl{\"a}rt von einer Unsichtbaren Hand, die der
    Preismechanismus darstellt, {\"u}ber Angebot und Nachfrage, die in ein Gleichgewicht dr{\"a}ngen,\ldots"'

    Sie unterstellen den Individuen ein rationales Verhalten (homo oeconomicus), aus einer
     Vielzahl von Angeboten das Beste herauszufiltern. Preise dienen dazu die Angebote
    unterscheiden und sortieren zu k{\"o}nnen\ldots mit diesen Argumentationen  (der Realpreise\ldots[Die
    Preise von G{\"u}tern kommen nie unabh{\"a}ngig von Preisen anderer G{\"u}ter zustande])befassten sich
    vornehmlich die Neoklassiker. Die Preise wiederrum k{\"o}nnen bequemer zugeteilt,
    identifiziert und bewertet werden wenn sie eine (be)rechenbare Einheit erhalten.
    Diese Einheit, ob es nun US-Ameriaknische Dollar, Japanische Yen oder Europ{\"a}ische Euro
    sind, stellt das Geld ganz allgemein dar.( um nicht auf Zuf{\"a}lle hoffen zu m{\"u}ssen, dass
    ausgerechnet ein B{\"a}cker 2000 Br{\"o}tchen ben{\"o}tigt um die gegen einen neuen Fernseher eintauschen zu k{\"o}nnen [anm. ich habe irgentwo ein sch{\"o}nes Zitat zu dem Problem, ich find es blo{\ss} gerade nicht])

    \ldots

    - Fragestellung ob der Staat eine Rolle einnimmt? -> definitiv nicht Aktiv,
    [aber hei{\ss}t passiv nicht, dass er sich schon einmischt, blo{\ss} so, dass es keiner Mitbekommt?]
    - Die Mittel zur Steuerung sind ganz klar sich nicht einzumischen, laissez faire,
    [die Unsichtbare Hand wird es schon richten]


    - Sind diese Mittel kontr{\"a}r zur derzeitigen Geldpolitik?
    heutige Situation: staatliches Eingreifen notwendig, weil sonst Unterversorgung bestimmter
    G{\"u}ter durch privat Angebot. Au{\ss}erdem: Staat hat Sorge zu tragen, f{\"u}r jedes Mitglied der
    Gesellschaft keine Nachteile durch marktwirtschaftliche Vorg{\"a}nge entstehen zu lassen -> daraus
    w{\"u}rden sonst Spannungen zwischen den sozialen Schichten entstehen, daraus folgt ein
    Auseinanderbrechen des sozial Systems. "`\ldots seit Entwicklung der Keynesianischen VWL besteht,
    \ldots, {\"U}bereinstimmung hinsichtlich der Auffassung, dass zum Zwecke einer Nivellierung der
    konjunkturellen Schwankungen eine aktive staatliche Wirtschaftspolitik nicht nur f{\"o}rderlich,
     sondern auch erforderliche ist."'

    \ldots

    (Dabei vernachl{\"a}ssigen die Klassiker bewusst eine Wertaufbewahrungsfuktion des Geldes, weil es
    der {\"o}konomischen Rationalit{\"a}t wiederspricht)[ -> da kommt dann schon Keynes ins Spiel, deswegen
    den Absatz ganz zum Schluss]


\subsection{Beschreibung des eigenen Verst{\"a}ndnis von Geldpolitik}

    Nach Basseler hat die Geldpolitik die Hauptaufgabe, eine optimale Geldversorgung der Wirtschaft zu gew{\"a}hrleisten \citep[Vgl.][S. 551]{Basseler2010}.
    Real wird diese Aufgabe von einer gr{\"o}{"ss}tenteils staatlichen, aber unabh{\"a}ngigen Zentralbank {\"u}bernommen. Dabei herrscht weit verbreitet Konsens dar{\"u}ber, dass Geldpolitik staatliche Aufgabe bleibt, auch wenn Ideen einer dezentralen, dem Wettbewerb unterliegenden Geldversorgung durch private Gesch{\"a}ftsbanken und ein System konkurierender Parallelw{\"a}hrungen kursieren. (Hayek)
    "`Zentrale Zielgr{\"o}{"ss}e der Geldpolitik ist die Geldmenge M3. (...) Dabei ist zu beachten, dass das herk{\"o}mmliche Konzept von Banken zum Konzept der Mont{\"a}ren Finanzinstitute (MFIs) erweitert worden ist."`
    \citep[vgl.][S. 507]{Basseler2010}

    \citep[vgl.][S.508]{Basseler2010} "`MFIs sind also im Wesentlichen:
    - Zentralbanken,
    - Kreditinstitute und
    - Geldmarktfonds.
    (...) Innerhalb der Geldmenge M3 spielen Bargeldumlauf und t{\"a}glich f{\"a}llige Einlagen die gr{\"o}{"ss}te Rolle; Einlagen mit vereinbarterter K{\"u}ndigungsfrist (...) sind ebenfalls quantitativ bedeutsam; die {\"u}brigen Komponenten machen insgesamt nur knapp 20 Prozent der Geldmenge M3 aus."`


    \subsubsection{ Akteure des Finanzbereiches}

    \citep[vgl.][S.511f]{Basseler2010} "`Die Akteure im Finanzbereich werden allgemein Finanzintermedi{\"a}re genannt. [Diese] vermitteln Finanzprodukte zwischen den Anbietern und Nachfragern. (...) Dies sind vor allem Banken und Kapitalanlagegesellschaften, die selbst Finanzprodukte kreieren sowie institutionelle Anleger."`

     "`Es muss im Finanzbereich eine staatliche Institition geben, eine staatlich organisierte Zentralbank, die folgende Aufgaben erf{\"u}llt:
     \begin{itemize}
    \item{die Ausgabe der gesetzlichen Zahlungsmittel (Staatliches Emissionsmonopol),}
    \item{die Durchf{\"u}hrung einer Geldpolitik mit dem Ziel einer angemessenen Begrenzung der Geldmenge,}
    \item{die Organisation eines reibungslosen Zahlungs- und Kreditverkehrs als >>Bank der Banken<<,}
    \item{die Wahrung der Geldwertstabilit{\"a}t,}
    \item{die Bereitstellung einer ausreichenden Menge an Geld in Krisenzeiten"'}
    \end{itemize}


    \citep[vgl.][S.512-13]{Basseler2010} "`Gesch{\"a}ftsbanken (oder auch Kreditinstitute) sind die zentralen Akteure im Finanzbereich einer Volkswirtschaft. Die erste zentrale Funktion von Gesch{\"a}ftsbanken (kurz: Banken) ist die Abwicklung des **Zahlungsverkehrs** einer Volkswirtschaft. (...) Die zweite zentrale Funktion von Banken ist die Organisation und Durchf{\"u}hrung des **Kreditverkehrs** einer Volkswirtschaft. "`
    "`(...)die Organisation des Kreditverkehrs ist ein klassisches Gesch{\"a}ft der banken: Sie beschaffen Geld, sie verleihen Geld und sie versuchen, dieses Geld mit Gewinn wieder zur{\"u}ckzubekommen.(...)Die Kreditgew{\"a}hrung war neben der Abwicklung des Zahlungsverkehrs die klassische Aufgabe der Gesch{\"a}ftsbank."`
     "`Daneben gibt es weitere Aufgaben der Banken, vor allem die Verm{\"o}gensverwaltung der Kunden, die Ausgabe und den Handel mit Wertpapieren, die Beratung und Unterst{\"u}tzung bei Unternehmenszusammenschl{\"u}ssen oder die Unterst{\"u}tzung von Unternehmen bei ihrer Kapitalaufnahme, etwa bei B{\"o}rseng{\"a}ngen. Dies wird zusammenfassend **Investmentbanking** bezeichnet. (...) In diesem Segment des Bankengesch{\"a}fts  werden auf Zertifikate entwickelt und verkauft oder Fonds aufgelegt und Fondsanteile verkauft. (...)In Kontinentaleuropa ist (...)das Universalbankensystem etabliert(...). Sparkassen {\"u}bernehmen (...)die Abwicklung des Zahlungsverkehrs und des >>kleinen<< Kreditverkehrs, kleine Privatbanken {\"u}bernehmen eher die Funktionen des Investmentbankings und gro{"ss}e Universalbanken {\"u}bernehmen alle Gesch{\"a}ftssparten."`

    \citep[vgl.][S.515]{Basseler2010} "`Das Eigenkapital [einer Bank] setzt sich konkret zusammen aus dem Grundkapital, den Kapitalr{\"u}cklagen und den Gewinnr{\"u}cklagen (einbehaltene Gewinne) sowie einer stillen Einlage des Finanzmarktstabilisierungsfonds (...). Grunds{\"a}tzlich ist das Eigenkapital der Banken von zentraler Bedeutung. Es ist letztlich das Kapital, das die Bank zum Ausgleich von Verlusten aus ihrem Kredit- und Investmentgesch{\"a}ft einsetzen kann.  Daher sind in der Bankenaufsicht bestimmte Mindestanforderungen an die H{\"o}he des haftenden Eigenkapitals (...)vorgesehen"`

    \citep[vgl.][S.512]{Basseler2010}  "`Mit dem (...) 01.01.1999 ist (...)die Europ{\"a}ische Zentralbank (EZB) die zentrale Institution - also die Zentralbank - f{\"u}r die Festlegung und Ausf{\"u}hrung der Geldpolik. Daneben existiert weiterhin die Deutsche Bundesbank, die als Zentralbank der Bundesrepublik Deutschland Teil des Europ{\"a}ischen Systems der Zentralbanken ist. Sie ist (...) ausf{\"u}hrendes organ der geldpolitischen Entscheidungen der (...)EZB. "`
    \clearpage

\section{Technisches System}
	\label{sec2:technischesSystem}



\subsection{Organisationen der Europ{\"a}ischen Geldpolitk}
%%Autor:Gregor May %%

In der Europ{\"a}ischen Union (EU) wird die Geldpolitik durch die Europ{\"a}ische Zentralbank (EZB) und das Europ{\"a}ische System der Zentralbanken (ESZB) organisiert. Dabei umfasst das ESZB alle 28 nationalen Zentralbanken (NZBen) der Mitgliedsstaaten der EU sowie die Europ{\"a}ische Zentralbank. Zuletzt wurden Estland und Lettland in die Euro-Zone aufgenommen.\footnote[98]{Die Eurozone besteht derzeit aus 18 EU-Staaten und hat daher auch den Beinamen "`Euro-18"' erhalten.} Sonderstatus im ESZB haben dabei die sogenannten "`Outs"', jene Mitgliederstaaten der EU, die den Euro nicht eingef{\"u}hrt haben.
Dies umfasst derzeit: D{\"a}nemark, Gro{"s}britanien, Schweden sowie die meisten neuen EU-Mitgliedsstaaten nach 2001. Sie sind vom Entscheidungsprozess der ESZB ausgeschlossen und vollziehen eine eigenst{\"a}ndige nationale Geldpolitik.
Formal unterscheidet der EG-Vertrag\footnote[25]{EGV, Art. 105-109 d} zwischen EZB und ESZB, faktisch entscheidet jedoch nur eine Institution: die EZB mit ihren Beschlussorganen (EZB-Rat und Direktorium der EZB.) \footnote[99]{\citep[vgl.][S.553]{Basseler2010}}
Mit Organisation von von Geldarten und Geldsch{\"o}pfung durch die EZB und das ESZB sind diese bei einer supranationalen Institution innerhalb des Euro-Systems monopolisiert.

\subsection{Die Europ{\"a}ische Zentralbank}

Der EZB-Rat und das Direktorium der EZB leiten als Beschlussorgane der Europ{\"a}ischen Zentralbank die EZB. Das Direktorium der EZB setzt sich aus dem Pr{\"a}sidenten und dem Vizepr{\"a}sidenten der EZB, sowie weiteren vier Mitgliedern zusammen, die von den Regierungen der Mitgliedsstaaten auf der Ebene der Staats- und Regierungschefs einvernehmlich ernannt werden. Dem EU-Rat steht hierbei ein Empfehlungsrecht zu. Erweitert besteht beim Europ{\"a}ischem Parlament und beim EZB-Rat ein Anh{\"o}rungsrecht.\footnote[46]{\citep[vgl.][S.553]{Basseler2010}}

Der EZB-Rat wiederum besteht aus dem Direktorium und den Pr{\"a}sidenten aller nationalen Zentralbanken, die den Euro gemeinsam eingef{\"u}hrt haben. Innerhalb der Europ{\"a}ischen Zentralbank liegt die exekutive Gewalt beim Direktorium, welches "`f{\"u}r die Durchf{\"u}hrung der Geldpolitik nach den Leitlinien und Beschl{\"u}ssen des EZB-Rates verantwortlich ist.\footnote[47]{\citep[S.553]{Basseler2010}}"'

Das Direktorium der EZB ist gegen{\"u}ber den nationalen Zentralbanken des Eurosystems weisungsbefugt. Somit entsteht ein duales System, bestehend aus dem Exekutivorgan der EZB in Form des Direktoriums und ein ein Beschlussorgan in Form des EZB-Rates.

Beschl{\"u}sse der gemeinschaftlichen europ{\"a}ischen Geldpolitik des Euro-Raumes werden vom EZB-Rat erarbeitet und erl{\"a}sst um die Ausgabe von M{\"u}nzen und Banknoten zu regeln oder um die vom ESZB {\"u}bertragenen Aufgaben zu erf{\"u}llen. \footnote[100]{\citep[vgl.][S.553]{Basseler2010}}
Derzeit umfasst der EZB-Rat 18 L{\"a}nder und das Direktorium, abgestimmt wird mit einfacher Mehrheit der Anwesenden, bei Stimmengleichheit entscheidet die Stimme des Pr{\"a}sidenten.


\subsection{Ziele und Aufgaben von ESZB und EZB}
Die EG-Vertr{\"a}ge legen als vorrangiges Ziel des ESZB und damit der EZB "`die Gew{\"a}hrleistung der Preisstabilit{\"a}t(\ldots). (\ldots)Soweit dies ohne Beeintr{\"a}chtigung des Zieles der Preisstabilit{\"a}t m{\"o}glich ist, unterst{\"u}tz das ESZB die allgemeine Wirtschaftspolitik der Gemeinschaft, um die Verwirklichung der in Artikel 2 festgelegten Ziele der Gemeinschaft beizutragen. Das ESZB handelt im Einklang mit dem Grundsatz einer offenen Marktwirtschaft mit freien Wettbewerb \ldots"'\footnote[48]{\citep[vgl.][S.554]{Basseler2010}}

Hier wird von der EZB eine Priorisierung der Preisstabilit{\"a}t gegen{\"u}ber anderen Zielen wie Vollbesch{\"a}ftigung und Wachstum festgeschrieben - diese weiteren Ziele werden der Preisstabilit{\"a}t untergeordnet. Verglichen mit der Zielvorschrift der Deutschen Bundesbank, entspricht diese Formulierung weitestgehend der damals geltenden Zielvorschrift.\footnote[49]{ebd.}

Ideologische Grundlage f{\"u}r diese Priorisierung ist die von monetaristischen Str{\"o}mungen ausgehende Vorstellung, dass eine Zentralbank zu aller erst die Verantwortung f{\"u}r eine Preisstabilit{\"a}t besitzt, da andere Akteure f{\"u}r die Vollbesch{\"a}ftigung zust{\"a}ndig sind (z.B. Gewerkschaften, Tarifparteien, etc.) und der Wachstum sich aus dem technischen Fortschritt und dem Bev{\"o}lkerungswachstum ergibt.\footnote[50]{ebd.}

Die Aufgaben des ESZB werden im Art. 105, Abs. 2 EGV wie folgt festgelegt:
\begin{itemize}
 \item{die Geldpolitik der Gemeinschaft festzulegen und auszuf{\"u}hren, Divisengesch{\"a}fte im Einklang mit Artikel 111 durchzuf{\"u}hren,}
 \item{die offiziellen W{\"a}hrungsreserven der Mitgliedstaaten zu halten und zu verwalten,}
 \item{das reibungslose Funktionieren des Zahlungssystems zu f{\"o}rdern}\footnote[50]{\citep[vgl.][S.555]{Basseler2010}}
\end{itemize}

\subsubsection{Unabh{\"a}ngigkeit der EZB}

Preisstabilit{\"a}t zu gew{\"a}hrleisten ist als vorrangige Ziel der EZB festgeschrieben worden. Da die EZB eine Gesamteuropa{\"a}ische Geldpolitik organisieren soll, steht eine Unabh�ngigkeit von den nationalen Interessen einzelner Regierungen als unabdingbar im Raum. Der Einfach halber sei festgestellt, dass die EZB auf verschiedene Arten als unabh{\"a}ngig gelten darf.\footnote[51]{\citep[vgl.][S.555-557]{Basseler2010}}

Zum Ersten ist die EZB funktional relativ unabh{\"a}ngig, da sie Weisungen nicht entgegennehmen darf.\footnote[34]{vgl. Artikel 107, EGV (Unabh{\"a}ngigkeit der EZB)}

Eine solche Unabh{\"a}ngigkeit - keinerlei Kontrollen durch Regierungen und Parlamente zu unterliegen - ist relativ einzigartig. Sie wird nur dadurch ein klein wenig beschr{\"a}nkt, dass die Verpflichtung besteht, eine allgemeine Wirtschaftspolitik der Gemeinschaft zu unterst{\"u}tzen, allerdings nur, wenn dadurch das Ziel der Preisstabilit{\"a}t nicht beeintr{\"a}chtigt ist.

Zum Zweiten ist die EZB dar{\"u}her hinaus personell unabh{\"a}gig, einzig durch die Ernennung der Pr{\"a}sidenten der Nationalbanken k{\"o}nnen einzelne Regierungen Einfluss aus{\"u}ben.

Zum Dritten ist die EZB auch finanziell unabh{\"a}ngig - sie verf{\"u}gt {\"u}ber eigene Einnahmen, einen eigenen Haushalt - und sie besitzt die Kontrolle {\"u}ber alle Instrumente der Geldpolitik.

Bevor wir weiter auf die Geldpolitik und die Instrumente der EZB eingehen, sei angemerkt, was jener Artikel 111 EGV, welcher die Devisengesch{\"a}fte der EZB regelt beinhaltet:
So legt dieser Artikel fest, das alle Entscheidungen {\"u}ber die Wechselkurssysteme, ob nun flexible oder feste Wechselkurse, oder ihre H{\"o}he bei Festlegung fester Wechselkurse dem Ministerrat\footnote[52]{\citep[S.555]{Basseler2010}} vorbehalten sind.


\subsection{Instrumente der Europ{\"a}ischen Geldpolitik}

	Im weiteren Verlauf wollen wir kurz die geldpolitischen Instrumente anschauen, die der Europ{\"a}ischen Zentralbank, vertreten durch den EZB-Rat und das Direktorium, zur Verf{\"u}ging steht.

\subsubsection{Offenmarktpolitik der EZB}



Zentrales Instrument f{\"u}r die Geldversorgung einer Wirtschaft ist die Offenmarktpolitik. Man versteht hierunter den An- und Verkauf von Wertpapieren gegen Zentralbankgeld durch die Zentralbank.
Durch Offenmarktk{\"a}ufe bzw. -verk{\"a}ufe kann die Zentralbank die Zentralbankgeldsch{\"o}pfung steuern. F{\"u}r die EZB im ESZB sind nur finanziell solide MFIs, sprich Finanzinstitute, die in das Mindestreservesystem einbezogen sind, als Gesch{\"a}ftspartner der EZB zugelassen. Es wird allgemein zwischen der expansiven (Kauf von Wertpapieren) und kontraktiven Offenmarktpolitik (Verkauf von Wertpapieren) unterschieden.\footnote[54]{\citep[S.556]{Basseler2010}}

Nach einem Kauf von Wertpapieren hat sicher der Bestand an Zentralbankgeld der Gesch{\"a}ftsbanken erh{\"o}ht, diese erh{\"o}hte Geldbasis erm{\"o}glicht nun den Gesch{\"a}ftsbanken Kredite an Nichtbanken weiterzugeben.\footnote[55]{\citep[S.557]{Basseler2010}}

Die Zentralbank kann Gesch{\"a}ftsbanken nicht zur Offenmarktpolitik zwingen, sondern muss entsprechende Anreize bieten. Diese bestehen in niedrigen Zinss{\"a}tzen f{\"u}r die Zentralbankgeld-Kreditgew{\"a}hrung unterhalb des Geldmarktzinses; bei kontraktiver Offenmarktpolitik in Zinss{\"a}tzen {\"u}ber dem Geldmarktzins.
Zeitlich befristete Offenmarktgesch{\"a}fte werden auch Wertpapierpensionsgesch{\"a}fte, oder Repo-Gesch{\"a}fte genannt \footnote[36]{Repo-Gesch{\"a}fte leitet sich vom englischen Repurchase - R{\"u}ckkauf ab}, der Zinssatz wird hier Pensionssatz genannt \footnote[37]{Entsprechend hei{"s}t der Pensionssatz auch Repo-Rate}
Mit Repo-Gesch{\"a}ften l{\"a}sst sich die Zentralbankgeldmenge durch die Zentralbank recht einfach steuern, da, wenn zeitlich begrenzte Gesch{\"a}fte auslaufen und nicht erneuert werden, automatisch kontraktive Effekte auftreten.


Das Bankensystem im Eurogebiet ist auf die Bereitstellung von Zentralbankgeld durch die EZB angewiesen, weshalb die Offenmarktpolitik der EZB am Interbankengeldmarkt\footnote[53]{\citep[vgl.][S.558f]{Basseler2010}} ansetzt. Dementsprechend greift die EZB auf folgende Instrumente zur Steuerung der Offenmarktpolitik zur{\"u}ck:
\begin{enumerate}
 \item{Hauptrefinanzierungsinstrument}
 \item{l{\"a}ngerfristige Refinanzierungsgesch{\"a}fte}
 \item{Feinsteuerungsoption}
 \item{Strukturelle Operationen}
 \end{enumerate}

Das \textbf{Hauptrefinanzierungsinstrument} und das \textbf{l{\"a}ngerfristige Refinanzierungsgesch{\"a}ft} sind regelm{\"a}{"s}ig stattfindende, dem Interbankenmarkt Liquidit{\"a}t zuf{\"u}hrende Repo-Gesch{\"a}fte.
F{\"u}r das Hauptrefinanzierungsinstrument sind die w{\"o}chentlichen Transaktionen auf eine Woche G�ltigkeit begrenzt.Das Hauptrefinanzierungsinstrument steuert die Zinss{\"a}tze (Zentraler Leitzins der EZB) sowie die Liquidit{\"a}t am europ{\"a}ischen Geldmarkt. Die Refinanzierungsgesch{\"a}fte finden mit monatlichem Abstand statt und haben  Laufzeiten �ber mehre Monate bis hin zu einem Jahr Dauer.


Sollte es zu unerwarteten marktm{\"a}{"s}igen Liquidit{\"a}tsschwankungen auf die Zinss{\"a}tze kommen, stehen immernoch die \textbf{Feinsteuerungsoperationen} zur Verf{\"u}gung. Diese werden von Fall zu Fall mit befristeten Transaktionen, oder in der Form von definitiven Verk{\"a}ufen oder K{\"a}ufen von Wertpapieren ausgef{\"u}hrt.

Der EZB stehen als letztes Instrument der Offenmarktpolitik die \textbf{Strukturellen Operationen} zur Verf{\"u}gung. Sie haben das Ziel die grundlegenden Liquidit{\"a}tspositionen des Finanzsektors zu beeinflussen. Dies passiert {\"u}ber die Emission von Schuldverschreibungen, definitive (Ver-)K{\"a}ufe und befristete Transaktionen.\footnote[58]{\citep[S.560]{Basseler2010}}

\subsubsection{St{\"a}ndige Fazilit{\"a}ten}

Will die EZB eine genaue Steuerung des Geldmarktzinssatze, so greift sie auf das Instrument der St{\"a}ndigen Fazilit{\"a}t zur{\"u}ck. Hierunter versteht man die Bereitstellung, bzw. Absch{\"o}pfung von Liquidit{\"a}t jeweils bis zum n{\"a}chsten Gesch{\"a}ftstag in Form von Tageskrediten oder t{\"a}glichen Anlagen {\"u}bersch{\"u}ssiger Liqudit{\"a}t. Im Unterschied zur Offenmarktpolitik erfolgt die Inanspruchnahme der st{\"a}ndigen Fazilit{\"a}ten auf Initiative der Banken und ist grunds{\"a}tzlich unbeschr{\"a}nkt m{\"o}glich. Bisland werden sie jedoch nur in geringem Umfang genutzt, da die Konditionen im Vergleich zu den Konditionen am Interbankenmarkt ung{\"u}nstiger sind.\footnote[59]{\citep[vgl.][S.560ff]{Basseler2010}}

Gesch{\"a}ftsbanken k{\"o}nnen zur Deckung eines vor{\"u}bergehenden Liquidit{\"a}tsbedarfs unbegrenzt die Spitzenrefinanzierungsfazilit{\"a}t in Anspruch nehmen\footnote[36]{Dieser Bedarf wird mit einem im Voraus bekanntgegebenen Kreditzinssatz verzinst, dieser Zinssatz ist die Obergrenze des allgemeinen Tagesgeldsatzes am Geldmarkt}, m{\"u}ssen dazu aber von der EZB als Gesch{\"a}ftspartner zugelassen sein.

Haben die Gesch{\"a}ftsbanken eine {\"u}bersch{\"u}ssige Liqudit{\"a}t, sin k{\"o}nnen sie auf die Einlagefazilit{\"a}t zur{\"u}ckgreifen, und die Liquidit{\"a}t bis zum n{\"a}chsten Gesch{\"a}ftstag bei den nationalen Zentralbanken anlegen.\footnote[37]{Diese Einlage wird mit einem im Voraus bekanntgegebenen Zinssatz verzinst, dieser Zinssatz ist die Untergrenze des allgemeinen Tagesgeldsatzes am Geldmarkt}.

F{\"u}r gew{\"o}hnlich bewegt sich der Leitzins der EZB zwischen dem Zinssatz f{\"u}r die Einlagefazilit{\"a}t und der Spitzenrefinanzierungsfazilit{\"a}t, der Zinssatz von weniger als 1 Prozent f{\"u}r die Hauptrefinanzierung stellt einen bisher unerreichten historischen Tiefpunkt f{\"u}r die EZB dar.
Das die EZB Wert darauf legt, den Geldmarktzinssatz genau zu steuern, zeigt, dass auch Keynsianische Elemente in die Geldpolitik der EZB einflie{"s}en.\footnote[59]{\citep[vgl.][S.562f]{Basseler2010}}

\subsubsection{Mindestreservepolitik} \citep[vgl.][S.562f]{Basseler2010}
In der Europ{\"a}ischen Geldpolitik ist den Gesch{\"a}ftsbanken vorgeschrieben bei der Zentralbank einen bestimmten Prozentsatz ihrer Einlagen - den Mindestreservesatz - als Sichtguthaben vor zu halten. Die Mindestreservepolitik ist daf{\"u}r gedacht, einen stark wirkenden Einfluss auf die Geldsch{\"o}pfung, bzw. das Geldsch{\"o}pfungspotentioal der EZB zu erhalten. Gibt die EZB eine Erh{\"o}hung des Mindestreservesatz vor, nimmt das Geldsch{\"o}pfungspotential zu, und bei Senkung nimmt es ab. Ein gewollter Nebeneffekt ist, dass bei einer {\"a}nderung des Mindestreservesatzes auch die Liqudit{\"a}tsreserven der Gesch{\"a}ftsbanken sich ver{\"a}ndern.
Die Mindestreservepolitik schafft so einen stabilen zus{\"a}tzlichen Zentralbankgeldbedarf und stellt so die direkte Verbindung zwischen Mindestreserve und Geldsch{\"o}pfung her.
Mindestreserven muss jedes Kreditinstitut im Eurosystem halten, wobei die Mindestreserven nur im Monatsdurchschnitt gehalten werden m{\"u}ssen. Gehaltene verzinste Mindestreserven werden dabei mit dem Zinssatz f{\"u}r das Hauptrefinanzierungsgesch{\"a}ft verzinst.

\subsubsection{Geldpolitische Strategien in Europa}
::anmerkung:: Geldmenge M3 erkl{\"a}ren

Die geldpolitische Strategie des Eurosystems ist von Rat der Europ{\"a}ischen Zentralbanken entwickelt und am 13.10.1998 der {\"o}ffentlichkeit pr{\"a}sentiert worden.
Zentrale Elemente sind seitdem:
\begin{itemize}
 \item Das Hauptziel der Preisstabilit{\"a}t, definiert als Anstieg des sogenannten Harmonisierten Verbraucherpreisindex (HVPI) f{\"u}r den Euroraum von unter 2 Prozent\footnote[78]{Seit einf{\"u}hrung des Euros ist diese Marke allerdings jedes Jahr {\"u}berschritten worden, gerade in der Finanzkrise mit mehr als 1 Prozent j{\"a}hrlich.} Allerdings muss die Preisstabilit{\"a}t nur mittelfristig gew{\"a}hrleistet werden.
 %\cite{EZB 2004, S.52}
 \item Eine Geldmengenpolitik mit der Verk{\"u}ndung eines Referenzwertes f{\"u}r das Wachstum der Geldmenge M3,
 \item Beobachtung und Einsch{\"a}tzung der k{\"u}nftigen Preisentwicklung sowie der Preisstabilit{\"a}t des Euroraums insgesamt.
\end{itemize}
Als Oberstes Ziel der Geldpolitik ist die Preisstabili{\"a}t ausgegeben, wobei das Ziel 2003 dahingehend pr{\"a}zisiert wurde, dass mittelfristig eine Preissteigerungsrate unter, aber ann{\"a}hernd 2 Prozent gegen{\"u}ber den Vorjahr sein muss. Hiermit wird auch versucht, deflation{\"a}re Geldpolitik seitens der Nationalen Zentralbanken zu unterbinden.\footnote[101]{\citep[vgl.][S.564-568]{Basseler2010}}

Durch die Priorisierung der Preisstabilit{\"a}t als Hauptziel europ{\"a}ischer Geld(mengen-)politik, nimmt die Analyse und Beurteilung zuk{\"u}nftiger Preisentwicklung zentrale Rolle in der Arbeit der EZB ein. Diese Arbeit ist als eine sogenannte "`Zwei-S{\"a}ulen-Strategie"' umschrieben worden\footnote[95]{\citep[S.568]{Basseler2010}}, welche zum einen die wirtschaftliche Analyse mit der Beobachtung von kurz- und mittelfristigen realwirtschaftlichen Bestimmungsfaktoren der Preisentwicklung umfasst,
zum Anderen durch eine monet{\"a}re Analyse der Entwicklung der Geldmengen, vor allem der Geldmenge M3 als mittel- bis langfristigen Faktor der Preisentwicklung erg{\"a}nzt.
W{\"a}hrend sich die wirtschaftliche Analyse mit de Zusammenspiel von Angebot- und Nachfrageentwicklung an den G{\"u}ter-, Dienstleistungs- und Faktorm{\"a}rkten zur Vermeidung von Inflation sehr Keynsianisch gepr{\"a}gt ist, so wird die Geldmengenanalyse aus monetaristischer Sichtweise betrieben\footnote[97]{\citep[S.568]{Basseler2010}}. Somit l{\"a}sst sich die europ{\"a}ische Geldpolitik als ein Kompromiss, eine Mischung aus unterschiedlichen geldpolitischen Ansichten und geldtheorietischen Positionen verstehen. Eine vereinigte Geldpolitik Europas.\footnote[99]{\citep[vgl.][S.558f]{Basseler2010}}

\clearpage









\section{Einblick in die Grundgedanken der {\"O}konomen des 19. und 20. Jahrhunderts}
  \label{sec3:stroemungen}
  TEXT TEXT TEXT


\subsection{{\"U}bersicht der zu behandelnden Str{\"o}mungen und Begr{\"u}ndung der Auswahl}

TEXT TEXT TEXT





\subsection{Einblick in die Grundgedanken der Keynsianischen Schule}

Lord John Maynard Keynes\footnote[14]{Geboren:  5. Juni 1883 in Cambridge; Gestorben: 21. April 1946 in Tilton, im weiteren Keynes}





\subsection{Einblick in die Grundgedanken der {\"O}stereichischen Schule (Hayek)}











\subsection{Einblick in die Grundgedanken der Monetaristen (Milton Friedman)}
Milton Friedman \footnote[16]{Geboren: 31. Juli 1912 in Brooklyn, New York City; Gestorben: 16. November 2006 in San Francisco, im Weiteren Friedman}



"`Die Monetaristen, an der Spitze Milton Friedman und die Chicagoer Schule,
gehen von der Stabilit{\"a}t des privaten Sektors aus und sehen im
funktionsf{\"a}higen Marktmechanismus die Garantie, dass die Pl{\"a}ne der
Wirtschaftssubjekte optimal koordiniert werden, mit der Folge eines
permanenten Trends zur Vollbesch{\"a}ftigung der Produktionsfaktoren."' \footnote[501]{\citep[Vgl.][S.210]{peters2000}}

Sie hatten die Auffassung, das starke Impulse von der Geldpolitik ausgehen und
dass sie diese stabilisierende Wirkung aus{\"u}bt.\footnote[502]{\citep[S.181]{bombach1981theorie}}
 Friedman folgert aber, dass die stabilisierende Wirkung der Geldpolitik nur bei konsequenter Verfolgung
der Ziele gehalten werden kann. Das ruft dann einen Effekt des Vertrauens
hervor und zwar schreibt Friedman: "`Unser Wirtschaftssystem wird dann am
besten funktionieren, wenn Produzenten und Konsumenten, Arbeitgeber und
Arbeitnehmer(...) darauf vertrauen k{\"o}nnen, dass das durchschnittliche
Preisniveau sich zuk{\"u}nftig nach bekannten Regeln gestaltet - und zwar, dass
es vorzugsweise sehr stabil sein wird."'\footnote[503]{\citep[vgl.][S.150]{friedman1970die}}

St{\"o}rungen des Wirtschaftsablaufs bzw. extreme Besch{\"a}ftigungsschwankungen
werden nach Meinungen der Monetaristen durch Staatseingriffe verursacht. Wenn
z.B. die Staatsausgaben erh{\"o}ht werden tritt der Staat als zus{\"a}tzlicher
Nachfrager auf dem Geld- und Kreditmarkt auf. Doch wird dadurch keine
Mehrnachfrage erzeugt, sondern lediglich die private Nachfrage aus diesem
Sektor ersetzt. Nach monetaristischer Lehre bleibt das Preisniveau stabil,
wenn die Geldmenge dem Produktionspotential entspricht. Demzufolge ist die
Ursache f{\"u}r eine Inflation eine zu hohe Geldmengenausweitung im Verh{\"a}ltnis zum
Wachstum der G{\"u}terproduktion.\footnote[504]{\citep[Vgl.][S.213]{peters2000}}
Friedman schl{\"a}gt also vor: "`Geldmengenzuwachs mit j{\"a}hrlich konstanten Raten, wobei die Wachstumsrate so zu bemessen ist, dass ann{\"a}hernd stabile Endproduktpreise resultieren."' \footnote[505]{\citep[vgl.][S.132]{friedman1970die}} Somit kann mit dieser
Stabilit{\"a}t durch eine "`gem{\"a}{"s}igte Geldmengenausweitung"' eine Inflation und auch
Deflation verhindert werden.\footnote[506]{\citep[vgl.][S.155]{friedman1970die}}





\subsubsection{Friedman {\"u}ber Keynes}
Friedmans Sichtweise auf Keynes ist verst{\"a}ndlicherweise durch einen zeitlichen Abstand undden sich damit aufgezeigten Problemen der Keynes'schen Fiskalpolitik gepr{\"a}gt.  "`Das Komplement zur Kenes'schen Vernachl{\"a}ssigung des Geldes war die Betonung der
Fiskalpolitik... Besonders in den USA haben sich die Staatsausgaben als das wohl instabilste Element in der Wirtschaft der Nachkriegszeit erwiesen... Dies f{\"u}hrte zu einer Wiederbetonung des flexiblen Instruments der Geldpolitik."'\footnote[801]{\citep[S.105]{friedman1970die}}


 Eine Ursache der Fikalpolitischen Probleme sieht Friedman dabei in der Betrachtung kurzfristiger Entwicklungen zur Hinf{\"u}hrung in ein Gleichgewicht indem die Marktanpassung durch staatliche Eingriffe ersetzt werden.\footnote[802]{\citep[S.127]{friedman1970die}}

Doch empfindet Friedman dabei auch Bewunderung f{\"u}r Keynes. Obwohl er (Keynes) in Anbetracht der Krise von 1929 den besten Weg,
um aus ihr heraus zu kommen, in Fiskalpolitischen Empfehlungen sah, habe er doch ein gutes
Verst{\"a}ndnis von Geldtheorie."`Bei Keynes finde ich die Gelttheorie scharfsinnig und
modern, jedoch seine politischen Empfehlungen inakzeptabel."' Tats{\"a}chlich verteidigt
Friedman sogar Keynes'  "`Fehlentscheidungen"' als Fehlinterpretationder Periode von 1929
bis 1933 \footnote[803]{\citep[S.121]{friedman1970die}}:"`Fr{\"u}her war Keynes ein eifriger Verfechter der Ansicht gewesen, dass man sich
prim{\"a}r auf die orthodoxe Geldpolitik verlassen m{\"u}sse, wolle man die Stabilit{\"a}t der
Wirtschaft f{\"o}rdern. Er gab diese Meinung auf, als er entdeckte, dass die Liquidit{\"a}ts-
pr{\"a}ferenz Versuche, die langfristigen Zinss{\"a}tze zu {\"a}ndern durchkreuzen konnte... wandte er
sich stattdessen der Fiskalpolitik zu. \footnote[804]{\citep[S.126]{friedman1970die}}"'




\section{Case Study}
\label{sec4:CaseStudy}
Ein Instrument herausgreifen, bzw. eine Instrumentendiskussion f{\"u}hren.




\clearpage
\ac{ESZB}
\ac{EZB}

\clearpage

\section{Abk{\"u}rzungsverzeichnis}
	\label{sec5:Abkuerzungsverzeichnis}

\begin{acronym}[ESZB]

 	\acro{EG}{Europ{\"a}ische Gemeinschaft}
	\acro{EGV}{Europ{\"a}ische Gemeinschaftsvertr{\"a}ge}
  	\acro{ESZB}{Europ{\"a}isches System der Zentralbanken}
  	\acro{EU}{Europ{\"a}ische Union}
	\acro{EZB}{Europ{\"a}ische Zentralbank}
 	\acrodefplural{EZB}{Europ{\"a}ische Zentralbanken}


\end{acronym}

%%%% LITERATURE %%%%%
\vspace{10pt}
	\newpage
  	\singlespacing

% Literaturliste endgueltig anzeigen
	         %\bibliographystyle{diss_fk}

\bibliographystyle{authordate1}
\section{Literaturverzeichnis}
	\label{sec6:Literaturverzeichnis}
	\bibliography{literatur_EWA}	% Sie benoetigen eine *.bib-Datei
	\newpage





%%%% APPENDIX %%%%%
	%\section*{Anhang}
	%\label{sec:anhang}

\end{document}
