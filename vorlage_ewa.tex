\documentclass[
onecolumn,
a4paper,
abstracton,
parskip=half
%,draft
,final
]{scrartcl}

\usepackage[pdftex
,draft
]{graphicx}

\usepackage{booktabs}

\usepackage[cmex10]{amsmath}

\interdisplaylinepenalty=2500
\usepackage{url}

\usepackage[breaklinks]{hyperref}
\hyphenation{nothing} % correct bad hyphenation here

\usepackage{eurosym}

\usepackage{listings}
\lstset{basicstyle=\small\ttfamily,breaklines=true}
\emergencystretch 1000pt

\usepackage{subcaption}

\usepackage{mathtools}

% deutsche Silbentrennung
\usepackage[ngerman]{babel}

\usepackage[printonlyused, withpage]{acronym}

\usepackage[a4paper]{geometry}

% wegen deutschen Umlauten
\usepackage[ansinew]{inputenc}

% fuer Zitate
\usepackage[round]{natbib}

\usepackage{setspace}
\usepackage{tabulary}
\usepackage{units}
\usepackage{cite}
%%% Deutsche Verzeichnis-ueberschriften

\renewcommand{\contentsname}{Inhalt}
\renewcommand{\listtablename}{Tabellenverzeichnis}
\renewcommand{\listfigurename}{Abbildungsverzeichnis}

%%% Kommentarfunktion %%%
\usepackage[textwidth=2.2cm
,obeyFinal
]{todonotes}


\begin{document}

%%%%%% CREATING THE TITLE %%%%%
\titlehead{
\centering
\includegraphics[width = 0.3\textwidth ]{logo_wip.jpg} \\
\small Technische Universit{\"a}t Berlin \\
Fakult{\"a}t VII (Wirtschaft \& Management) \\
Fachgebiet Wirtschafts- und Infrastrukturpolitik (WIP)
}
\title{Geldpolitische Str{\"o}mungen und Instrumente}
%\subtitle{Untertitel}
\author{
Gregor May
\footnotesize \textit{(Matr: 357150)}
\and
Marius Hanniske
\footnotesize \textit{(Matr: XXXXX)}
}

\date{\today}

\maketitle

%%%% ABSTRACT %%%%%

\selectlanguage{ngerman}
\begin{abstract}
ZUSAMMENFASSUNG
\ldots

\end{abstract}


\begin{flushleft}
%\nointend \textbf{JEL Codes:} C61, H54, L94 \\
%\nointend \textbf{Keywords:} Money, Money Politics, Econimics
\end{flushleft}

\tableofcontents
\listoffigures
\listoftables

\newpage
\onehalfspacing

%%%%% TEXTK{\"O}RPER %%%%%

\section{Einleitung}
\label{sec1:einleitung}

Es war Einmal...
Das Leben des 1671 in Schottland geborenen John Law liest sich wie ein M{\"a}rchen \citep[vgl.][Kap. 2]{strathern2006schumpeters}. Ein Draufg{\"a}nger im Kasino und mit den Frauen. In England zum Tode verurteilt, weil er als Sieger aus einem Duell hervorging, floh er nach Europa und bekam in den dortigen Kasinos und der fortschrittlicheren Finanzpolitik anderer L{\"a}nder, wie z.B. die Niederlande \footnote[2]{Belebung der Schifffart dank Windm{\"u}hlen, erster B{\"o}rsencrash auf dem TulpenmarktText} ein f{\"u}r diese Zeit, neues Verst{\"a}ndnis von Geld: "Ungenutztes Geld war nichts - nichts als das Potential f{\"u}r Aktion." \citep[vgl.][Kap. 2]{strathern2006schumpeters}Law erkannte rasch, dass mehr Bargeld zu mehr Handelsaktivit{\"a}t f{\"u}hrt und unterbreitete seine Ideen u.a. in Schottland und Turin. Die Staaten k{\"o}nnen Noten auf sich selbst ausstellen,d.h. Kredite gew{\"a}hren im Tausch mit der F{\"a}higkeit, in Zukunft Geld aufzubringen, mit Hilfe von Steuern. Dieses Geld wird in England noch heute fiat money [4] genannt und ist heute in Europa und auch den USA\footnote[5]{"In God we trust" steht u.a. auf der 1-Dollar-Note und Gottesvertrauen braucht man auch} gel{\"a}ufig.

Durch Freunde bekam Law Kontakt zum Regenten von Frankreich\footnote[6]{Philippe , Herzog von OrlÈans, nachdem der Sonnenk{\"o}nig Ludwig 14. 1715 starb}, der ihm vollstes Vertrauen entgegen brachte. Die franz{\"o}sische Staatskasse, die zu der Zeit 3 milliarden Livres \footnote[7]{Livre frz.: Pfund war vom 9. bis zum 18. Jh. die franz{\"o}sische W{\"a}hrung, 1795 abgel{\"o}st vom Franc} an Schulden angeh{\"a}uft hat, ben{\"o}tigte neue Ideen und Law war der Mann der sie hatte. Er gr{\"u}ndete 1716 die Banque National, die erste Bank Frankreichs, ausgestattet mit 6 Millionen Livres Aktienkapital. Bareinzahlungen waren in Form von M{\"u}nzen auf Konten m{\"o}glich, auch {\"U}berweisungen durch Schecks auf andere Konten. Was aber neu und besonders war stellte "raison d'etre" \footnote[8]{gedeckt durch die 6 Millionen Aktienkapital, Versprechen(!) auf Auszahlung, allerdings Barreserven von 350 tausend Livres} dar, d.h. die Ausgabe von Papiergeld. Das war die grandiose Idee und sie funktionierte. Die Leute hatten mehr Geld zur Verf{\"u}gung und gaben es auch aus, die G{\"u}ternachfrage stieg an und damit auch die G{\"u}terproduktion. Und einen weiteren Plan setzte Law um, die Gr{\"u}ndung der Mississippi-Gesellschaft. Philippe {\"u}bereignete ihm dazu Louisana und mit der Ausgabe von Aktien sollte damit eine Expedition \footnote[9]{ebenda} finanziert werden. Law und seine Familie bewegten sich in den h{\"o}chsten Kreisen, ja befanden sich f{\"o}rmlich an der Spitze, der High Societie. Der p{\"a}bstliche Nimbus, der bewegt war zur Geburtstagsfeier von Laws Tochter eingeladen zu sein, gab der Tochter einen Kuss und ihr {\"a}lterer Bruder ging mit Ludwig 15. jagen. Wenn das wirklich ein M{\"a}rchen gewesen w{\"a}re w{\"u}rde es jetzt mit den Worten: "... und wenn sie nicht gestorben sind, Leben sie noch Heute", enden. Doch es war wie schon erw{\"a}hnt kein M{\"a}rchen und das SystemË, wie Law es nannte, brach in sich zusammen. Die Gewinne wurden nicht wie versprochen f{\"u}r eine Expedition ausgegeben, sondern zur Begleichung der imensen Staatsschuld. Bei einer Abwertung der Noten verloren die Menschen endg{\"u}ltig das Vertrauen in diese.
Wie schon erw{\"a}hnt funktioniert die heutige Geldpolitik wie bei dem von Law durchgef{\"u}hrten Feldexperiment, doch Sicherheitsmechanismen verhindern meist eine derartige Eskalation.
Wir befassen uns in dieser Hausarbeit mit der Ausarbeitung dieser Sicherheitsmechanismen durch verschiedene Denkans{\"a}tze, Funktionen die die Instabilit{\"a}t des Systems vertr{\"a}glicher f{\"u}r seine Benutzer gemacht haben. Dabei gehen wir auf die Denkanst{\"o}{\ss}e der Str{\"o}mungen des Keynesianismus, der Monetaristen und der Gruppe der neuen politischen ÷konomie, der ÷sterreichischen Schule ein.




\subsection{Benennung der Fragestellung}








\begin{itemize}
\item Einblick in die Grundgedanken der {\"O}konomen des 19. und 20. Jahrhunderts
\item Sichtweise auf den Staat: Aktive oder passive Rolle?
\item Welche Mittel zur Steuerung der Wirtschaft schl{\"a}gt diese Str{\"o}mung vor?
\item Sind die Mittel kontr{\"a}r zur derzeitigen Geldpolitk?
\item Ist diese Str{\"o}mung Nachfrage oder Angebotsseitig?
\item Ist eine Kredit-/Schuldentheorie mit integriert?
\item Handelt es sich um eine derzeit in Anwendung befindliche Str{\"o}mung?
\end{itemize}



\subsection{Beschreibung des eigenen Verst{\"a}ndnis von Geldpolitik}

\clearpage

\section{Technisches System}
\label{sec1:technischesSystem}
TEXT TEXT TEXT
\subsection{Institutionen} Von Ba{\ss}eler
\subsection{Ziel}  Von Ba{\ss}eler
\subsection{Instrumente}  Von Ba{\ss}eler

Die Europ{\"a}ische Geldpolitik wird durch das \ac{ESZB} und die \ac{EZB} organisiert, wobei sich die \ac{ESZB} aus der \acf{EZB} und allen 27 nationalen Zentralbanken (NZBen) der Mitgliedsstaaten der \ac{EU} organisiert. Sonderstatus haben dabei die sogenannten >>Outs<<, jene Mitgliederstaaten der \ac{EU}, die den Euro noch nicht eingef{\"u}hrt haben. Dies sind Derzeit: D{\"a}nemark, Gro{\ss}-britanien, Schweden sowie die meisten neuen \acs{EU}-Mitgliedsstaaten nach 2001. Sie sind vom Entscheidungsprozess der \acs{ESZB} ausgeschlossen und vollziehen eine eigenst{\"a}ndige nationale Geldpolitik. Auch wenn der \acs{EG}-Vertrag (EGV, Art. 105-109 d) formal zwischen \ac{EZB} und \ac{ESZB} unterscheidet, entscheidet doch faktisch nur eine Institution, die der \ac{EZB} mit ihren Beschlussorganen (EZB-Rat und Direktorium der \ac{EZB}.) (Vergleiche
%\cite{Baߟeler2010}[S.553])


\clearpage

\subsection{Organisationen der Europ{\"a}ischen Geldpolitk}

Die Europ{\"a}ische Geldpolitik wird durch das Europ{\"a}ische System der Zentralbanken (ESZB) und die EZB organisiert, wobei sich die ESZB aus der Europ{\"a}ischen Zentralbank und allen 27 nationalen Zentralbanken (NZBen) der Mitgliedsstaaten der EU organisiert.
Sonderstatus haben dabei die sogenannten >>Outs<<, jene Mitgliederstaaten der EU, die den Euro  nicht eingef{\"u}hrt haben. Dies sind Derzeit: D{\"a}nemark, Großbritanien, Schweden sowie die meisten neuen EU-Mitgliedsstaaten nach 2001. Sie sind vom Entscheidungsprozess der ESZB ausgeschlossen und vollziehen eine eigenst{\"a}ndige nationale Geldpolitik.
Auch wenn der EG-Vertrag (EGV, Art. 105-109 d) formal zwischen EZB und ESZB unterscheidet, entscheidet doch faktisch nur eine Institution, die der EZB mit ihren Beschlussorganen (EZB-Rat und Direktorium der EZB.) Vergleiche  %\cite{Basseler2010}[S.553] )

\subsection{Die Europ{\"a}ische Zentralbank}
Die Beschlussorgane der Europ{\"a}ischen Zentralbank sind der EZB-Rat und das Direktorium der EZB, welche gemeinsam die EZB leiten. Das Direktorium der EZB besteht aus dem Pr{\"a}sidenten und dem Vizepr{\"a}sidenten der EZB, sowie weiteren vier Mitgliedern, die von den Regierungen der Mitgliedsstaaten auf der Ebene der Staats- und Regierungschefs auf Empfehlung des EU-Rats einvernehmlich ernennt. Ein Anh{\"o}rungsrecht besteht dabei beim Europ{\"a}ischem Parlament und beim EZB-Rat.
Der EZB-Rat wiederum besteht aus dem Direktorium und den Pr{\"a}sidenten aller nationalen Zentralbanken, die den Euro gemeinsam eingef{\"u}hrt haben.
Die exekutive Gewalt liegt innerhalb der Europ{\"a}ischen Zentralbank beim Direktorium, das die f{\"u}r die Durchf{\"u}hrung der Geldpolitik nach den Leitlinien und Beschl{\"u}ssen des EZB-Rates verantwortlich ist. Auch ist das Direktorium der EZB weisungsbefugt gegen{\"u}ber den nationalen Zentralbanken des Eurosystems.  Damit haben wir ein duales System, bestehend aus dem Exekutivorgan der EZB in Form des Direktoriums und ein ein Beschlussorgan in Form des EZB-Rates.
Der EZB-Rat erarbeitet und erl{\"a}sst die Beschl{\"u}sse einer gemeinschaftlichen europ{\"a}ischen Geldpolitik des Euro-Raumes, um die Ausgabe von M{\"u}nzen und Banknoten zu regeln und die Erf{\"u}llung von dem ESZB {\"u}bertragenen Aufgaben zu erf{\"u}llen.  (Vergleiche  \citep{Basseler2010}  ) Derzeit umfasst der EZB-Rat 18 L{\"a}nder und das Direktorium, abgestimmt wird mit einfacher Mehrheit der Anwesenden, bei Stimmengleichheit entscheidet die Stimme des Pr{\"a}sidenten.


Instrumente der Europ{\"a}ischen Geldpolitik

Im Folgenden soll es um die Instrumente der Europ{\"a}ischen Geldpolitik durch die EZB gehen, als da w{\"a}ren:
1. Die Offenmarktpolitik
2. Die Politik der St{\"a}ndigen Fazilit{\"a}ten
3. Die Mindestreservepolitik.
4. Weitere Instrumente

\subsubsection{Die Offenmarktpolitik}
Nach g{\"a}ngiger Politik- und Wirtschaftstheorie bedarf es in einer wachsenden Volkswirtschaft, um eine hinreichende Geldversorgung der Wirtschaft zu gew{\"a}hrleisten, einer fortw{\"a}hrenden Ausweitung der nominalen Geldmenge. Als zentrales Instrument der Geldpolitik kommt hier die Offenmarktpolitik der Europ{\"a}ischen Zentralbank zum Tragen.
Vereinfacht gesprochen, wird unter der Offenmarktpolitk nichts anderes verstanden, als der An- und Verkauft von Wertpapieren gegen Zentralbankgeld durch die europ{\"a}ische Zentralbank. Bezweckt wird mit den Offenmarktk{\"a}ufen, bzw. -verk{\"a}ufen eine Zentralbankgeldsch{\"o}pfung bzw. -vernichtung zur Ver{\"a}nderung der nominalen Geldmenge. Allerdings ist der Begriff "Offen"-Marktgesch{\"a}ft irref{\"u}hrend: In der ESZB sind als Gesch{\"a}ftspartner der EZB nur finanziell solide MFIs zugelassen, die in das Mindestreservesystem einbezogen sind.

Bei Offenmarktgesch{\"a}ften erh{\"o}ht sich der Bestand an zentralbankgeld der Gesch{\"a}ftsbanken beim Kauf von Wertpapieren, doch beim Verkauft von Wertpapieren sinkt der Bestand an Zentralbankgeld der Gesch{\"a}ftsbanken. Der Kauf von Wertpapieren durch die Zentralbank wird mitunter auch expansive Offenmarktpolitik genannt.
Wichtig ist zu wissen, dass die Europ{\"a}ische Zentralbank die Gesch{\"a}ftsbanken nicht zum Kauf von Wertpapieren zwingen kann und somit attraktive Konditionen bieten muss: So sinkt bei einer geplanten expansiven Offenmarktpolitik der Zinssatz f{\"u}r die Zentralbankgeld-Kreditgew{\"a}hrung unter den {\"u}blichen Geldmarktzins. Entsprechend umgekehrt ist es bei der {\"a}ußerst selten vorkommenden, sogenannten kontraktiven Offenmarktpolitik. Hier m{\"u}ssen die von der Zentralbank angebitenen Zinss{\"a}tze h{\"o}her sein, als die sonst {\"u}blichen Geldmarktzinsen.

Offenmarktgesch{\"a}fte treten h{\"a}ufig in der Form von Repo-Gesch{\"a}ften auf. Dabei handelt es sich nach der englischen Bezeichnung Repurchase um R{\"u}ckkauf-Gesch{\"a}fte, d.h. es wird mit den Gesch{\"a}ftsbanken eine Laufzeit der Wertpapiere vereinbart, an deren Ende diese ihre Wertpapiere zur{\"u}ckkaufen m{\"u}ssen. Hiermit entsteht ein automatischer R{\"u}ckfluss von Zentralbankgeld. Diese auf Frist gesetzten Repo-Gesch{\"a}fte werden auch Wertpapierpensionsgesch{\"a}fte genannt, der entsprechend vorkommende Zinssatz wird demgem{\"a}ß als Pensionssatz bezeichnet.
Die Repo-Gesch{\"a}fte sind aus Sicht der g{\"a}ngigen Wirtschaftstheorie ein gut steuerbares Instrument f{\"u}r die Entwicklung der Zentralbankgeldmenge im Gesch{\"a}ftsbankensektor, da sich auch kontraktive Effekte einstellen, wenn keine neuen Repo-Gesch{\"a}fte abgeschlossen werden. Eine r{\"u}ckl{\"a}ufige Anzahl an Repo-Gesch{\"a}ften f{\"u}hrt somit zu einer Abnahme der Zentralbankgeldmenge im Sektor der Gesch{\"a}ftsbanken.


\subsubsection{Die Politik der St{\"a}ndigen Fazilit{\"a}ten}
\subsubsection{Die Mindestreservepolitik.}
\subsubsection{Weitere Instrumente}





\ac{EG}



\section{Einblick in die Grundgedanken der {\"O}konomen des 19. und 20. Jahrhunderts}
\label{sec1:stroemungen}
TEXT TEXT TEXT


\subsection{{\"U}bersicht der zu behandelnden Str{\"o}mungen und Begr{\"u}ndung der Auswahl}

\subsection{Einblick in die Grundgedanken der Keynsianischen Schule}

\subsection{Einblick in die Grundgedanken der {\"O}stereichischen Schule (Hayek)}

\subsubsection{Hayek  {\"u}ber die Monetaristen}

In Entnationalisierung des Geldes {\"a}u{\ss}ert sich Hayek auch {\"u}ber die Monetaristen, \citep[vgl.][]{Hayek1977}



\subsection{Einblick in die Grundgedanken der Monetaristen (Milton Friedman)}


\clearpage
\ac{ESZB}
\ac{EZB}



\section{Case Study}
\label{sec1:caseStudy}
Ein Instrument herausgreifen, bzw. eine Instrumentendiskussion f{\"u}hren.

\subsection{}
\subsection{}
\subsection{}











\clearpage

\section{Abk{\"u}rzungsverzeichnis}


\begin{acronym}[ESZB]


 	\acro{EG}{Europ{\"a}ische Gemeinschaft}
	\acro{ESZB}{Europ{\"a}isches System der Zentralbanken}
  	\acro{EU}{Europ{\"a}ische Union}
	\acro{EZB}{Europ{\"a}ische Zentralbank}
 	 \acrodefplural{EZB}{Europ{\"a}ische Zentralbanken}



\end{acronym}

%%%% LITERATURE %%%%%

\vspace{10pt}
	\newpage
\singlespacing

% Literaturliste endgueltig anzeigen
	         %\bibliographystyle{diss_fk}
\bibliographystyle{authordate1}

\bibliography{literatur_EWA}	% Sie benötigen eine *.bib-Datei
\newpage





%%%% APPENDIX %%%%%
%\section*{Anhang}
%\label{sec:anhang}
%

\end{document}
