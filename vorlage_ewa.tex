\documentclass[
onecolumn,
a4paper,
abstracton,
parskip=half
%,draft
,final
]{scrartcl}

\usepackage[pdftex
,draft
]{graphicx}

\usepackage{booktabs}
\usepackage[cmex10]{amsmath}
\interdisplaylinepenalty=2500
\usepackage{url}
\usepackage[breaklinks]{hyperref}
\hyphenation{nothing} % correct bad hyphenation here
\usepackage[ansinew]{inputenc}
\usepackage{eurosym}
\usepackage{listings}
\lstset{basicstyle=\small\ttfamily,breaklines=true}
\emergencystretch 1000pt
\usepackage{subcaption}
\usepackage{mathtools}

\usepackage[ngerman]{babel}


\usepackage[a4paper]{geometry}

\usepackage{natbib}
\usepackage{setspace}
\usepackage{tabulary}
\usepackage{units}

%%% Deutsche Verzeichnis-�berschriften

\renewcommand{\contentsname}{Inhalt}
\renewcommand{\listtablename}{Tabellenverzeichnis}
\renewcommand{\listfigurename}{Abbildungsverzeichnis}

%%% Kommentarfunktion %%%
\usepackage[textwidth=2.2cm
,obeyFinal
]{todonotes}


\begin{document}

%%%%%% CREATING THE TITLE %%%%%
\titlehead{
\centering
\includegraphics[width = 0.3\textwidth ]{logo_wip.jpg} \\
\small Technische Universit{\"a}t Berlin \\
Fakult{\"a}t VII (Wirtschaft \& Management) \\
Fachgebiet Wirtschafts- und Infrastrukturpolitik (WIP)
}
\title{Geldpolitische Str{\"o}mungen und Instrumente}
%\subtitle{Untertitel}
\author{
Gregor May
\footnotesize \textit{(Matr: 357150)}
\and
Marius Hanniske
\footnotesize \textit{(Matr: XXXXX)}
}

\date{\today}

\maketitle

%%%% ABSTRACT %%%%%

\selectlanguage{ngerman}
\begin{abstract}
ZUSAMMENFASSUNG
\end{abstract}


\begin{flushleft}
%\nointend \textbf{JEL Codes:} C61, H54, L94 \\
%\nointend \textbf{Keywords:} Money, Money Politics, Econimics
\end{flushleft}

\tableofcontents
\listoffigures
\listoftables

\newpage
\onehalfspacing

%%%%% TEXTK{\"O}RPER %%%%%

\section{Einleitung}
\label{sec1:einleitung}
\subsection{Benennung der Fragestellung}

\begin{itemize}
\item Einblick in die Grundgedanken der {\"O}konomen des 19. und 20. Jahrhunderts
\item Sichtweise auf den Staat: Aktive oder passive Rolle?
\item Welche Mittel zur Steuerung der Wirtschaft schl{\"a}gt diese Str{\"o}mung vor?
\item Sind die Mittel kontr{\"a}r zur derzeitigen Geldpolitk?
\item Ist diese Str{\"o}mung Nachfrage oder Angebotsseitig?
\item Ist eine Kredit-/Schuldentheorie mit integriert?
\item Handelt es sich um eine derzeit in Anwendung befindliche Str{\"o}mung?
\end{itemize}



\subsection{Beschreibung des eigenen Verst{\"a}ndnis von Geldpolitik}

\clearpage

\section{Technisches System}
\label{sec1:technischesSystem}
TEXT TEXT TEXT
\subsection{Institutionen} Von Ba{\ss}eler
\subsection{Ziel}  Von Ba{\ss}eler
\subsection{Instrumente}  Von Ba{\ss}eler

\clearpage






\section{Einblick in die Grundgedanken der {\"O}konomen des 19. und 20. Jahrhunderts}
\label{sec1:stroemungen}
TEXT TEXT TEXT
\subsection{{\"U}bersicht der zu behandelnden Str{\"o}mungen und Begr{\"u}ndung der Auswahl}
\subsection{Einblick in die Grundgedanken der Keynsianischen Schule}

\subsection{Einblick in die Grundgedanken der Monetaristen}

\subsection{Einblick in die Grundgedanken der Planwirtschaftler}


\clearpage




\section{Case Study}
\label{sec1:caseStudy}
Ein Instrument herausgreifen, bzw. eine Instrumentendiskussion f{\"u}hren.

\subsection{}
\subsection{}
\subsection{}



\clearpage

%%%% LITERATURE %%%%%

\vspace{10pt}
%\newpage
\singlespacing
%bibliographystyle{diss_fk}
\bibliographystyle{authordate1}
\bibliography{literatur_EWA}	% Sie ben�tigen eine *.bib-Datei
\newpage

%%%% APPENDIX %%%%%

%\section*{Anhang}
%\label{sec:anhang}
%

\end{document}
