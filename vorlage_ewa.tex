\documentclass[
        onecolumn,
        a4paper,
        abstracton,
        parskip=half
        %,draft
        ,final
        ]{scrartcl}

        \usepackage[pdftex
        ,draft
        ]{graphicx}

        \usepackage{booktabs}

        \usepackage[cmex10]{amsmath}

        \interdisplaylinepenalty=2500
        \usepackage{url}

        \usepackage[breaklinks]{hyperref}
        \hyphenation{nothing} % correct bad hyphenation here

        \usepackage{eurosym}

        \usepackage{listings}
        \lstset{basicstyle=\small\ttfamily,breaklines=true}
        \emergencystretch 1000pt

        \usepackage{subcaption}

        \usepackage{mathtools}

        % deutsche Silbentrennung
        \usepackage[ngerman]{babel}

        \usepackage[printonlyused, withpage]{acronym}

        \usepackage[a4paper]{geometry}

        % wegen deutschen Umlauten
        \usepackage[ansinew]{inputenc}

        % fuer Zitate
        \usepackage[round]{natbib}

        \usepackage{setspace}
        \usepackage{tabulary}
        \usepackage{units}
        \usepackage{cite}
        %%% Deutsche Verzeichnis-ueberschriften

        \renewcommand{\contentsname}{Inhalt}
        \renewcommand{\listtablename}{Tabellenverzeichnis}
        \renewcommand{\listfigurename}{Abbildungsverzeichnis}

        %%% Kommentarfunktion %%%
        \usepackage[textwidth=2.2cm
        ,obeyFinal
        ]{todonotes}

\begin{document}

%%%%%% CREATING THE TITLE %%%%%
\titlehead{
          \centering
          \includegraphics[width = 0.3\textwidth ]{logo_wip.jpg} \\
          \small Technische Universit{\"a}t Berlin \\
          Fakult{\"a}t VII (Wirtschaft \& Management) \\
          Fachgebiet Wirtschafts- und Infrastrukturpolitik (WIP)
          }
          \title{Geldpolitische Str{\"o}mungen und Instrumente}
          %\subtitle{Untertitel}
          \author{
          Gregor May
          \footnotesize \textit{(Matr: 357150)}
          \and
          Marius Hanniske
          \footnotesize \textit{(Matr: 311263)}
          }

          \date{\today}

          \maketitle

%%%% ABSTRACT %%%%%

\selectlanguage{ngerman}

\begin{abstract}
ZUSAMMENFASSUNG
\ldots

\end{abstract}


  \begin{flushleft}
  %\nointend \textbf{JEL Codes:} C61, H54, L94 \\
  %\nointend \textbf{Keywords:} Money, Money Politics, Econimics
  \end{flushleft}

    \tableofcontents
    \listoffigures
    \listoftables

    \newpage
    \onehalfspacing

  %%%%% TEXTK{\"O}RPER %%%%%

\section{Einleitung}
    \label{sec1:einleitung}

    Es war Einmal\ldots
    Das Leben des 1671 in Schottland geborenen John Law liest sich wie ein M{\"a}rchen \citep[vgl.][Kap. 2]{strathern2006schumpeters}. Ein Draufg{\"a}nger im Kasino und mit den Frauen. In England zum Tode verurteilt, weil er als Sieger aus einem Duell hervorging, floh er nach Europa und bekam in den dortigen Kasinos und der fortschrittlicheren Finanzpolitik anderer L{\"a}nder, wie z.B. die Niederlande \footnote[2]{Belebung der Schifffart dank Windm{\"u}hlen, erster B{\"o}rsencrash auf dem TulpenmarktText} ein f{\"u}r diese Zeit, neues Verst{\"a}ndnis von Geld: "Ungenutztes Geld war nichts - nichts als das Potential f{\"u}r Aktion." \citep[vgl.][Kap. 2]{strathern2006schumpeters}Law erkannte rasch, dass mehr Bargeld zu mehr Handelsaktivit{\"a}t f{\"u}hrt und unterbreitete seine Ideen u.a. in Schottland und Turin. Die Staaten k{\"o}nnen Noten auf sich selbst ausstellen,d.h. Kredite gew{\"a}hren im Tausch mit der F{\"a}higkeit, in Zukunft Geld aufzubringen, mit Hilfe von Steuern. Dieses Geld wird in England noch heute fiat money [4] genannt und ist heute in Europa und auch den USA\footnote[5]{"In God we trust" steht u.a. auf der 1-Dollar-Note und Gottesvertrauen braucht man auch} gel{\"a}ufig.

    Durch Freunde bekam Law Kontakt zum Regenten von Frankreich\footnote[6]{Philippe , Herzog von OrlÈans, nachdem der Sonnenk{\"o}nig Ludwig 14. 1715 starb}, der ihm vollstes Vertrauen entgegen brachte. Die franz{\"o}sische Staatskasse, die zu der Zeit 3 milliarden Livres \footnote[7]{Livre frz.: Pfund war vom 9. bis zum 18. Jh. die franz{\"o}sische W{\"a}hrung, 1795 abgel{\"o}st vom Franc} an Schulden angeh{\"a}uft hat, ben{\"o}tigte neue Ideen und Law war der Mann der sie hatte. Er gr{\"u}ndete 1716 die Banque National, die erste Bank Frankreichs, ausgestattet mit 6 Millionen Livres Aktienkapital. Bareinzahlungen waren in Form von M{\"u}nzen auf Konten m{\"o}glich, auch {\"U}berweisungen durch Schecks auf andere Konten. Was aber neu und besonders war stellte "raison d'etre" \footnote[8]{gedeckt durch die 6 Millionen Aktienkapital, Versprechen(!) auf Auszahlung, allerdings Barreserven von 350 tausend Livres} dar, d.h. die Ausgabe von Papiergeld. Das war die grandiose Idee und sie funktionierte. Die Leute hatten mehr Geld zur Verf{\"u}gung und gaben es auch aus, die G{\"u}ternachfrage stieg an und damit auch die G{\"u}terproduktion. Und einen weiteren Plan setzte Law um, die Gr{\"u}ndung der Mississippi-Gesellschaft. Philippe {\"u}bereignete ihm dazu Louisana und mit der Ausgabe von Aktien sollte damit eine Expedition \footnote[9]{ebenda} finanziert werden. Law und seine Familie bewegten sich in den h{\"o}chsten Kreisen, ja befanden sich f{\"o}rmlich an der Spitze, der High Societie. Der p{\"a}bstliche Nimbus, der bewegt war zur Geburtstagsfeier von Laws Tochter eingeladen zu sein, gab der Tochter einen Kuss und ihr {\"a}lterer Bruder ging mit Ludwig 15. jagen. Wenn das wirklich ein M{\"a}rchen gewesen w{\"a}re w{\"u}rde es jetzt mit den Worten: "\ldots und wenn sie nicht gestorben sind, Leben sie noch Heute", enden. Doch es war wie schon erw{\"a}hnt kein M{\"a}rchen und das SystemË, wie Law es nannte, brach in sich zusammen. Die Gewinne wurden nicht wie versprochen f{\"u}r eine Expedition ausgegeben, sondern zur Begleichung der imensen Staatsschuld. Bei einer Abwertung der Noten verloren die Menschen endg{\"u}ltig das Vertrauen in diese.
    Wie schon erw{\"a}hnt funktioniert die heutige Geldpolitik wie bei dem von Law durchgef{\"u}hrten Feldexperiment, doch Sicherheitsmechanismen verhindern meist eine derartige Eskalation.
    Wir befassen uns in dieser Hausarbeit mit der Ausarbeitung dieser Sicherheitsmechanismen durch verschiedene Denkans{\"a}tze, Funktionen die die Instabilit{\"a}t des Systems vertr{\"a}glicher f{\"u}r seine Benutzer gemacht haben. Dabei gehen wir auf die Denkanst{\"o}{\ss}e der Str{\"o}mungen des Keynesianismus, der Monetaristen und der Gruppe der neuen politischen  {\"O}konomie, der {\"O}sterreichischen Schule ein.


\subsection{Benennung der Fragestellung}

    -> Einblick in die Grundgedanken der {\"O}konomen des 19.und 20.Jahrhunderts

    \ldots Doch am Anfang steht die Forschung. Dabei sind zwei Motive herausgearbeitet \footnote[9]{der Keynesianismus 1},
    die die Triebfedern der wissenschaftlichen Forschung darstellen:
    1.) die Neugierde nach Wissen {\"u}ber die Welt, sie ist eine regelm{\"a}{\ss}ige Motivation zur Forschung, wie auch John Law versucht hat seine Ideen umzusetzen um Herauszufinden
     ob sie funktionieren. Eine 2.) unabh{\"a}ngig davon treibende Kraft ist die Unvollst{\"a}ndige Information, die daf{\"u}r sorgt, dass die Menschen nur eine vage oder gar keine Kenntnis von der Zukunft besitzen. Die Angst der Menschen vor dem Unbekannten veranlasst sie dazu diese Unkenntnis zu beseitigen. Dieser Drang zur Ver{\"a}nderung der Situation schwindet jedoch, wenn die Zukunft vielversprechend aussieht.

    Nicht zu vernachl{\"a}ssigen sind die M{\"o}glichkeiten einer Ver{\"a}nderung, als Reaktion auf
    die entt{\"a}uschenden Ergebnisse wissenschaftlicher Theorien. Was tun, wenn sich eine
    Diskrepanz zwischen der Theorie und Realit{\"a}t aufzeigt? Die erste M{\"o}glichekeit:
    Evolution -> Anpassung der Theorie an die Realit{\"a}t. Schwierig, wenn die Theorie fest
    gefahren ist und auf Probleme mit den immer gleichen Argumentationen antworten will.
    Dann bleibt nur die zweite M{\"o}glichkeit der Revolution -> bei der eine v{\"o}llig neue
    Konstruktion der Analytik angefertigt wird, die die Realit{\"a}t besser wiedergibt als ihre
    Vorg{\"a}ngertheorie\ldots
    (Nicht zu vergessen das Falsifizierbarkeitskriterium: Wiederlegung einer Theorie hat ein
    st{\"a}rkeres Gewicht als ihre Best{\"a}tigun)
    Verdeutlichen wir uns das mal an den {\"o}konomischen Theorien des 19. und 20. Jahrhunderts,
    den Klassiker und Neoklassikern.
    Auf Kr{\"a}fte des freien Marktes berufen sich die Klassiker\footnote[10]{in dem Fall sind Klassiker und Neoklassiker zusammengefasst}, den Anfang machte dabei
     Adam Smith mit seinem 1776 ver{\"o}ffentlichten Werk "`Wohlstand der Nationen"'.
    "`Die freie Marktwirtschaft wird erkl{\"a}rt von einer Unsichtbaren Hand, die der
    Preismechanismus darstellt, {\"u}ber Angebot und Nachfrage, die in ein Gleichgewicht dr{\"a}ngen,\ldots"'

    Sie unterstellen den Individuen ein rationales Verhalten (homo oeconomicus), aus einer
     Vielzahl von Angeboten das Beste herauszufiltern. Preise dienen dazu die Angebote
    unterscheiden und sortieren zu k{\"o}nnen\ldots mit diesen Argumentationen  (der Realpreise\ldots[Die
    Preise von G{\"u}tern kommen nie unabh{\"a}ngig von Preisen anderer G{\"u}ter zustande])befassten sich
    vornehmlich die Neoklassiker. Die Preise wiederrum k{\"o}nnen bequemer zugeteilt,
    identifiziert und bewertet werden wenn sie eine (be)rechenbare Einheit erhalten.
    Diese Einheit, ob es nun US-Ameriaknische Dollar, Japanische Yen oder Europ{\"a}ische Euro
    sind, stellt das Geld ganz allgemein dar.( um nicht auf Zuf{\"a}lle hoffen zu m{\"u}ssen, dass
    ausgerechnet ein B{\"a}cker 2000 Br{\"o}tchen ben{\"o}tigt um die gegen einen neuen Fernseher eintauschen zu k{\"o}nnen [anm. ich habe irgentwo ein sch{\"o}nes Zitat zu dem Problem, ich find es blo{\ss} gerade nicht])

    \ldots

    - Fragestellung ob der Staat eine Rolle einnimmt? -> definitiv nicht Aktiv,
    [aber hei{\ss}t passiv nicht, dass er sich schon einmischt, blo{\ss} so, dass es keiner Mitbekommt?]
    - Die Mittel zur Steuerung sind ganz klar sich nicht einzumischen, laissez faire,
    [die Unsichtbare Hand wird es schon richten]


    - Sind diese Mittel kontr{\"a}r zur derzeitigen Geldpolitik?
    heutige Situation: staatliches Eingreifen notwendig, weil sonst Unterversorgung bestimmter
    G{\"u}ter durch privat Angebot. Au{\ss}erdem: Staat hat Sorge zu tragen, f{\"u}r jedes Mitglied der
    Gesellschaft keine Nachteile durch marktwirtschaftliche Vorg{\"a}nge entstehen zu lassen -> daraus
    w{\"u}rden sonst Spannungen zwischen den sozialen Schichten entstehen, daraus folgt ein
    Auseinanderbrechen des sozial Systems. "`\ldots seit Entwicklung der Keynesianischen VWL besteht,
    \ldots, {\"U}bereinstimmung hinsichtlich der Auffassung, dass zum Zwecke einer Nivellierung der
    konjunkturellen Schwankungen eine aktive staatliche Wirtschaftspolitik nicht nur f{\"o}rderlich,
     sondern auch erforderliche ist."'

    \ldots

    (Dabei vernachl{\"a}ssigen die Klassiker bewusst eine Wertaufbewahrungsfuktion des Geldes, weil es
    der {\"o}konomischen Rationalit{\"a}t wiederspricht)[ -> da kommt dann schon Keynes ins Spiel, deswegen
    den Absatz ganz zum Schluss]


\subsection{Beschreibung des eigenen Verst{\"a}ndnis von Geldpolitik}

    Nach Basseler hat die Geldpolitik die Hauptaufgabe, eine optimale Geldversorgung der Wirtschaft zu gew{\"a}hrleisten \citep[Vgl.][S. 551]{Basseler2010}.
    Real wird diese Aufgabe von einer gr{\"o}{"ss}tenteils staatlichen, aber unabh{\"a}ngigen Zentralbank {\"u}bernommen. Dabei herrscht weit verbreitet Konsens dar{\"u}ber, dass Geldpolitik staatliche Aufgabe bleibt, auch wenn Ideen einer dezentralen, dem Wettbewerb unterliegenden Geldversorgung durch private Gesch{\"a}ftsbanken und ein System konkurierender Parallelw{\"a}hrungen kursieren. (Hayek)
    "`Zentrale Zielgr{\"o}{"ss}e der Geldpolitik ist die Geldmenge M3. (...) Dabei ist zu beachten, dass das herk{\"o}mmliche Konzept von Banken zum Konzept der Mont{\"a}ren Finanzinstitute (MFIs) erweitert worden ist."`
    \citep[vgl.][S. 507]{Basseler2010}

    \citep[vgl.][S.508]{Basseler2010} "`MFIs sind also im Wesentlichen:
    - Zentralbanken,
    - Kreditinstitute und
    - Geldmarktfonds.
    (...) Innerhalb der Geldmenge M3 spielen Bargeldumlauf und t{\"a}glich f{\"a}llige Einlagen die gr{\"o}{"ss}te Rolle; Einlagen mit vereinbarterter K{\"u}ndigungsfrist (...) sind ebenfalls quantitativ bedeutsam; die {\"u}brigen Komponenten machen insgesamt nur knapp 20 Prozent der Geldmenge M3 aus."`


    \subsubsection{ Akteure des Finanzbereiches}

    \citep[vgl.][S.511f]{Basseler2010} "`Die Akteure im Finanzbereich werden allgemein Finanzintermedi{\"a}re genannt. [Diese] vermitteln Finanzprodukte zwischen den Anbietern und Nachfragern. (...) Dies sind vor allem Banken und Kapitalanlagegesellschaften, die selbst Finanzprodukte kreieren sowie institutionelle Anleger."`

     "`Es muss im Finanzbereich eine staatliche Institition geben, eine staatlich organisierte Zentralbank, die folgende Aufgaben erf{\"u}llt:
     \begin{itemize}
    \item{die Ausgabe der gesetzlichen Zahlungsmittel (Staatliches Emissionsmonopol),}
    \item{die Durchf{\"u}hrung einer Geldpolitik mit dem Ziel einer angemessenen Begrenzung der Geldmenge,}
    \item{die Organisation eines reibungslosen Zahlungs- und Kreditverkehrs als >>Bank der Banken<<,}
    \item{die Wahrung der Geldwertstabilit{\"a}t,}
    \item{die Bereitstellung einer ausreichenden Menge an Geld in Krisenzeiten"'}
    \end{itemize}


    \citep[vgl.][S.512-13]{Basseler2010} "`Gesch{\"a}ftsbanken (oder auch Kreditinstitute) sind die zentralen Akteure im Finanzbereich einer Volkswirtschaft. Die erste zentrale Funktion von Gesch{\"a}ftsbanken (kurz: Banken) ist die Abwicklung des **Zahlungsverkehrs** einer Volkswirtschaft. (...) Die zweite zentrale Funktion von Banken ist die Organisation und Durchf{\"u}hrung des **Kreditverkehrs** einer Volkswirtschaft. "`
    "`(...)die Organisation des Kreditverkehrs ist ein klassisches Gesch{\"a}ft der banken: Sie beschaffen Geld, sie verleihen Geld und sie versuchen, dieses Geld mit Gewinn wieder zur{\"u}ckzubekommen.(...)Die Kreditgew{\"a}hrung war neben der Abwicklung des Zahlungsverkehrs die klassische Aufgabe der Gesch{\"a}ftsbank."`
     "`Daneben gibt es weitere Aufgaben der Banken, vor allem die Verm{\"o}gensverwaltung der Kunden, die Ausgabe und den Handel mit Wertpapieren, die Beratung und Unterst{\"u}tzung bei Unternehmenszusammenschl{\"u}ssen oder die Unterst{\"u}tzung von Unternehmen bei ihrer Kapitalaufnahme, etwa bei B{\"o}rseng{\"a}ngen. Dies wird zusammenfassend **Investmentbanking** bezeichnet. (...) In diesem Segment des Bankengesch{\"a}fts  werden auf Zertifikate entwickelt und verkauft oder Fonds aufgelegt und Fondsanteile verkauft. (...)In Kontinentaleuropa ist (...)das Universalbankensystem etabliert(...). Sparkassen {\"u}bernehmen (...)die Abwicklung des Zahlungsverkehrs und des >>kleinen<< Kreditverkehrs, kleine Privatbanken {\"u}bernehmen eher die Funktionen des Investmentbankings und gro{"ss}e Universalbanken {\"u}bernehmen alle Gesch{\"a}ftssparten."`

    \citep[vgl.][S.515]{Basseler2010} "`Das Eigenkapital [einer Bank] setzt sich konkret zusammen aus dem Grundkapital, den Kapitalr{\"u}cklagen und den Gewinnr{\"u}cklagen (einbehaltene Gewinne) sowie einer stillen Einlage des Finanzmarktstabilisierungsfonds (...). Grunds{\"a}tzlich ist das Eigenkapital der Banken von zentraler Bedeutung. Es ist letztlich das Kapital, das die Bank zum Ausgleich von Verlusten aus ihrem Kredit- und Investmentgesch{\"a}ft einsetzen kann.  Daher sind in der Bankenaufsicht bestimmte Mindestanforderungen an die H{\"o}he des haftenden Eigenkapitals (...)vorgesehen"`

    \citep[vgl.][S.512]{Basseler2010}  "`Mit dem (...) 01.01.1999 ist (...)die Europ{\"a}ische Zentralbank (EZB) die zentrale Institution - also die Zentralbank - f{\"u}r die Festlegung und Ausf{\"u}hrung der Geldpolik. Daneben existiert weiterhin die Deutsche Bundesbank, die als Zentralbank der Bundesrepublik Deutschland Teil des Europ{\"a}ischen Systems der Zentralbanken ist. Sie ist (...) ausf{\"u}hrendes organ der geldpolitischen Entscheidungen der (...)EZB. "`
    \clearpage

\section{Technisches System}
	\label{sec2:technischesSystem}




\subsection{Organisationen der Europ{\"a}ischen Geldpolitk}

Die Europ{\"a}ische Geldpolitik wird durch das Europ{\"a}ische System der Zentralbanken (ESZB) und die EZB organisiert, wobei sich die ESZB aus der Europ{\"a}ischen Zentralbank und allen 27 nationalen Zentralbanken (NZBen) der Mitgliedsstaaten der EU organisiert.  Damit sind Geldarten und Geldsch{\"o}pfung bei einer supranationalen Institution monopolisiert. Sonderstatus haben dabei die sogenannten "`Outs"', jene Mitgliederstaaten der EU, die den Euro  nicht eingef{\"u}hrt haben. Dies sind Derzeit: D{\"a}nemark, Gro{"ss}britanien, Schweden sowie die meisten neuen EU-Mitgliedsstaaten nach 2001. Sie sind vom Entscheidungsprozess der ESZB ausgeschlossen und vollziehen eine eigenst{\"a}ndige nationale Geldpolitik.
-Auch wenn der EG-Vertrag\footnote[25]{EGV, Art. 105-109 d} formal zwischen EZB und ESZB unterscheidet, entscheidet doch faktisch nur eine Institution, die der EZB mit ihren Beschlussorganen (EZB-Rat und Direktorium der EZB.) \citep[vgl.][S.553]{Basseler2010})

\subsection{Die Europ{\"a}ische Zentralbank}
Die Beschlussorgane der Europ{\"a}ischen Zentralbank sind der EZB-Rat und das Direktorium der EZB, welche gemeinsam die EZB leiten. Das Direktorium der EZB besteht aus dem Pr{\"a}sidenten und dem Vizepr{\"a}sidenten der EZB, sowie weiteren vier Mitgliedern, die von den Regierungen der Mitgliedsstaaten auf der Ebene der Staats- und Regierungschefs auf Empfehlung des EU-Rats einvernehmlich ernennt. Ein Anh{\"o}rungsrecht besteht dabei beim Europ{\"a}ischem Parlament und beim EZB-Rat.
Der EZB-Rat wiederum besteht aus dem Direktorium und den Pr{\"a}sidenten aller nationalen Zentralbanken, die den Euro gemeinsam eingef{\"u}hrt haben.
Die exekutive Gewalt liegt innerhalb der Europ{\"a}ischen Zentralbank beim Direktorium, das die f{\"u}r die Durchf{\"u}hrung der Geldpolitik nach den Leitlinien und Beschl{\"u}ssen des EZB-Rates verantwortlich ist. Auch ist das Direktorium der EZB weisungsbefugt gegen{\"u}ber den nationalen Zentralbanken des Eurosystems.  Damit haben wir ein duales System, bestehend aus dem Exekutivorgan der EZB in Form des Direktoriums und ein ein Beschlussorgan in Form des EZB-Rates.
Der EZB-Rat erarbeitet und erl{\"a}sst die Beschl{\"u}sse einer gemeinschaftlichen europ{\"a}ischen Geldpolitik des Euro-Raumes, um die Ausgabe von M{\"u}nzen und Banknoten zu regeln und die Erf{\"u}llung von dem ESZB {\"u}bertragenen Aufgaben zu erf{\"u}llen.  ( \citep[vgl.][S.553]{Basseler2010} ) Derzeit umfasst der EZB-Rat 18 L{\"a}nder und das Direktorium, abgestimmt wird mit einfacher Mehrheit der Anwesenden, bei Stimmengleichheit entscheidet die Stimme des Pr{\"a}sidenten.

\subsection{Ziele und Aufgaben von ESZB und EZB}
Die EG-Vertr{\"a}ge definieren als vorangiges Ziel des ESZB und damit der EZB "`die Gew{\"a}hrleistung der Preisstabilit{\"a}t."'
"`Soweit dies ohne Beeintr{\"a}chtigung des Zieles der Preisstabilit{\"a}t m{\"o}glich ist, unterst{\"u}tz das ESZB die allgemeine Wirtschaftpolitik der Gemeinschaft, um die Verwirklichung der in Artikel 2 festgelegten Ziele der Gemeinsachft beizutragen. Das ESZB handelt im Einklang mit dem Grundsatz einer offenen Marktswirtschaft mit freien Wettbewerb \ldots"'\citep[vgl.][S.554]{Basseler2010}
Hier wird von der EZB eine Priorisierung der Preisstabilit{\"a}t gegen{\"u}ber anderen Zielen wie Vollbesch{\"a}ftigung und Wachstum festgeschrieben - diese weiteren Ziele werden der Preisstabilit{\"a}t untergeordnet. Verglichen mit der Zielvorschrift der Deutschen Bundesbank, entspricht diese Formulierung weitestgehend der damals geltenden Zielvorschrift.\citep[vgl.][S.554]{Basseler2010}

Die ideologische Grundlage f{\"u}r diese Priorisierung ist die von monetaristischen Str{\"o}mungen ausgehende Vorstellung, dass eine Zentralbank zu aller erst die Verantwortung f{\"u}r eine Preisstabilit{\"a}t besitzt, da andere Akteure f{\"u}r die Vollbesch{\"a}ftigung zust{\"a}ndig sind (z.B. Gewerkschaften, Tarifparteien, etc.) und der Wachstum sich aus dem technischen Fortschritt und dem Bev{\"o}lkerungswachstum ergibt.

Die haupts{\"a}chlichen Aufgaben des ESZB werden im Art. 105, Abs. 2 EGV wie folgt festgelegt:
\begin{itemize}
    \item{die Geldpolitik der Gemeinschaft festzulegen und auszuf{\"u}hren, Divisengesch{\"a}fte im Einklang mit Artikel 111 durchzuf{\"u}hren,}
    \item{die offiziellen W{\"a}hrungsreserven der Mitgliedstaaten zu halten und zu verwalten,}
    \item{das reibungslose Funktionieren des Zahlungssystems zu f{\"o}rdern}
\end{itemize} \citep[vgl.][S.555]{Basseler2010} "

Bevor wir weiter auf die Geldpolitik und die Instrumente der EZB eingehen, sei angemerkt, was jener Artikel 111 EGV, welcher die Devisengesch{\"a}fte der EZB regelt beinhaltet:
So legt dieser Artikel fest, das alle Entscheidungen {\"u}ber die Wechselkurssysteme, ob nun flexible oder feste Wechselkurse, oder ihre H{\"o}he bei Festlegung fester Wechselkurse dem Ministerrat vorbehalten sind. \citep[vgl.][S.555]{Basseler2010}




\subsection{Instrumente der Europ{\"a}ischen Geldpolitik}

Im Folgenden soll es um die Instrumente der Europ{\"a}ischen Geldpolitik durch die EZB gehen, als da w{\"a}ren:
\begin{enumerate}
  \item{Die Offenmarktpolitik}
  \item{Die Politik der St{\"a}ndigen Fazilit{\"a}ten}
  \item{Die Mindestreservepolitik.}
  \item{Weitere Instrumente}
\end{enumerate}

\subsubsection{Die Offenmarktpolitik}
Nach g{\"a}ngiger Politik- und Wirtschaftstheorie bedarf es in einer wachsenden Volkswirtschaft, um eine hinreichende Geldversorgung der Wirtschaft zu gew{\"a}hrleisten, einer fortw{\"a}hrenden Ausweitung der nominalen Geldmenge. Als zentrales Instrument der Geldpolitik kommt hier die Offenmarktpolitik der Europ{\"a}ischen Zentralbank zum Tragen.
Vereinfacht gesprochen, wird unter der Offenmarktpolitk nichts anderes verstanden, als der An- und Verkauft von Wertpapieren gegen Zentralbankgeld durch die europ{\"a}ische Zentralbank. Bezweckt wird mit den Offenmarktk{\"a}ufen, bzw. -verk{\"a}ufen eine Zentralbankgeldsch{\"o}pfung bzw. -vernichtung zur Ver{\"a}nderung der nominalen Geldmenge. Allerdings ist der Begriff "`Offen"'-Marktgesch{\"a}ft irref{\"u}hrend: In der ESZB sind als Gesch{\"a}ftspartner der EZB nur finanziell solide MFIs zugelassen, die in das Mindestreservesystem einbezogen sind.

Bei Offenmarktgesch{\"a}ften erh{\"o}ht sich der Bestand an zentralbankgeld der Gesch{\"a}ftsbanken beim Kauf von Wertpapieren, doch beim Verkauft von Wertpapieren sinkt der Bestand an Zentralbankgeld der Gesch{\"a}ftsbanken. Der Kauf von Wertpapieren durch die Zentralbank wird mitunter auch expansive Offenmarktpolitik genannt.
Wichtig ist zu wissen, dass die Europ{\"a}ische Zentralbank die Gesch{\"a}ftsbanken nicht zum Kauf von Wertpapieren zwingen kann und somit attraktive Konditionen bieten muss: So sinkt bei einer geplanten expansiven Offenmarktpolitik der Zinssatz f{\"u}r die Zentralbankgeld-Kreditgew{\"a}hrung unter den {\"u}blichen Geldmarktzins. Entsprechend umgekehrt ist es bei der {\"a}u{"ss}erst selten vorkommenden, sogenannten kontraktiven Offenmarktpolitik. Hier m{\"u}ssen die von der Zentralbank angebitenen Zinss{\"a}tze h{\"o}her sein, als die sonst {\"u}blichen Geldmarktzinsen.

Offenmarktgesch{\"a}fte treten h{\"a}ufig in der Form von Repo-Gesch{\"a}ften auf. Dabei handelt es sich nach der englischen Bezeichnung Repurchase um R{\"u}ckkauf-Gesch{\"a}fte, d.h. es wird mit den Gesch{\"a}ftsbanken eine Laufzeit der Wertpapiere vereinbart, an deren Ende diese ihre Wertpapiere zur{\"u}ckkaufen m{\"u}ssen. Hiermit entsteht ein automatischer R{\"u}ckfluss von Zentralbankgeld. Diese auf Frist gesetzten Repo-Gesch{\"a}fte werden auch Wertpapierpensionsgesch{\"a}fte genannt, der entsprechend vorkommende Zinssatz wird demgem{\"a}{"ss} als Pensionssatz bezeichnet.
Die Repo-Gesch{\"a}fte sind aus Sicht der g{\"a}ngigen Wirtschaftstheorie ein gut steuerbares Instrument f{\"u}r die Entwicklung der Zentralbankgeldmenge im Gesch{\"a}ftsbankensektor, da sich auch kontraktive Effekte einstellen, wenn keine neuen Repo-Gesch{\"a}fte abgeschlossen werden. Eine r{\"u}ckl{\"a}ufige Anzahl an Repo-Gesch{\"a}ften f{\"u}hrt somit zu einer Abnahme der Zentralbankgeldmenge im Sektor der Gesch{\"a}ftsbanken.











\clearpage

\section{Einblick in die Grundgedanken der {\"O}konomen des 19. und 20. Jahrhunderts}
  \label{sec3:stroemungen}
  TEXT TEXT TEXT


\subsection{{\"U}bersicht der zu behandelnden Str{\"o}mungen und Begr{\"u}ndung der Auswahl}

\subsection{Einblick in die Grundgedanken der Keynsianischen Schule}

Lord John Maynard Keynes\footnote[14]{Geboren:  5. Juni 1883 in Cambridge; Gestorben: 21. April 1946 in Tilton, im weiteren Keynes}


\subsection{Einblick in die Grundgedanken der {\"O}stereichischen Schule (Hayek)}


Friedrich August von Hayek\footnote[15]{Geboren: 8. Mai 1899 in Wien; Gestorben: 23. M�rz 1992 in Freiburg im Breisgau, im weiteren Hayek}

Jedoch macht Hayek das zugest{\"a}ndnis, dass es \citep[vgl.][S.23]{Hayek1977}: "``Einer Regierung muss es nat{\"u}rlich freistehen, dar{\"u}ber zu entscheiden, in welchem Zahlungsmittel die Steuern zu entrichten sind, und sie muss Vertr{\"a}ge in jedem beliebigen Zahlungsmittel abschlie{\ss}en k{\"o}nnen (wodurch sie ein von ihr ausgegebenes Zahlungsmittel beg{\"u}nstigen kann)."`



\citep[vgl.][S.40f]{Hayek1977}: "`Wenn wir von unterschiedlichen Geldarten sprechen, denken wir an unterschiedlich bezeichnete Einheiten, die in ihrem relativen Wert zueinander schwanken k{\"o}nnen. (\ldots)Ich hielt es immer f{\"u}r n{\"u}tzlich, (\ldots) dass es f{\"u}r die Erkl{\"a}rung monet{\"a}rer Ph{\"a}nomene viel hilfreicher w{\"a}re, wenn ?��Geld?�� als Adjektiv eine Eigenschaft beschriebe, die unterschiedliche Dinge (\ldots)besitzen k{\"o}nnen. `Umlaufmittel' ist aus diesem Grund passender, da Dinge in unterschiedlichem ma{\ss} in verschiedenen Regionen oder Bev{\"o}lkerungsgruppen `in Umlauf sein' k{\"o}nnen."`

\citep[vgl.][S.43]{Hayek1977}: "`Zitat fehlt"

F{\"u}r die W{\"a}hrung sieht Hayek ein Marktmodell vor, wobei \citep[vgl.][S.31]{Hayek1977}: "`Der Verkauf (am Schalter oder durch Versteigerung) w{\"a}re anf{\"a}nglich die wichtigste Emissionsform der neuen W{\"a}hrung. Nachdem sich jedoch ein regul{\"a}rer Markt herausgebildet h{\"a}tte, w{\"u}rde sie normalerweise im Wege des {\"u}blichen Bankgesch{\"a}fts, d.h. durch kurzfristige Kreditvergabe in Umlauf gebracht."`

Die Wertstabilit{\"a}t dieser W{\"a}hrung kommt f{\"u}r Hayek daher: \citep[vgl.][S.32]{Hayek1977}: "`Wettbewerb w{\"u}rde sicherlich die emittierenden Institutionen weit wirksamer dazu zwingen, den Wert ihres Geldes (in Bezug auf ein festgesetztes G{\"u}terb{\"u}ndel) konstant zu halten, als es irgendeine Verpflichtung zur Einl{\"o}sung des Geldes in diese G{\"u}ter (oder in Gold) k{\"o}nnte."`

Jedoch ist es die Akzeptierbarkeit, die es zu Geld macht \citep[vgl.][S.40]{Hayek1977}: Es ist der "`Grad ihrer Akzeptierbarkeit (oder Liquidit{\"a}t, d.h. in der Eigenschaft, die sie zu Geld macht)"

Zur genaueren Beschreibung seines Systems, soll Hayek selbst zu Wort kommen: \citep[vgl.][S.45]{Hayek1977}: "`Der emittierenden Bank werden zwei Methoden zur {\"a}nderung des Volumens ihrer zirkulierenden Umlaufmittel zur Verf{\"u}gung stehen: Sie kann ihr Umlaufmittel gegen andere (oder gegen Wertpapiere und m{\"o}glicherweise einige Waren) verkaufen oder kaufen; und sie kann ihre Kreditgew{\"a}hgungst{\"a}tigkeit einschr{\"a}nken oder ausdehnen. Um die ausstehende Menge ihres Umlaufmittels unter Kontrolle zu halten, wird sie sich im ganzen auf die Einr{\"a}umung relativ kurzfristiger Kredite beschr{\"a}nken, so dass bei Reduktion oder zeitweisem Einstellen neuer Kreditvergabe die laufenden R{\"u}ckzahlungen ausstehender Forderungen eine rasche Verminderung ihres gesamten Geldumlaufes mit sich bringen w{\"u}rden."`

\subsubsection{Kommentar zu Friedman}

\subsubsection{Auseinandersetzung mit Keynes}

















\subsection{Einblick in die Grundgedanken der Monetaristen (Milton Friedman)}
Milton Friedman \footnote[16]{Geboren: 31. Juli 1912 in Brooklyn, New York City; Gestorben: 16. November 2006 in San Francisco, im Weiteren Friedman}

\clearpage
\ac{ESZB}
\ac{EZB}



\section{Case Study}
\label{sec4:CaseStudy}
Ein Instrument herausgreifen, bzw. eine Instrumentendiskussion f{\"u}hren.





\clearpage

\section{Abk{\"u}rzungsverzeichnis}
	\label{sec5:Abkuerzungsverzeichnis}

\begin{acronym}[ESZB]

 	\acro{EG}{Europ{\"a}ische Gemeinschaft}
	\acro{EGV}{Europ{\"a}ische Gemeinschaftsvertr{\"a}ge}
  	\acro{ESZB}{Europ{\"a}isches System der Zentralbanken}
  	\acro{EU}{Europ{\"a}ische Union}
	\acro{EZB}{Europ{\"a}ische Zentralbank}
 	\acrodefplural{EZB}{Europ{\"a}ische Zentralbanken}


\end{acronym}

%%%% LITERATURE %%%%%
\vspace{10pt}
	\newpage
  	\singlespacing

% Literaturliste endgueltig anzeigen
	         %\bibliographystyle{diss_fk}

\bibliographystyle{authordate1}
\section{Literaturverzeichnis}
	\label{sec6:Literaturverzeichnis}
	\bibliography{literatur_EWA}	% Sie benoetigen eine *.bib-Datei
	\newpage





%%%% APPENDIX %%%%%
	%\section*{Anhang}
	%\label{sec:anhang}

\end{document}
