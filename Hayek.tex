\documentclass[
    onecolumn,
    a4paper,
    abstracton,
    parskip=half
    %,draft
    ,final
    ]{scrartcl}

    \usepackage[pdftex
    ,draft
    ]{graphicx}

    \usepackage{booktabs}

    \usepackage[cmex10]{amsmath}

    \interdisplaylinepenalty=2500
    \usepackage{url}

    \usepackage[breaklinks]{hyperref}
    \hyphenation{nothing} % correct bad hyphenation here

    \usepackage{eurosym}

    \usepackage{listings}
    \lstset{basicstyle=\small\ttfamily,breaklines=true}
    \emergencystretch 1000pt

    \usepackage{subcaption}

    \usepackage{mathtools}

    % deutsche Silbentrennung
    \usepackage[ngerman]{babel}

    \usepackage[printonlyused, withpage]{acronym}

    \usepackage[a4paper]{geometry}

    % wegen deutschen Umlauten
    \usepackage[ansinew]{inputenc}

    % fuer Zitate
    \usepackage[round]{natbib}

    \usepackage{setspace}

    \usepackage{units}
    \usepackage{cite}
    %%% Deutsche Verzeichnis-ueberschriften



    %%% Kommentarfunktion %%%
    \usepackage[textwidth=2.2cm
    ,obeyFinal
    ]{todonotes}

\begin{document}

 Der Teil mit Hayek

Jedoch macht Hayek das zugest{\"a}ndnis, dass es \citep[vgl.][S.23]{Hayek1977}: "`Einer Regierung muss es nat{\"u}rlich freistehen, dar{\"u}ber zu entscheiden, in welchem Zahlungsmittel die Steuern zu entrichten sind, und sie muss Vertr{\"a}ge in jedem beliebigen Zahlungsmittel abschlie{\ss}en k{\"o}nnen (wodurch sie ein von ihr ausgegebenes Zahlungsmittel beg{\"u}nstigen kann)."`



\citep[vgl.][S.40f]{Hayek1977}: "`Wenn wir von unterschiedlichen Geldarten sprechen, denken wir an unterschiedlich bezeichnete Einheiten, die in ihrem relativen Wert zueinander schwanken k{\"o}nnen. (...)Ich hielt es immer f{\"u}r n{\"u}tzlich, (...) dass es f{\"u}r die Erkl{\"a}rung monet{\"a}rer Ph{\"a}nomene viel hilfreicher w{\"a}re, wenn "`Geld"' als Adjektiv eine Eigenschaft beschriebe, die unterschiedliche Dinge (...)besitzen k{\"o}nnen. "`Umlaufmittel"'ist aus diesem Grund passender, da Dinge in unterschiedlichem ma{\ss} in verschiedenen Regionen oder Bev{\"o}lkerungsgruppen "`in Umlauf sein"' k{\"o}nnen."`

\citep[vgl.][S.43]{Hayek1977}: "`Wir werden "`Umlaufmittel"' au{\ss}erdem, vielleicht etwas im Widerspruch zur urspr{\"u}nglichen Bedeutung des Begriffes, in dem Sinn verwenden, dass nicht nur Papier und andere Sorten eines von "`Hand-zu-Hand-gehenden Geldes"' eingeschlossen sind, sondern auch Scheckkonten und andere Tauschmittel, die f{\"u}r die meisten Zwecke genutzt werden k{\"o}nnen, f{\"u}r die auch Schecks in Frage kommen."`

F{\"u}r die W{\"a}hrung sieht Hayek ein Marktmodell vor, wobei \citep[vgl.][S.31]{Hayek1977}: "`Der Verkauf (am Schalter oder durch Versteigerung) w{\"a}re anf{\"a}nglich die wichtigste Emissionsform der neuen W{\"a}hrung. Nachdem sich jedoch ein regul{\"a}rer Markt herausgebildet h{\"a}tte, w{\"u}rde sie normalerweise im Wege des {\"u}blichen Bankgesch{\"a}fts, d.h. durch kurzfristige Kreditvergabe in Umlauf gebracht."`

Die Wertstabilit{\"a}t dieser W{\"a}hrung kommt f{\"u}r Hayek daher: \citep[vgl.][S.32]{Hayek1977}: "`Wettbewerb w{\"u}rde sicherlich die emittierenden Institutionen weit wirksamer dazu zwingen, den Wert ihres Geldes (in Bezug auf ein festgesetztes G{\"u}terb{\"u}ndel) konstant zu halten, als es irgendeine Verpflichtung zur Einl{\"o}sung des Geldes in diese G{\"u}ter (oder in Gold) k{\"o}nnte."`

Jedoch ist es die Akzeptierbarkeit, die es zu Geld macht \citep[vgl.][S.40]{Hayek1977}: Es ist der "`Grad ihrer Akzeptierbarkeit (oder Liquidit{\"a}t, d.h. in der Eigenschaft, die sie zu Geld macht)"`

Zur genaueren Beschreibung seines Systems, soll Hayek selbst zu Wort kommen: \citep[vgl.][S.45]{Hayek1977}: "`Der emittierenden Bank werden zwei Methoden zur {\"a}nderung des Volumens ihrer zirkulierenden Umlaufmittel zur Verf{\"u}gung stehen: Sie kann ihr Umlaufmittel gegen andere (oder gegen Wertpapiere und m{\"o}glicherweise einige Waren) verkaufen oder kaufen; und sie kann ihre Kreditgew{\"a}hgungst{\"a}tigkeit einschr{\"a}nken oder ausdehnen. Um die ausstehende Menge ihres Umlaufmittels unter Kontrolle zu halten, wird sie sich im ganzen auf die Einr{\"a}umung relativ kurzfristiger Kredite beschr{\"a}nken, so dass bei Reduktion oder zeitweisem Einstellen neuer Kreditvergabe die laufenden R{\"u}ckzahlungen ausstehender Forderungen eine rasche Verminderung ihres gesamten Geldumlaufes mit sich bringen w{\"u}rden."`


Kommentar zu Friedman

Auseinandersetzung mit Keynes
Gerade eine Kritik an Keynes "`General Theorie"'  zu schreiben, wollte Hayek nicht mehr machen, denn \citep[vgl.][S.91]{Hayek1969}: "`Obwohl er [Keynes] das Werk eine "`allgemeine"' Theorie genannt hatte, war sie f{\"u}r mich zu offensichtlich wieder nur eine zeitbedingte Abhandlung, zugeschnitten auf die augenblicklichen politischen Notwendigkeiten, wie er sie sah."`

\citep[vgl.][S.93]{Hayek1969}: "`Es ist leicht zu sehen, wie die Anschauung, nach der eine zus{\"a}tzliche Geldsch{\"o}pfung zur Erzeugung einer entsprechenden G{\"u}termenge f{\"u}hren wird, zu einem Wiederaufleben der eher naiven inflationistischen Trugschl{\"u}sse f{\"u}hren musste(...). Ich habe wenig Zweifel, dass wir die Nachkriegsinflation zum Gro{\ss}teil dem starken Einfluss eines solchen {\"u}bervereinfachten Keynsianismus verdanken."`

und weiter kommentiert Hayek die "`General Theorie"' wie folgt:
\citep[vgl.][S.96]{Hayek1969}: "`Ich wage vorauszusagen, dass wenn diese Frage der Methode einmal entschieden ist, die "`Keynessche Revolution"' als eine Episode erscheinen wird, in der irrt{\"u}mliche Auffassungen {\"u}ber die geeignete wissenschaftliche Mehtode zu einem zeitweiligen Vergessen vieler wichtiger Einsichten f{\"u}hrten, die wir schon gewonnen hatten und die wir dann m{\"u}hevoll wiedergewinnen m{\"u}ssen."`



\end{document}
