\documentclass[
      onecolumn,
      a4paper,
      abstracton,
      parskip=half
      %,draft
      ,final
      ]{scrartcl}

      \usepackage[pdftex
      ,draft
      ]{graphicx}

      \usepackage{booktabs}

      \usepackage[cmex10]{amsmath}

      \interdisplaylinepenalty=2500
      \usepackage{url}

      \usepackage[breaklinks]{hyperref}
      \hyphenation{nothing} % correct bad hyphenation here

      \usepackage{eurosym}

      \usepackage{listings}
      \lstset{basicstyle=\small\ttfamily,breaklines=true}
      \emergencystretch 1000pt

      \usepackage{subcaption}

      \usepackage{mathtools}

      % deutsche Silbentrennung
      \usepackage[ngerman]{babel}

      \usepackage[printonlyused, withpage]{acronym}

      \usepackage[a4paper]{geometry}

      % wegen deutschen Umlauten
      \usepackage[ansinew]{inputenc}

      % fuer Zitate
      \usepackage[round]{natbib}

      \usepackage{setspace}

      \usepackage{units}
      \usepackage{cite}
      %%% Deutsche Verzeichnis-ueberschriften



      %%% Kommentarfunktion %%%
      \usepackage[textwidth=2.2cm
      ,obeyFinal
      ]{todonotes}
\begin{document}

"`Man muss die Geschichte der Meinungen studieren, ehe man den eigenen Geist befreien kann."'[1]

In die Zeit der Schaffensphase von Keynes f{\"a}llt die Weltwirtschaftskrise von 1929. Wenn man
der Frage nachgeht, wie es zu dieser Weltumfassenden Depression kommen konnte, ist die Antwort
daf{\"u}r plausibel nach derer eine unzureichende strukturelle Anpassung nach dem  1. Weltkrieg
die Ursache daf{\"u}r sein kann. Beispielsweise vergr{\"o}{"s}erte die Landwirtschaft w{\"a}hrend des
1. Weltkrieges seine Kapazit{\"a}ten nach dem Abbruch der internationalen Handelsbeziehungen und
litt nach Beendigung des Krieges an {\"u}berproduktion die sie nicht abbaute, so fielen in den
Jahren vor der Kriese die Preise f{\"u}r Anbauprodukte ins Bodenlose.[2] Die Nachfrage wurde dem
Angebot nicht mehr Gerecht.[3] Weiterer Preis- und Lohnverfall waren die Folge und damit ein
Anstieg der Arbeitslosigkeit. Hinzu kam eine Staats{\"u}bergreifende Deflationspolitik nach dem
Ausbruch der Krise. Fehler die gemacht wurden, die eindeutig den falschen Annahmen der
Klassiker zuzuschreiben sind, demnach die Selbstheilungskr{\"a}fte der Wirtschaft am
Wikungsvollsten sind wenn der Staat sich nicht einmischt.[4]
Mit diesen, in die Irre f{\"u}hrenden, Ansichten wollte Keynes und die die seine Gedanken zu
diesem Thema teilten ein Ende bereiten. Die Denkrichtung des Keynesianismus verurteilte
die sich selbst regulierenden Kr{\"a}fte des Kapitalismus nicht als Unsinn, vielmehr gaben
sie zu bedenken, dass ein Engreifen des Staates in bestimmten Situationen f{\"o}rderlich ja sogar
erforderlich ist.[5] Keynes sah seine Einw{\"a}nde selbst als Evolution zur vorherrschenden klassischen
Meinung[6] doch genauer betrachtet war es doch eine revolution{\"a}re Sichtweise.
Den Klassikern war es nicht m{\"o}glich die Wege aufzuzeigen, bzw. zu Erkl{\"a}ren, wie sich die Wirtschaft
von der Krise erholen k{\"o}nne[7] - Keynes hingegen zeigte Anhand neuer Werkzeuge[8] dass die
Deflationspolitik kontraproduktiv ist, sogar noch verst{\"a}rkend in die falsche Richtung wirkt.
Ganz prek{\"a}r zeigte sich das Versagen des "`laissez-faire"' in Deutschland mit dem Sturz der Regierung
Br{\"u}ning. Dem Vorausgehend versuchten Reformer mit den Argumenten des Multiplikator-Effektes[9], das
Ankurbeln der Wirtschaft mit Hilfe von Arbeitsbeschaffungsprogrammen[10] in Gang zu setzen, sowie
aus der Sorge heraus, die Arbeitslosen, in ihrer Verzweiflung, k{\"o}nnten Zuflucht zu einer der
extremistischen politischen Parteien suchen.[11]
Man kann also zusammenfassen, dass in Keynes Wirtschaftspolitik die Besch{\"a}ftigung die prim{\"a}re
Zielgr{\"o}{"ss}e ist die angestrebt werden sollte und demnach die Fiskalpolitik eine starke Wirkung
hat, st{\"a}rker als die Geldpolitik, die nur indirekt wirke. [12] Das Wird auch nochmal in dem
Folgenden Zit{\"a}t von John Maynard Keynes deutlich:
"`Die Bedeutung des Geldes liegt allein in seiner Kaufkraft. Eine Ver{\"a}nderung in der M{\"u}nzeinheit(...),
hat (...)keine Auswirkungen."' [13]



[1] Keynes - Das Ende des laissez faire 2011 Hauptschrift I S.24
[2] Der Keynesianismus 1 - S.14 ff
[3] Der Keynesianismus 1 - S.55
[4] Keynes - Das Ende des laissez faire 2011 Hauptschrift II S.36
[5] Der Keynesianismus 1 - S.37
[6] National Self-Sufficiency http://panarchy.org/keynes/national.1933.html
  Absatz V Abgerufen 05.09.2014 15:15
[7] siehe 1.1 Benennung der Fragestellung - 2.Absatz (Def. Revolution)
[8] Gesamtwirtschaftliche Produktionsfunktion, Multiplikator, IS-LM-Modell
[9] %http://www.macroeconomics.tu-berlin.de/fileadmin/fg124/avwl2/vorlesung/06-Konsum_und_Guetermarkt-GG.pdf
  S.36 Abgerufen 05.09.2014 14:20
[10] Arbeitsbeschaffungsprogramme entsprechen dabei Kreditfinanzierte Staatsausgabenerh{\"o}hung
[11] Der Keynesianismus 1 - S.111
[12] Der Keynesianismus 1 - S.181
[13] J.M. Keynes - Ein Traktat {\"u}ber W{\"a}hrungsreform S.1

\end{document}
