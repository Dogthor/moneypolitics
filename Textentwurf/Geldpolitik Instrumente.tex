\documentclass[
    onecolumn,
    a4paper,
    abstracton,
    parskip=half
    %,draft
    ,final
    ]{scrartcl}

    \usepackage[pdftex
    ,draft
    ]{graphicx}

    \usepackage{booktabs}

    \usepackage[cmex10]{amsmath}

    \interdisplaylinepenalty=2500
    \usepackage{url}

    \usepackage[breaklinks]{hyperref}
    \hyphenation{nothing} % correct bad hyphenation here

    \usepackage{eurosym}

    \usepackage{listings}
    \lstset{basicstyle=\small\ttfamily,breaklines=true}
    \emergencystretch 1000pt

    \usepackage{subcaption}

    \usepackage{mathtools}

    % deutsche Silbentrennung
    \usepackage[ngerman]{babel}

    \usepackage[printonlyused, withpage]{acronym}

    \usepackage[a4paper]{geometry}

    % wegen deutschen Umlauten
    \usepackage[ansinew]{inputenc}

    % fuer Zitate
    \usepackage[round]{natbib}

    \usepackage{setspace}

    \usepackage{units}
    \usepackage{cite}
    %%% Deutsche Verzeichnis-ueberschriften



    %%% Kommentarfunktion %%%
    \usepackage[textwidth=2.2cm
    ,obeyFinal
    ]{todonotes}

\begin{document}

\subsection{Organisationen der Europ{\"a}ischen Geldpolitk}

Die Europ{\"a}ische Geldpolitik wird durch das Europ{\"a}ische System der Zentralbanken (ESZB) und die EZB organisiert, wobei sich die ESZB aus der Europ{\"a}ischen Zentralbank und allen 27 nationalen Zentralbanken (NZBen) der Mitgliedsstaaten der EU organisiert.  Damit sind Geldarten und Geldsch{\"o}pfung bei einer supranationalen Institution monopolisiert. Sonderstatus haben dabei die sogenannten "`Outs"', jene Mitgliederstaaten der EU, die den Euro  nicht eingef{\"u}hrt haben. Dies sind Derzeit: D{\"a}nemark, Gro{"s}britanien, Schweden sowie die meisten neuen EU-Mitgliedsstaaten nach 2001. Sie sind vom Entscheidungsprozess der ESZB ausgeschlossen und vollziehen eine eigenst{\"a}ndige nationale Geldpolitik.
-Auch wenn der EG-Vertrag\footnote[25]{EGV, Art. 105-109 d} formal zwischen EZB und ESZB unterscheidet, entscheidet doch faktisch nur eine Institution, die der EZB mit ihren Beschlussorganen (EZB-Rat und Direktorium der EZB.) \citep[vgl.][S.553]{Basseler2010})

\subsection{Die Europ{\"a}ische Zentralbank}
Die Beschlussorgane der Europ{\"a}ischen Zentralbank sind der EZB-Rat und das Direktorium der EZB, welche gemeinsam die EZB leiten. Das Direktorium der EZB besteht aus dem Pr{\"a}sidenten und dem Vizepr{\"a}sidenten der EZB, sowie weiteren vier Mitgliedern, die von den Regierungen der Mitgliedsstaaten auf der Ebene der Staats- und Regierungschefs auf Empfehlung des EU-Rats einvernehmlich ernennt. Ein Anh{\"o}rungsrecht besteht dabei beim Europ{\"a}ischem Parlament und beim EZB-Rat.
Der EZB-Rat wiederum besteht aus dem Direktorium und den Pr{\"a}sidenten aller nationalen Zentralbanken, die den Euro gemeinsam eingef{\"u}hrt haben.
Die exekutive Gewalt liegt innerhalb der Europ{\"a}ischen Zentralbank beim Direktorium, das die f{\"u}r die Durchf{\"u}hrung der Geldpolitik nach den Leitlinien und Beschl{\"u}ssen des EZB-Rates verantwortlich ist. Auch ist das Direktorium der EZB weisungsbefugt gegen{\"u}ber den nationalen Zentralbanken des Eurosystems.  Damit haben wir ein duales System, bestehend aus dem Exekutivorgan der EZB in Form des Direktoriums und ein ein Beschlussorgan in Form des EZB-Rates.
Der EZB-Rat erarbeitet und erl{\"a}sst die Beschl{\"u}sse einer gemeinschaftlichen europ{\"a}ischen Geldpolitik des Euro-Raumes, um die Ausgabe von M{\"u}nzen und Banknoten zu regeln und die Erf{\"u}llung von dem ESZB {\"u}bertragenen Aufgaben zu erf{\"u}llen.  ( \citep[vgl.][S.553]{Basseler2010} ) Derzeit umfasst der EZB-Rat 18 L{\"a}nder und das Direktorium, abgestimmt wird mit einfacher Mehrheit der Anwesenden, bei Stimmengleichheit entscheidet die Stimme des Pr{\"a}sidenten.

\subsection{Ziele und Aufgaben von ESZB und EZB}
Die EG-Vertr{\"a}ge definieren als vorangiges Ziel des ESZB und damit der EZB "`die Gew{\"a}hrleistung der Preisstabilit{\"a}t."'
"`Soweit dies ohne Beeintr{\"a}chtigung des Zieles der Preisstabilit{\"a}t m{\"o}glich ist, unterst{\"u}tz das ESZB die allgemeine Wirtschaftpolitik der Gemeinschaft, um die Verwirklichung der in Artikel 2 festgelegten Ziele der Gemeinsachft beizutragen. Das ESZB handelt im Einklang mit dem Grundsatz einer offenen Marktswirtschaft mit freien Wettbewerb \ldots"'\citep[vgl.][S.554]{Basseler2010}
Hier wird von der EZB eine Priorisierung der Preisstabilit{\"a}t gegen{\"u}ber anderen Zielen wie Vollbesch{\"a}ftigung und Wachstum festgeschrieben - diese weiteren Ziele werden der Preisstabilit{\"a}t untergeordnet. Verglichen mit der Zielvorschrift der Deutschen Bundesbank, entspricht diese Formulierung weitestgehend der damals geltenden Zielvorschrift.\citep[vgl.][S.554]{Basseler2010}

Die ideologische Grundlage f{\"u}r diese Priorisierung ist die von monetaristischen Str{\"o}mungen ausgehende Vorstellung, dass eine Zentralbank zu aller erst die Verantwortung f{\"u}r eine Preisstabilit{\"a}t besitzt, da andere Akteure f{\"u}r die Vollbesch{\"a}ftigung zust{\"a}ndig sind (z.B. Gewerkschaften, Tarifparteien, etc.) und der Wachstum sich aus dem technischen Fortschritt und dem Bev{\"o}lkerungswachstum ergibt.

Die haupts{\"a}chlichen Aufgaben des ESZB werden im Art. 105, Abs. 2 EGV wie folgt festgelegt:
\begin{itemize}
    \item{die Geldpolitik der Gemeinschaft festzulegen und auszuf{\"u}hren, Divisengesch{\"a}fte im Einklang mit Artikel 111 durchzuf{\"u}hren,}
    \item{die offiziellen W{\"a}hrungsreserven der Mitgliedstaaten zu halten und zu verwalten,}
    \item{das reibungslose Funktionieren des Zahlungssystems zu f{\"o}rdern}
\end{itemize} \citep[vgl.][S.555]{Basseler2010} "

\subsubsection{Unabh{\"a}ngigkeit der EZB}
\citep[vgl.][S.555-557]{Basseler2010}
Wenn es das vorrangige Ziel einer Zentralbank ist, Preisstabilit{\"a}t zu gew{\"a}hrleisten, dann gilt ihre Unabh{\"a}ngigkeit als Zentral. (...)

Zum Ersten ist die EZB funktional relativ unabh{\"a}ngig, weil sie Weisungen nicht entgegenhemen darf. \footnote[34]{vgl. Artikel 107, EGV (Unabh{\"a}ngigkeit der EZB)}

Eine solche Unabh{\"a}ngigkeit - keinerlei Kontrollen durch Regierungen und Parlamente unterworfen zu sein -  ist relativ einzigartig. Sie wird nur dadurch ein klein wenig beschr{\"a}nkt, dass die Verpflichtung besteht, die allgemeine Wirtschaftspolitik der Gemeinschaft zu unterst{\"u}tzen, aber nur, wenn dadurch das Ziel der Preisstabilit{\"a}t nicht beeintr{\"a}chtigt ist.

Zum Zweiten ist die EZB auch personell unabh{\"a}gig, einzig {\"u}ber die Ernennung der Pr{\"a}sidenten der Nationalbanken k{\"o}nnen einzelne Regierungen Einfluss aus{\"u}ben.

Zum Dritten ist die EZB auch finanziell unabh{\"a}ngig - sie verf{\"u}gt {\"u}ber eigene Einnahmen und einen eigenen Haushalt - und sie hat Kontrolle {\"u}ber die Instrumente der Geldpolitik.



Bevor wir weiter auf die Geldpolitik und die Instrumente der EZB eingehen, sei angemerkt, was jener Artikel 111 EGV, welcher die Devisengesch{\"a}fte der EZB regelt beinhaltet:
So legt dieser Artikel fest, das alle Entscheidungen {\"u}ber die Wechselkurssysteme, ob nun flexible oder feste Wechselkurse, oder ihre H{\"o}he bei Festlegung fester Wechselkurse dem Ministerrat vorbehalten sind. \citep[vgl.][S.555]{Basseler2010}




\subsection{Instrumente der Europ{\"a}ischen Geldpolitik}
\citep[vgl.][S.558f]{Basseler2010}

Offenmarktpolitik der EZB
Das Bankensystem im Eurogebiet ist auf die Bereitstellung von Zentralbankgeld durch die EZB angewiesen, weshalb die Offenmarktpolitik der EZB am Interbankengeldmarkt ansetzt. Dabei greift die EZB auf folgende Instrumente zur{\"u}ck:
\begin{enumerate}
  \item{Hauptrefinanzierungsinstrument}
  \item{l{\"a}ngerfristige Refinanzierungsgesch{\"a}fte}
  \item{Feinsteuerungsoption}
  \item{Strukturelle Operationen}
  \end{enumerate}

Das \textbf{Hauptrefinanzierungsinstrument} und das \textbf{l{\"a}ngerfristige Refinanzierungsgesch{\"a}ft} sind regelm{\"a}{"ss}ig stattfindende Liquidit{\"a}t zuf{\"u}hrende Repo-Gesch{\"a}fte, wobei f{\"u}r das Hauptrefinanzierungsinstrument w{\"o}chentliche, auf eine Woche begrenzte Transaktionen gelten, w{\"a}hrend die Refinanzierungsgesch{\"a}fte mit monatlichem Abstand und mehre Monate (bis zu einem Jahr) Laufzeit haben. Das Hauptrefinanzierungsinstrument steuert die Zinss{\"a}tze (Zentraler Leitzins der EZB) sowie die Liquidit{\"a}t am europ{\"a}ischen Geldmarkt.

Sollte es zu unerwarteten marktm{\"a}{"ss}igen Liquidit{\"a}tsschwankungen auf die Zinss{\"a}tze kommen, stehen immernoch die \textbf{Feinsteuerungsoperationen} zur Verf{\"u}gung, die von Fall zu Fall mit befristeten Transaktionen, oder in der Form von definitiven Verk{\"a}ufen oder K{\"a}ufen ausgef{\"u}hrt werden k{\"o}nnen.


Als letztes Instrument der Offenmarktpolitik stehen der EZB die \textbf{Strukturellen Operationen} zur Verf{\"u}gung, die das Ziel haben, grundlegende Liquidit{\"a}tspositionen des Finanzsektors zu beeinflussen. Dies passiert {\"u}ber die Emission von Schuldverschreibungen, definitive (Ver-)K{\"a}ufe und befristete Transaktionen.

(S.560)

\citep[vgl.][S.560ff]{Basseler2010}
\subsubsection{St{\"a}ndige Fazilit{\"a}ten}

Will die EZB eine genaue Steuerung des Geldmarktzinssatze, so greift sie auf das Instrument der St{\"a}ndigen Fazilit{\"a}t zur{\"u}ck. Hierunter versteht man die Bereitstellung, bzw. Absch{\"o}pfung von Liquidit{\"a}t jeweils bis zum n{\"a}chsten Gesch{\"a}ftstag in form von Tageskrediten oder t{\"a}glichen Anlagen {\"u}bersch{\"u}ssiger Liqudit{\"a}t. Im Unterschied zur Offenmarktpolitik erfolgt die Inanspruchnahme der st{\"a}ndigen Fazilit{\"a}ten auf Initiative der Banken und ist grunds{\"a}tzlich unbeschr{\"a}nkt m{\"o}glich, werden jedoch nur in geringem Umfang genutzt, weil die Konditionen im Vergleich zu den Konditionen am Interbankenmarkt relativ ung{\"u}nstig sind.

Gesch{\"a}ftsbanken k{\"o}nnen zur Deckung eines vor{\"u}bergehenden Liquidit{\"a}tsbedarfs unbegrenz die Spitzenrefinanzierungsfazilit{\"a}t in Anspruch nehmen\footnote[36]{Dieser Bedarf wird mit einem im Voraus bekanntgegebenen Kreditzinssatz verzinst, dieser Zinssatz ist die Obergrenze des allgemeinen Tagesgeldsatzes am Geldmarkt}
, m{\"u}ssen dazu aber von der EZB als Gesch{\"a}ftspartner zugelassen sein.

Haben die Gesch{\"a}ftsbanken eine {\"u}bersch{\"u}ssige Liqudit{\"a}t, sin k{\"o}nnen sie auf die Einlagefazilit{\"a}t zur{\"u}ckgreifen, und die Liquidit{\"a}t bis zum n{\"a}chsten Gesch{\"a}ftstag bei den nationalen Zentralbanken anlegen.\footnote[37]{Diese Einlage wird mit einem im Voraus bekanntgegebenen Zinssatz verzinst, dieser Zinssatz ist die Untergrenze des allgemeinen Tagesgeldsatzes am Geldmarkt}

(S.562)
F{\"u}r gew{\"o}hnlich bewegt sich der Leitzins der EZB zwischen dem Zinssatz f{\"u}r die Einlagefazilit{\"a}t und der Spitzenrefinanzierungsfazilit{\"a}t, der Zinssatz von weniger als 1 Prozent f{\"u}r die Hauptrefinanzierung stellt einen bisher unerreichten historischen Tiefpunkt f{\"u}r die EZB dar.
Das die EZB Wert darauf legt, den Geldmarktzinssatz genau zu steuern, zeigt, dass auch Keynsianische Elemente in die Geldpolitik der EZB einflie{"ss}en.


\subsubsection{Mindestreservepolitik} \citep[vgl.][S.562f]{Basseler2010}
In der Europ{\"a}ischen Geldpolitik ist den Gesch{\"a}ftsbanken  vorgeschrieben bei der Zentralbank einen bestimmten Prozentsatz ihrer Einlagen - den Mindestreservesatz -  als Sichtguthaben vor zu halten. Die Mindestreservepolitik ist daf{\"u}r gedacht, einen stark wirkenden Einfluss auf die Geldsch{\"o}pfung, bzw. das Geldsch{\"o}pfungspotentioal der EZB zu erhalten. Gibt die EZB eine Erh{\"o}hung des Mindestreservesatz vor, nimmt das Geldsch{\"o}pfungspotential zu, und  bei Senkung nimmt es ab. Ein gewollter Nebeneffekt ist, dass bei einer {\"a}nderung des Mindestreservesatzes auch die Liqudit{\"a}tsreserven der Gesch{\"a}ftsbanken sich ver{\"a}ndern.
Die Mindestreservepolitik schafft so einen stabilen zus{\"a}tzlichen Zentralbankgeldbedarf und stellt so die direkte Verbindung zwischen Mindestreserve und Geldsch{\"o}pfung her.
Mindestreserven muss jedes Kreditinstitut im Eurosystem halten, wobei die Mindestreserven nur im Monatsdurchschnitt gehalten werden m{\"u}ssen. Gehaltene verzinste Mindestreserven werden dabei mit dem Zinssatz f{\"u}r das Hauptrefinanzierungsgesch{\"a}ft verzinst.


(S.564\citep[vgl.][S.564-568]{Basseler2010}

Geldpolitische Strategien in Europa
::anmerkung:: Geldmenge M3 erkl{\"a}ren

(S.565)
Die geldpolitische Strategie des Eurosystems ist von Rat der Europ{\"a}ischen Zentralbanken entwickelt und am 13.10.1998 der {\"o}ffentlichkeit pr{\"a}sentiert worden.
Zentrale Elemente sind seitdem:
\begin{itemize}
    \item{Das Hauptziel der Preisstabilit{\"a}t, definiert als Anstieg des sogenannten Harmonisierten Verbraucherpreisindex (HVPI) f{\"u}r den Euroraum von unter 2 Prozent\footnote[78]{Seit einf{\"u}hrung des Euros ist diese Marke allerdings jedes Jahr {\"u}berschritten worden, gerade in der Finanzkrise mit mehr als 1 Prozent j{\"a}hrlich.}Wobei die Preisstabilit{\"a}t nur mittelfristig gew{\"a}hrleistet werden muss. %\cite{EZB 2004, S.52}
    }
    \item{Eine Geldmengenpolitik mit der Verk{\"u}ndung eines Referenzwertes f{\"u}r das Wachstum der Geldmenge M3}
    \item{Beobachtung und Einsch{\"a}tzung der k{\"u}nftigen Preisentwicklung sowie der Preisstabilit{\"a}t des Euroraums insgesamt.}
\end{itemize}
Als Oberstes Ziel der Geldpolitik ist die Preisstabili{\"a}t ausgegeben, wobei das Ziel 2003 dahingehend pr{\"a}zisiert wurde, dass mittelfristig eine Preissteigerungsrate unter, aber ann{\"a}hernd 2 Prozent gegen{\"u}ber den Vorjahr sein muss. Hiermit wird auch versucht, delation{\"a}re Geldpolitik seitens der Nationalen Zentralbanken zu unterbinden.




\end{document}
