\documentclass[
    onecolumn,
    a4paper,
    abstracton,
    parskip=half
    %,draft
    ,final
    ]{scrartcl}

    \usepackage[pdftex
    ,draft
    ]{graphicx}

    \usepackage{booktabs}

    \usepackage[cmex10]{amsmath}

    \interdisplaylinepenalty=2500
    \usepackage{url}

    \usepackage[breaklinks]{hyperref}
    \hyphenation{nothing} % correct bad hyphenation here

    \usepackage{eurosym}

    \usepackage{listings}
    \lstset{basicstyle=\small\ttfamily,breaklines=true}
    \emergencystretch 1000pt

    \usepackage{subcaption}

    \usepackage{mathtools}

    % deutsche Silbentrennung
    \usepackage[ngerman]{babel}

    \usepackage[printonlyused, withpage]{acronym}

    \usepackage[a4paper]{geometry}

    % wegen deutschen Umlauten
    \usepackage[ansinew]{inputenc}

    % fuer Zitate
    \usepackage[round]{natbib}

    \usepackage{setspace}

    \usepackage{units}
    \usepackage{cite}
    %%% Deutsche Verzeichnis-ueberschriften



    %%% Kommentarfunktion %%%
    \usepackage[textwidth=2.2cm
    ,obeyFinal
    ]{todonotes}

\begin{document}
Anmerkung: Geldschöpfung und Geldarten sind bei supranationaler Institution monopolisiert.



\subsection{Organisationen der Europ{\"a}ischen Geldpolitk}

Die Europ{\"a}ische Geldpolitik wird durch das Europ{\"a}ische System der Zentralbanken (ESZB) und die EZB organisiert, wobei sich die ESZB aus der Europ{\"a}ischen Zentralbank und allen 27 nationalen Zentralbanken (NZBen) der Mitgliedsstaaten der EU organisiert.
Sonderstatus haben dabei die sogenannten "`Outs"', jene Mitgliederstaaten der EU, die den Euro  nicht eingef{\"u}hrt haben. Dies sind Derzeit: D{\"a}nemark, Gro{"ss}britanien, Schweden sowie die meisten neuen EU-Mitgliedsstaaten nach 2001. Sie sind vom Entscheidungsprozess der ESZB ausgeschlossen und vollziehen eine eigenst{\"a}ndige nationale Geldpolitik.
Auch wenn der EG-Vertrag\footnote[25]{EGV, Art. 105-109 d} formal zwischen EZB und ESZB unterscheidet, entscheidet doch faktisch nur eine Institution, die der EZB mit ihren Beschlussorganen (EZB-Rat und Direktorium der EZB.) \citep[vgl.][S.553]{Basseler2010})

\subsection{Die Europ{\"a}ische Zentralbank}
Die Beschlussorgane der Europ{\"a}ischen Zentralbank sind der EZB-Rat und das Direktorium der EZB, welche gemeinsam die EZB leiten. Das Direktorium der EZB besteht aus dem Pr{\"a}sidenten und dem Vizepr{\"a}sidenten der EZB, sowie weiteren vier Mitgliedern, die von den Regierungen der Mitgliedsstaaten auf der Ebene der Staats- und Regierungschefs auf Empfehlung des EU-Rats einvernehmlich ernennt. Ein Anh{\"o}rungsrecht besteht dabei beim Europ{\"a}ischem Parlament und beim EZB-Rat.
Der EZB-Rat wiederum besteht aus dem Direktorium und den Pr{\"a}sidenten aller nationalen Zentralbanken, die den Euro gemeinsam eingef{\"u}hrt haben.
Die exekutive Gewalt liegt innerhalb der Europ{\"a}ischen Zentralbank beim Direktorium, das die f{\"u}r die Durchf{\"u}hrung der Geldpolitik nach den Leitlinien und Beschl{\"u}ssen des EZB-Rates verantwortlich ist. Auch ist das Direktorium der EZB weisungsbefugt gegen{\"u}ber den nationalen Zentralbanken des Eurosystems.  Damit haben wir ein duales System, bestehend aus dem Exekutivorgan der EZB in Form des Direktoriums und ein ein Beschlussorgan in Form des EZB-Rates.
Der EZB-Rat erarbeitet und erl{\"a}sst die Beschl{\"u}sse einer gemeinschaftlichen europ{\"a}ischen Geldpolitik des Euro-Raumes, um die Ausgabe von M{\"u}nzen und Banknoten zu regeln und die Erf{\"u}llung von dem ESZB {\"u}bertragenen Aufgaben zu erf{\"u}llen.  ( \citep[vgl.][S.553]{Basseler2010} ) Derzeit umfasst der EZB-Rat 18 L{\"a}nder und das Direktorium, abgestimmt wird mit einfacher Mehrheit der Anwesenden, bei Stimmengleichheit entscheidet die Stimme des Pr{\"a}sidenten.




\subsubsection{Die Politik der St{\"a}ndigen Fazilit{\"a}ten}
\subsubsection{Die Mindestreservepolitik.}
\subsubsection{Weitere Instrumente}






Die Unabh{\"a}ngigkeit der EZB



\end{document}
