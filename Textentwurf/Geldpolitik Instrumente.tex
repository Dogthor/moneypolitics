\documentclass[
    onecolumn,
    a4paper,
    abstracton,
    parskip=half
    %,draft
    ,final
    ]{scrartcl}

    \usepackage[pdftex
    ,draft
    ]{graphicx}

    \usepackage{booktabs}

    \usepackage[cmex10]{amsmath}

    \interdisplaylinepenalty=2500
    \usepackage{url}

    \usepackage[breaklinks]{hyperref}
    \hyphenation{nothing} % correct bad hyphenation here

    \usepackage{eurosym}

    \usepackage{listings}
    \lstset{basicstyle=\small\ttfamily,breaklines=true}
    \emergencystretch 1000pt

    \usepackage{subcaption}

    \usepackage{mathtools}

    % deutsche Silbentrennung
    \usepackage[ngerman]{babel}

    \usepackage[printonlyused, withpage]{acronym}

    \usepackage[a4paper]{geometry}

    % wegen deutschen Umlauten
    \usepackage[ansinew]{inputenc}

    % fuer Zitate
    \usepackage[round]{natbib}

    \usepackage{setspace}

    \usepackage{units}
    \usepackage{cite}
    %%% Deutsche Verzeichnis-ueberschriften



    %%% Kommentarfunktion %%%
    \usepackage[textwidth=2.2cm
    ,obeyFinal
    ]{todonotes}

\begin{document}



\subsection{Organisationen der Europ{\"a}ischen Geldpolitk}

Die Europ{\"a}ische Geldpolitik wird durch das Europ{\"a}ische System der Zentralbanken (ESZB) und die EZB organisiert, wobei sich die ESZB aus der Europ{\"a}ischen Zentralbank und allen 27 nationalen Zentralbanken (NZBen) der Mitgliedsstaaten der EU organisiert.  Damit sind Geldarten und Geldsch{\"o}pfung bei einer supranationalen Institution monopolisiert. Sonderstatus haben dabei die sogenannten "`Outs"', jene Mitgliederstaaten der EU, die den Euro  nicht eingef{\"u}hrt haben. Dies sind Derzeit: D{\"a}nemark, Gro{"ss}britanien, Schweden sowie die meisten neuen EU-Mitgliedsstaaten nach 2001. Sie sind vom Entscheidungsprozess der ESZB ausgeschlossen und vollziehen eine eigenst{\"a}ndige nationale Geldpolitik.
Auch wenn der EG-Vertrag\footnote[25]{EGV, Art. 105-109 d} formal zwischen EZB und ESZB unterscheidet, entscheidet doch faktisch nur eine Institution, die der EZB mit ihren Beschlussorganen (EZB-Rat und Direktorium der EZB.) \citep[vgl.][S.553]{Basseler2010})

\subsection{Die Europ{\"a}ische Zentralbank}
Die Beschlussorgane der Europ{\"a}ischen Zentralbank sind der EZB-Rat und das Direktorium der EZB, welche gemeinsam die EZB leiten. Das Direktorium der EZB besteht aus dem Pr{\"a}sidenten und dem Vizepr{\"a}sidenten der EZB, sowie weiteren vier Mitgliedern, die von den Regierungen der Mitgliedsstaaten auf der Ebene der Staats- und Regierungschefs auf Empfehlung des EU-Rats einvernehmlich ernennt. Ein Anh{\"o}rungsrecht besteht dabei beim Europ{\"a}ischem Parlament und beim EZB-Rat.
Der EZB-Rat wiederum besteht aus dem Direktorium und den Pr{\"a}sidenten aller nationalen Zentralbanken, die den Euro gemeinsam eingef{\"u}hrt haben.
Die exekutive Gewalt liegt innerhalb der Europ{\"a}ischen Zentralbank beim Direktorium, das die f{\"u}r die Durchf{\"u}hrung der Geldpolitik nach den Leitlinien und Beschl{\"u}ssen des EZB-Rates verantwortlich ist. Auch ist das Direktorium der EZB weisungsbefugt gegen{\"u}ber den nationalen Zentralbanken des Eurosystems.  Damit haben wir ein duales System, bestehend aus dem Exekutivorgan der EZB in Form des Direktoriums und ein ein Beschlussorgan in Form des EZB-Rates.
Der EZB-Rat erarbeitet und erl{\"a}sst die Beschl{\"u}sse einer gemeinschaftlichen europ{\"a}ischen Geldpolitik des Euro-Raumes, um die Ausgabe von M{\"u}nzen und Banknoten zu regeln und die Erf{\"u}llung von dem ESZB {\"u}bertragenen Aufgaben zu erf{\"u}llen.  ( \citep[vgl.][S.553]{Basseler2010} ) Derzeit umfasst der EZB-Rat 18 L{\"a}nder und das Direktorium, abgestimmt wird mit einfacher Mehrheit der Anwesenden, bei Stimmengleichheit entscheidet die Stimme des Pr{\"a}sidenten.

\subsection{Ziele und Aufgaben von ESZB und EZB}
Die EG-Vertr{\"a}ge definieren als vorangiges Ziel des ESZB und damit der EZB "`die Gew{\"a}hrleistung der Preisstabilit{\"a}t."'
"`Soweit dies ohne Beeintr{\"a}chtigung des Zieles der Preisstabilit{\"a}t m{\"o}glich ist, unterst{\"u}tz das ESZB die allgemeine Wirtschaftpolitik der Gemeinschaft, um die Verwirklichung der in Artikel 2 festgelegten Ziele der Gemeinsachft beizutragen. Das ESZB handelt im Einklang mit dem Grundsatz einer offenen Marktswirtschaft mit freien Wettbewerb \ldots"'\citep[vgl.][S.554]{Basseler2010}
Hier wird von der EZB eine Priorisierung der Preisstabilit{\"a}t gegen{\"u}ber anderen Zielen wie Vollbesch{\"a}ftigung und Wachstum festgeschrieben - diese weiteren Ziele werden der Preisstabilit{\"a}t untergeordnet. Verglichen mit der Zielvorschrift der Deutschen Bundesbank, entspricht diese Formulierung weitestgehend der damals geltenden Zielvorschrift.\citep[vgl.][S.554]{Basseler2010}

Die ideologische Grundlage f{\"u}r diese Priorisierung ist die von monetaristischen Str{\"o}mungen ausgehende Vorstellung, dass eine Zentralbank zu aller erst die Verantwortung f{\"u}r eine Preisstabilit{\"a}t besitzt, da andere Akteure f{\"u}r die Vollbesch{\"a}ftigung zust{\"a}ndig sind (z.B. Gewerkschaften, Tarifparteien, etc.) und der Wachstum sich aus dem technischen Fortschritt und dem Bev{\"o}lkerungswachstum ergibt.

Die haupts{\"a}chlichen Aufgaben des ESZB werden im Art. 105, Abs. 2 EGV wie folgt festgelegt:
\begin{itemize}
    \item{die Geldpolitik der Gemeinschaft festzulegen und auszuf{\"u}hren, Divisengesch{\"a}fte im Einklang mit Artikel 111 durchzuf{\"u}hren,}
    \item{die offiziellen W{\"a}hrungsreserven der Mitgliedstaaten zu halten und zu verwalten,}
    \item{das reibungslose Funktionieren des Zahlungssystems zu f{\"o}rdern}
\end{itemize}

Bevor wir weiter auf die Geldpolitik und die Instrumente der EZB eingehen, sei angemerkt, was jener Artikel 111 EGV, welcher die Devisengesch{\"a}fte der EZB regelt beinhaltet:
So legt dieser Artikel fest, das alle Entscheidungen {\"u}ber die Wechselkurssysteme, ob nun flexible oder feste Wechselkurse, oder ihre H{\"o}he bei Festlegung fester Wechselkurse dem Ministerrat vorbehalten sind.  \citep[vgl.][S.555]{Basseler2010}


\subsection{Instrumente der Europ{\"a}ischen Geldpolitik}

Im Folgenden soll es um die Instrumente der Europ{\"a}ischen Geldpolitik durch die EZB gehen, als da w{\"a}ren:
\begin{enumerate}
  \item{Die Offenmarktpolitik}
  \item{Die Politik der St{\"a}ndigen Fazilit{\"a}ten}
  \item{Die Mindestreservepolitik.}
  \item{Weitere Instrumente}
\end{enumerate}

\subsubsection{Die Offenmarktpolitik}
Nach g{\"a}ngiger Politik- und Wirtschaftstheorie bedarf es in einer wachsenden Volkswirtschaft, um eine hinreichende Geldversorgung der Wirtschaft zu gew{\"a}hrleisten, einer fortw{\"a}hrenden Ausweitung der nominalen Geldmenge. Als zentrales Instrument der Geldpolitik kommt hier die Offenmarktpolitik der Europ{\"a}ischen Zentralbank zum Tragen.
Vereinfacht gesprochen, wird unter der Offenmarktpolitk nichts anderes verstanden, als der An- und Verkauft von Wertpapieren gegen Zentralbankgeld durch die europ{\"a}ische Zentralbank. Bezweckt wird mit den Offenmarktk{\"a}ufen, bzw. -verk{\"a}ufen eine Zentralbankgeldsch{\"o}pfung bzw. -vernichtung zur Ver{\"a}nderung der nominalen Geldmenge. Allerdings ist der Begriff "`Offen"'-Marktgesch{\"a}ft irref{\"u}hrend: In der ESZB sind als Gesch{\"a}ftspartner der EZB nur finanziell solide MFIs zugelassen, die in das Mindestreservesystem einbezogen sind.

Bei Offenmarktgesch{\"a}ften erh{\"o}ht sich der Bestand an zentralbankgeld der Gesch{\"a}ftsbanken beim Kauf von Wertpapieren, doch beim Verkauft von Wertpapieren sinkt der Bestand an Zentralbankgeld der Gesch{\"a}ftsbanken. Der Kauf von Wertpapieren durch die Zentralbank wird mitunter auch expansive Offenmarktpolitik genannt.
Wichtig ist zu wissen, dass die Europ{\"a}ische Zentralbank die Gesch{\"a}ftsbanken nicht zum Kauf von Wertpapieren zwingen kann und somit attraktive Konditionen bieten muss: So sinkt bei einer geplanten expansiven Offenmarktpolitik der Zinssatz f{\"u}r die Zentralbankgeld-Kreditgew{\"a}hrung unter den {\"u}blichen Geldmarktzins. Entsprechend umgekehrt ist es bei der {\"a}u{"ss}erst selten vorkommenden, sogenannten kontraktiven Offenmarktpolitik. Hier m{\"u}ssen die von der Zentralbank angebitenen Zinss{\"a}tze h{\"o}her sein, als die sonst {\"u}blichen Geldmarktzinsen.

Offenmarktgesch{\"a}fte treten h{\"a}ufig in der Form von Repo-Gesch{\"a}ften auf. Dabei handelt es sich nach der englischen Bezeichnung Repurchase um R{\"u}ckkauf-Gesch{\"a}fte, d.h. es wird mit den Gesch{\"a}ftsbanken eine Laufzeit der Wertpapiere vereinbart, an deren Ende diese ihre Wertpapiere zur{\"u}ckkaufen m{\"u}ssen. Hiermit entsteht ein automatischer R{\"u}ckfluss von Zentralbankgeld. Diese auf Frist gesetzten Repo-Gesch{\"a}fte werden auch Wertpapierpensionsgesch{\"a}fte genannt, der entsprechend vorkommende Zinssatz wird demgem{\"a}{"ss} als Pensionssatz bezeichnet.
Die Repo-Gesch{\"a}fte sind aus Sicht der g{\"a}ngigen Wirtschaftstheorie ein gut steuerbares Instrument f{\"u}r die Entwicklung der Zentralbankgeldmenge im Gesch{\"a}ftsbankensektor, da sich auch kontraktive Effekte einstellen, wenn keine neuen Repo-Gesch{\"a}fte abgeschlossen werden. Eine r{\"u}ckl{\"a}ufige Anzahl an Repo-Gesch{\"a}ften f{\"u}hrt somit zu einer Abnahme der Zentralbankgeldmenge im Sektor der Gesch{\"a}ftsbanken.








\subsubsection{Die Politik der St{\"a}ndigen Fazilit{\"a}ten}
\subsubsection{Die Mindestreservepolitik.}
\subsubsection{Weitere Instrumente}






Die Unabh{\"a}ngigkeit der EZB



\end{document}
