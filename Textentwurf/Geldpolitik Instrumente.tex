\documentclass[
    onecolumn,
    a4paper,
    abstracton,
    parskip=half
    %,draft
    ,final
    ]{scrartcl}

    \usepackage[pdftex
    ,draft
    ]{graphicx}

    \usepackage{booktabs}

    \usepackage[cmex10]{amsmath}

    \interdisplaylinepenalty=2500
    \usepackage{url}

    \usepackage[breaklinks]{hyperref}
    \hyphenation{nothing} % correct bad hyphenation here

    \usepackage{eurosym}

    \usepackage{listings}
    \lstset{basicstyle=\small\ttfamily,breaklines=true}
    \emergencystretch 1000pt

    \usepackage{subcaption}

    \usepackage{mathtools}

    % deutsche Silbentrennung
    \usepackage[ngerman]{babel}

    \usepackage[printonlyused, withpage]{acronym}

    \usepackage[a4paper]{geometry}

    % wegen deutschen Umlauten
    \usepackage[ansinew]{inputenc}

    % fuer Zitate
    \usepackage[round]{natbib}

    \usepackage{setspace}

    \usepackage{units}
    \usepackage{cite}
    %%% Deutsche Verzeichnis-ueberschriften



    %%% Kommentarfunktion %%%
    \usepackage[textwidth=2.2cm
    ,obeyFinal
    ]{todonotes}

\begin{document}



\subsection{Organisationen der Europ{\"a}ischen Geldpolitk}

Die Europ{\"a}ische Geldpolitik wird durch das Europ{\"a}ische System der Zentralbanken (ESZB) und die EZB organisiert, wobei sich die ESZB aus der Europ{\"a}ischen Zentralbank und allen 27 nationalen Zentralbanken (NZBen) der Mitgliedsstaaten der EU organisiert.  Damit sind Geldarten und Geldsch{\"o}pfung bei einer supranationalen Institution monopolisiert. Sonderstatus haben dabei die sogenannten \"outs"', jene Mitgliederstaaten der EU, die den Euro  nicht eingef{\"u}hrt haben. Dies sind Derzeit: D{\"a}nemark, Gro{"ss}britanien, Schweden sowie die meisten neuen EU-Mitgliedsstaaten nach 2001. Sie sind vom Entscheidungsprozess der ESZB ausgeschlossen und vollziehen eine eigenst{\"a}ndige nationale Geldpolitik.
-Auch wenn der EG-Vertrag\footnote[25]{EGV, Art. 105-109 d} formal zwischen EZB und ESZB unterscheidet, entscheidet doch faktisch nur eine Institution, die der EZB mit ihren Beschlussorganen (EZB-Rat und Direktorium der EZB.) \citep[vgl.][S.553]{Basseler2010})

\subsection{Die Europ{\"a}ische Zentralbank}
Die Beschlussorgane der Europ{\"a}ischen Zentralbank sind der EZB-Rat und das Direktorium der EZB, welche gemeinsam die EZB leiten. Das Direktorium der EZB besteht aus dem Pr{\"a}sidenten und dem Vizepr{\"a}sidenten der EZB, sowie weiteren vier Mitgliedern, die von den Regierungen der Mitgliedsstaaten auf der Ebene der Staats- und Regierungschefs auf Empfehlung des EU-Rats einvernehmlich ernennt. Ein Anh{\"o}rungsrecht besteht dabei beim Europ{\"a}ischem Parlament und beim EZB-Rat.
Der EZB-Rat wiederum besteht aus dem Direktorium und den Pr{\"a}sidenten aller nationalen Zentralbanken, die den Euro gemeinsam eingef{\"u}hrt haben.
Die exekutive Gewalt liegt innerhalb der Europ{\"a}ischen Zentralbank beim Direktorium, das die f{\"u}r die Durchf{\"u}hrung der Geldpolitik nach den Leitlinien und Beschl{\"u}ssen des EZB-Rates verantwortlich ist. Auch ist das Direktorium der EZB weisungsbefugt gegen{\"u}ber den nationalen Zentralbanken des Eurosystems.  Damit haben wir ein duales System, bestehend aus dem Exekutivorgan der EZB in Form des Direktoriums und ein ein Beschlussorgan in Form des EZB-Rates.
Der EZB-Rat erarbeitet und erl{\"a}sst die Beschl{\"u}sse einer gemeinschaftlichen europ{\"a}ischen Geldpolitik des Euro-Raumes, um die Ausgabe von M{\"u}nzen und Banknoten zu regeln und die Erf{\"u}llung von dem ESZB {\"u}bertragenen Aufgaben zu erf{\"u}llen.  ( \citep[vgl.][S.553]{Basseler2010} ) Derzeit umfasst der EZB-Rat 18 L{\"a}nder und das Direktorium, abgestimmt wird mit einfacher Mehrheit der Anwesenden, bei Stimmengleichheit entscheidet die Stimme des Pr{\"a}sidenten.

\subsection{Ziele und Aufgaben von ESZB und EZB}
Die EG-Vertr{\"a}ge definieren als vorangiges Ziel des ESZB und damit der EZB "`die Gew{\"a}hrleistung der Preisstabilit{\"a}t."'
"`Soweit dies ohne Beeintr{\"a}chtigung des Zieles der Preisstabilit{\"a}t m{\"o}glich ist, unterst{\"u}tz das ESZB die allgemeine Wirtschaftpolitik der Gemeinschaft, um die Verwirklichung der in Artikel 2 festgelegten Ziele der Gemeinsachft beizutragen. Das ESZB handelt im Einklang mit dem Grundsatz einer offenen Marktswirtschaft mit freien Wettbewerb \ldots"'\citep[vgl.][S.554]{Basseler2010}
Hier wird von der EZB eine Priorisierung der Preisstabilit{\"a}t gegen{\"u}ber anderen Zielen wie Vollbesch{\"a}ftigung und Wachstum festgeschrieben - diese weiteren Ziele werden der Preisstabilit{\"a}t untergeordnet. Verglichen mit der Zielvorschrift der Deutschen Bundesbank, entspricht diese Formulierung weitestgehend der damals geltenden Zielvorschrift.\citep[vgl.][S.554]{Basseler2010}

Die ideologische Grundlage f{\"u}r diese Priorisierung ist die von monetaristischen Str{\"o}mungen ausgehende Vorstellung, dass eine Zentralbank zu aller erst die Verantwortung f{\"u}r eine Preisstabilit{\"a}t besitzt, da andere Akteure f{\"u}r die Vollbesch{\"a}ftigung zust{\"a}ndig sind (z.B. Gewerkschaften, Tarifparteien, etc.) und der Wachstum sich aus dem technischen Fortschritt und dem Bev{\"o}lkerungswachstum ergibt.

Die haupts{\"a}chlichen Aufgaben des ESZB werden im Art. 105, Abs. 2 EGV wie folgt festgelegt:
\begin{itemize}
    \item{die Geldpolitik der Gemeinschaft festzulegen und auszuf{\"u}hren, Divisengesch{\"a}fte im Einklang mit Artikel 111 durchzuf{\"u}hren,}
    \item{die offiziellen W{\"a}hrungsreserven der Mitgliedstaaten zu halten und zu verwalten,}
    \item{das reibungslose Funktionieren des Zahlungssystems zu f{\"o}rdern}
\end{itemize} \citep[vgl.][S.555]{Basseler2010} "

subsubsection{Unabh{\"a}ngigkeit der EZB}
(S.555)
Wenn es das vorrangige Ziel einer Zentralbank ist, Preisstabilit{\"a}t zu gew{\"a}hrleisten, dann gilt ihre Unabh{\"a}ngigkeit als Zentral. (...)

Zum Ersten ist die EZB funktional relativ unabh{\"a}ngig, weil sie Weisungen nicht entgegenhemen darf. \footnote[34]{vgl. Artikel 107, EGV (Unabh{\"a}ngigkeit der EZB)}

(S.556)
Eine solche Unabh{\"a}ngigkeit - keinerlei Kontrollen durch Regierungen und Parlamente unterworfen zu sein -  ist relativ einzigartig. Sie wird nur dadurch ein klein wenig beschr{\"a}nkt, dass die Verpflichtung besteht, die allgemeine Wirtschaftspolitik der Gemeinschaft zu unterst{\"u}tzen, aber nur, wenn dadurch das Ziel der Preisstabilit{\"a}t nicht beeintr{\"a}chtigt ist.

Zum Zweiten ist die EZB auch personell unabh{\"a}gig, einzig {\"u}ber die Ernennung der Pr{\"a}sidenten der Nationalbanken k{\"o}nnen einzelne Regierungen Einfluss aus{\"u}ben.

Zum Dritten ist die EZB auch finanziell unabh{\"a}ngig - sie verf{\"u}gt {\"u}ber eigene Einnahmen und einen eigenen Haushalt - und sie hat Kontrolle {\"u}ber die Instrumente der Geldpolitik.


Bevor wir weiter auf die Geldpolitik und die Instrumente der EZB eingehen, sei angemerkt, was jener Artikel 111 EGV, welcher die Devisengesch{\"a}fte der EZB regelt beinhaltet:
So legt dieser Artikel fest, das alle Entscheidungen {\"u}ber die Wechselkurssysteme, ob nun flexible oder feste Wechselkurse, oder ihre H{\"o}he bei Festlegung fester Wechselkurse dem Ministerrat vorbehalten sind. \citep[vgl.][S.555]{Basseler2010}




\subsection{Instrumente der Europ{\"a}ischen Geldpolitik}

Im Folgenden soll es um die Instrumente der Europ{\"a}ischen Geldpolitik durch die EZB gehen, als da w{\"a}ren:
\begin{enumerate}
  \item{Die Offenmarktpolitik}
  \item{Die Politik der St{\"a}ndigen Fazilit{\"a}ten}
  \item{Die Mindestreservepolitik.}
  \item{Weitere Instrumente}
\end{enumerate}




\subsubsection{Allgemeine Offenmarktpolitik und die Offenmarktpolitik der EZB im Speziellen}

(S.556)
Zentrales Instrument f{\"u}r die Geldversorgung der Wirtschaft ist die Offenmarktpolitik. Hierunter versteht man den An- und Verkauf von Wertpapieren gegen Zentralbankgeld durch die Zentralbank.
Durch Offenmarktk{\"a}ufe bzw. -verk{\"a}ufe kann die Zentralbank die Zentralbankgeldsch{\"o}pfung steuern. F{\"u}r die EZB im ESZB sind nur finanziell solide MFIs, sprich Finanzinstitute, die in das Mindestreservesystem einbezogen sind, als Gesch{\"a}ftspartner der EZB zugelassen.
Es wird allgemein zwischen der expansiven (Kauf von Wertpapieren) und kontraktiven Offenmarktpolitik (Verkauf von Wertpapieren) unterschieden.

(S.557)
Nach einem Kauf von Wertpapieren hat sicher der Bestand an Zentralbankgeld der Gesch{\"a}ftsbanken erh{\"o}ht, diese erh{\"o}hte Geldbasis erm{\"o}glicht den den Gesch{\"a}ftsbanken Kredite an Nichtbanken weiterzugeben.

Die Zentralbank kann die Gesch{\"a}ftsbanken nicht zur Offenmarktpolitik zwingen, sondern muss entsprechende Anreize bieten. Diese bestehen in niedrigen Zinss{\"a}tzen f{\"u}r die Zentralbankgeld-Kreditgew{\"a}hrung unterhalb des Geldmarktzinses. Bei Kontraktiver Offenmarktpolitik in Zinss{\"a}tzen {\"u}ber dem Geldmarktzins.
Zeitlich befristete Offenmarktgesch{\"a}fte werden auch Wertpapierpensionsgesch{\"a}fte, oder Repo-Gesch{\"a}fte genannt \footnote[36]{Repo-Gesch{\"a}fte leitet sich vom englischen Repurchase - R{\"u}ckkauf ab}, der Zinssatz wird dabei Pensionssatz genannt \footnote[37]{Entsprechend hei{"ss}t der Pensionssatz auch Repo-Rate}
Mit solchen Repo-Gesch{\"a}ften l{\"a}sst sich die Zentralbankgeldmenge recht einfach steuern, da, wenn zeitlich begrenzte Gesch{\"a}fte auslaufen und nicht erneuert werden, automatisch kontraktive Effekte auftreten.


(S.558)
Offenmarktpolitik der EZB
Das Bankensystem im Eurogebiet ist auf die Bereitstellung von Zentralbankgeld durch die EZB angewiesen, weshalb die Offenmarktpolitik der EZB am Interbankengeldmarkt ansetzt. Dabei greift die EZB auf folgende Instrumente zur{\"u}ck:
\begin{enumerate}
  \item{Hauptrefinanzierungsinstrument}
  \item{l{\"a}ngerfristige Refinanzierungsgesch{\"a}fte}
  \item{Feinsteuerungsoption}
  \item{Strukturelle Operationen}
  \end{enumerate}










\end{document}
