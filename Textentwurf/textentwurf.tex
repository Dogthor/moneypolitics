\documentclass[
      onecolumn,
      a4paper,
      abstracton,
      parskip=half
      %,draft
      ,final
      ]{scrartcl}

      \usepackage[pdftex
      ,draft
      ]{graphicx}

      \usepackage{booktabs}

      \usepackage[cmex10]{amsmath}

      \interdisplaylinepenalty=2500
      \usepackage{url}

      \usepackage[breaklinks]{hyperref}
      \hyphenation{nothing} % correct bad hyphenation here

      \usepackage{eurosym}

      \usepackage{listings}
      \lstset{basicstyle=\small\ttfamily,breaklines=true}
      \emergencystretch 1000pt

      \usepackage{subcaption}

      \usepackage{mathtools}

      % deutsche Silbentrennung
      \usepackage[ngerman]{babel}

      \usepackage[printonlyused, withpage]{acronym}

      \usepackage[a4paper]{geometry}

      % wegen deutschen Umlauten
      \usepackage[ansinew]{inputenc}

      % fuer Zitate
      \usepackage[round]{natbib}

      \usepackage{setspace}

      \usepackage{units}
      \usepackage{cite}
      %%% Deutsche Verzeichnis-ueberschriften



      %%% Kommentarfunktion %%%
      \usepackage[textwidth=2.2cm
      ,obeyFinal
      ]{todonotes}
\begin{document}

\subsection{Beschreibung des eigenen Verst{\"a}ndnis von Geldpolitik}

Nach \citep[S. 551]{Basseler2010}{Basseler hat die Geldpolitik die Hauptaufgabe, eine optimale Geldversorgung der Wirtschaft zu gewährleisten. Real wird diese Aufgabe von einer größtenteils staatlichen, aber unabhängigen Zentralbank übernommen. Dabei herrscht weit verbreitet Konsens darüber, dass Geldpolitik staatliche Aufgabe bleibt, auch wenn Ideen einer dezentralen, dem Wettbewerb unterliegenden Geldversorgung durch private Geschäftsbanken und ein System konkurierender Parallelwährungen kursieren. (Vgl. Hayek) }




\end{document}
