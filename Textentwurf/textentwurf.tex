\documentclass[
      onecolumn,
      a4paper,
      abstracton,
      parskip=half
      %,draft
      ,final
      ]{scrartcl}

      \usepackage[pdftex
      ,draft
      ]{graphicx}

      \usepackage{booktabs}

      \usepackage[cmex10]{amsmath}

      \interdisplaylinepenalty=2500
      \usepackage{url}

      \usepackage[breaklinks]{hyperref}
      \hyphenation{nothing} % correct bad hyphenation here

      \usepackage{eurosym}

      \usepackage{listings}
      \lstset{basicstyle=\small\ttfamily,breaklines=true}
      \emergencystretch 1000pt

      \usepackage{subcaption}

      \usepackage{mathtools}

      % deutsche Silbentrennung
      \usepackage[ngerman]{babel}

      \usepackage[printonlyused, withpage]{acronym}

      \usepackage[a4paper]{geometry}

      % wegen deutschen Umlauten
      \usepackage[ansinew]{inputenc}

      % fuer Zitate
      \usepackage[round]{natbib}

      \usepackage{setspace}

      \usepackage{units}
      \usepackage{cite}
      %%% Deutsche Verzeichnis-ueberschriften



      %%% Kommentarfunktion %%%
      \usepackage[textwidth=2.2cm
      ,obeyFinal
      ]{todonotes}
\begin{document}

\subsubsection{Unabhängigkeit der EZB}
(S.555)
Wenn es das vorrangige Ziel einer Zentralbank ist, Preisstabilität zu gewährleisten, dann gilt ihre Unabhängigkeit als Zentral. (...)

Zum Ersten ist die EZB funktional relativ unabhängig, weil sie Weisungen nicht entgegenhemen darf. \footnote[34]{vgl. Artikel 107, EGV (Unabhängigkeit der EZB)}

(S.556)
Eine solche Unabhängigkeit - keinerlei Kontrollen durch Regierungen und Parlamente unterworfen zu sein -  ist relativ einzigartig. Sie wird nur dadurch ein klein wenig beschränkt, dass die Verpflichtung besteht, die allgemeine Wirtschaftspolitik der Gemeinschaft zu unterstützen, aber nur, wenn dadurch das Ziel der Preisstabilität nicht beeinträchtigt ist.

Zum Zweiten ist die EZB auch personell unabhägig, einzig über die Ernennung der Präsidenten der Nationalbanken können einzelne Regierungen Einfluss ausüben.

Zum Dritten ist die EZB auch finanziell unabhängig - sie verfügt pber eigene Einnahmen und einen eigenen Haushalt - und sie hat Kontrolle über die Instrumente der Geldpolitik.

\subsubsection{Allgemeine Offenmarktpolitik und die Offenmarktpolitik der EZB im Speziellen}

(S.556)
Zentrales Instrument für die Geldversorgung der Wirtschaft ist die Offenmarktpolitik. Hierunter versteht man den An- und Verkauf von Wertpapieren gegen Zentralbankgeld durch die Zentralbank.
Durch Offenmarktkäufe bzw. -verkäufe kann die Zentralbank die Zentralbankgeldschöpfung steuern. Für die EZB im ESZB sind nur finanziell solide MFIs, sprich Finanzinstitute, die in das Mindestreservesystem einbezogen sind, als Geschäftspartner der EZB zugelassen.
Es wird allgemein zwischen der expansiven (Kauf von Wertpapieren) und kontraktiven Offenmarktpolitik (Verkauf von Wertpapieren) unterschieden.

(S.557)



(S.558)
\end{document}
