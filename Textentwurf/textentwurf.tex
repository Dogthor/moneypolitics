\documentclass[
      onecolumn,
      a4paper,
      abstracton,
      parskip=half
      %,draft
      ,final
      ]{scrartcl}

      \usepackage[pdftex
      ,draft
      ]{graphicx}

      \usepackage{booktabs}

      \usepackage[cmex10]{amsmath}

      \interdisplaylinepenalty=2500
      \usepackage{url}

      \usepackage[breaklinks]{hyperref}
      \hyphenation{nothing} % correct bad hyphenation here

      \usepackage{eurosym}

      \usepackage{listings}
      \lstset{basicstyle=\small\ttfamily,breaklines=true}
      \emergencystretch 1000pt

      \usepackage{subcaption}

      \usepackage{mathtools}

      % deutsche Silbentrennung
      \usepackage[ngerman]{babel}

      \usepackage[printonlyused, withpage]{acronym}

      \usepackage[a4paper]{geometry}

      % wegen deutschen Umlauten
      \usepackage[ansinew]{inputenc}

      % fuer Zitate
      \usepackage[round]{natbib}

      \usepackage{setspace}

      \usepackage{units}
      \usepackage{cite}
      %%% Deutsche Verzeichnis-ueberschriften



      %%% Kommentarfunktion %%%
      \usepackage[textwidth=2.2cm
      ,obeyFinal
      ]{todonotes}
\begin{document}



(S.558)
Offenmarktpolitik der EZB
Das Bankensystem im Eurogebiet ist auf die Bereitstellung von Zentralbankgeld durch die EZB angewiesen, weshalb die Offenmarktpolitik der EZB am Interbankengeldmarkt ansetzt. Dabei greift die EZB auf folgende Instrumente zur{\"u}ck:
\begin{enumerate}
  \item{Hauptrefinanzierungsinstrument}
  \item{l{\"a}ngerfristige Refinanzierungsgesch{\"a}fte}
  \item{Feinsteuerungsoption}
  \item{Strukturelle Operationen}
  \end{enumerate}





  
(S.559)

(S.560)
(S.561)
(S.562)
(S.563)
(S.564)
(S.565)
(S.566)
(S.567)
(S.568)
\end{document}
