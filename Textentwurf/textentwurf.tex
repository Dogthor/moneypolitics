\documentclass[
      onecolumn,
      a4paper,
      abstracton,
      parskip=half
      %,draft
      ,final
      ]{scrartcl}

      \usepackage[pdftex
      ,draft
      ]{graphicx}

      \usepackage{booktabs}

      \usepackage[cmex10]{amsmath}

      \interdisplaylinepenalty=2500
      \usepackage{url}

      \usepackage[breaklinks]{hyperref}
      \hyphenation{nothing} % correct bad hyphenation here

      \usepackage{eurosym}

      \usepackage{listings}
      \lstset{basicstyle=\small\ttfamily,breaklines=true}
      \emergencystretch 1000pt

      \usepackage{subcaption}

      \usepackage{mathtools}

      % deutsche Silbentrennung
      \usepackage[ngerman]{babel}

      \usepackage[printonlyused, withpage]{acronym}

      \usepackage[a4paper]{geometry}

      % wegen deutschen Umlauten
      \usepackage[ansinew]{inputenc}

      % fuer Zitate
      \usepackage[round]{natbib}

      \usepackage{setspace}

      \usepackage{units}
      \usepackage{cite}
      %%% Deutsche Verzeichnis-ueberschriften



      %%% Kommentarfunktion %%%
      \usepackage[textwidth=2.2cm
      ,obeyFinal
      ]{todonotes}
\begin{document}

\subsection{Ziele und Aufgaben von ESZB und EZB}
Die EG-Verträge definieren als vorangiges Ziel des ESZB und damit der EZB "`die Gewährleistung der Preisstabilität."'
"`Soweit dies ohne Beeinträchtigung des Zieles der Preisstabilität möglich ist, unterstütz das ESZB die allgemeine Wirtschaftpolitik der Gemeinschaft, um die Verwirklichung der in Artikel 2 festgelegten Ziele der Gemeinsachft beizutragen. Das ESZB handelt im Einklang mit dem Grundsatz einer offenen Marktswirtschaft mit freien Wettbewerb \ldots"'\citep[vgl.][S.554]{Basseler2010}
Hier wird von der EZB eine Priorisierung der Preisstabilität gegenüber anderen Zielen wie Vollbeschäftigung und Wachstum festgeschrieben - diese weiteren Ziele werden der Preisstabilität untergeordnet. Verglichen mit der Zielvorschrift der Deutschen Bundesbank, entspricht diese Formulierung weitestgehend der damals geltenden Zielvorschrift.\citep[vgl.][S.554]{Basseler2010}

Die ideologische Grundlage für diese Priorisierung ist die von monetaristischen Strömungen ausgehende Vorstellung, dass eine Zentralbank zu aller erst die Verantwortung für eine Preisstabilität besitzt, da andere Akteure für die Vollbeschäftigung zuständig sind (z.B. Gewerkschaften, Tarifparteien, etc.) und der Wachstum sich aus dem technischen Fortschritt und dem Bevölkerungswachstum ergibt.


\end{document}
