\documentclass[
      onecolumn,
      a4paper,
      abstracton,
      parskip=half
      %,draft
      ,final
      ]{scrartcl}

      \usepackage[pdftex
      ,draft
      ]{graphicx}

      \usepackage{booktabs}

      \usepackage[cmex10]{amsmath}

      \interdisplaylinepenalty=2500
      \usepackage{url}

      \usepackage[breaklinks]{hyperref}
      \hyphenation{nothing} % correct bad hyphenation here

      \usepackage{eurosym}

      \usepackage{listings}
      \lstset{basicstyle=\small\ttfamily,breaklines=true}
      \emergencystretch 1000pt

      \usepackage{subcaption}

      \usepackage{mathtools}

      % deutsche Silbentrennung
      \usepackage[ngerman]{babel}

      \usepackage[printonlyused, withpage]{acronym}

      \usepackage[a4paper]{geometry}

      % wegen deutschen Umlauten
      \usepackage[ansinew]{inputenc}

      % fuer Zitate
      \usepackage[round]{natbib}

      \usepackage{setspace}

      \usepackage{units}
      \usepackage{cite}
      %%% Deutsche Verzeichnis-ueberschriften



      %%% Kommentarfunktion %%%
      \usepackage[textwidth=2.2cm
      ,obeyFinal
      ]{todonotes}
\begin{document}



(S.558)
Offenmarktpolitik der EZB
Das Bankensystem im Eurogebiet ist auf die Bereitstellung von Zentralbankgeld durch die EZB angewiesen, weshalb die Offenmarktpolitik der EZB am Interbankengeldmarkt ansetzt. Dabei greift die EZB auf folgende Instrumente zur{\"u}ck:
\begin{enumerate}
  \item{Hauptrefinanzierungsinstrument}
  \item{l{\"a}ngerfristige Refinanzierungsgesch{\"a}fte}
  \item{Feinsteuerungsoption}
  \item{Strukturelle Operationen}
  \end{enumerate}

Das \textbf{Hauptrefinanzierungsinstrument} und das \textbf{l{\"a}ngerfristige Refinanzierungsgesch{\"a}ft} sind regelm{\"a}{"ss}ig stattfindende Liquidit{\"a}t zuf{\"u}hrende Repo-Gesch{\"a}fte, wobei f{\"u}r das Hauptrefinanzierungsinstrument w{\"o}chentliche, auf eine Woche begrenzte Transaktionen gelten, w{\"a}hrend die Refinanzierungsgesch{\"a}fte mit monatlichem Abstand und mehre Monate (bis zu einem Jahr) Laufzeit haben. Das Hauptrefinanzierungsinstrument steuert die Zinss{\"a}tze (Zentraler Leitzins der EZB) sowie die Liquidit{\"a}t am europ{\"a}ischen Geldmarkt.

Sollte es zu unerwarteten marktm{\"a}{"ss}igen Liquidit{\"a}tsschwankungen auf die Zinss{\"a}tze kommen, stehen immernoch die \textbf{Feinsteuerungsoperationen} zur Verf{\"u}gung, die von Fall zu Fall mit befristeten Transaktionen, oder in der Form von definitiven Verk{\"a}ufen oder K{\"a}ufen ausgef{\"u}hrt werden k{\"o}nnen.

(S.559)
Als letztes Instrument der Offenmarktpolitik stehen der EZB die \textbf{Strukturellen Operationen} zur Verf{\"u}gung, die das Ziel haben, grundlegende Liquidit{\"a}tspositionen des Finanzsektors zu beeinflussen. Dies passiert {\"u}ber die Emission von Schuldverschreibungen, definitive (Ver-)K{\"a}ufe und befristete Transaktionen.

(S.560)

\subsubsection{St{\"a}ndige Fazilit{\"a}ten}

Will die EZB eine genaue Steuerung des Geldmarktzinssatze, so greift sie auf das Instrument der St{\"a}ndigen Fazilit{\"a}t zur{\"u}ck. Hierunter versteht man die Bereitstellung, bzw. Absch{\"o}pfung von Liquidit{\"a}t jeweils bis zum n{\"a}chsten Gesch{\"a}ftstag in form von Tageskrediten oder t{\"a}glichen Anlagen {\"u}bersch{\"u}ssiger Liqudit{\"a}t. Im Unterschied zur Offenmarktpolitik erfolgt die Inanspruchnahme der st{\"a}ndigen Fazilit{\"a}ten auf Initiative der Banken und ist grunds{\"a}tzlich unbeschr{\"a}nkt m{\"o}glich, werden jedoch nur in geringem Umfang genutzt, weil die Konditionen im Vergleich zu den Konditionen am Interbankenmarkt relativ ung{\"u}nstig sind.

Gesch{\"a}ftsbanken k{\"o}nnen zur Deckung eines vor{\"u}bergehenden Liquidit{\"a}tsbedarfs unbegrenz die Spitzenrefinanzierungsfazilit{\"a}t in Anspruch nehmen\footnote[36]{Dieser Bedarf wird mit einem im Voraus bekanntgegebenen Kreditzinssatz verzinst, dieser Zinssatz ist die Obergrenze des allgemeinen Tagesgeldsatzes am Geldmarkt}
, m{\"u}ssen dazu aber von der EZB als Gesch{\"a}ftspartner zugelassen sein.

(S.561)
Haben die Gesch{\"a}ftsbanken eine {\"u}bersch{\"u}ssige Liqudit{\"a}t, sin k{\"o}nnen sie auf die Einlagefazilit{\"a}t zur{\"u}ckgreifen, und die Liquidit{\"a}t bis zum n{\"a}chsten Gesch{\"a}ftstag bei den nationalen Zentralbanken anlegen.\footnote[37]{Diese Einlage wird mit einem im Voraus bekanntgegebenen Zinssatz verzinst, dieser Zinssatz ist die Untergrenze des allgemeinen Tagesgeldsatzes am Geldmarkt}

(S.562)
F{\"u}r gew{\"o}hnlich bewegt sich der Leitzins der EZB zwischen dem Zinssatz f{\"u}r die Einlagefazilit{\"a}t und der Spitzenrefinanzierungsfazilit{\"a}t, der Zinssatz von weniger als 1 Prozent f{\"u}r die Hauptrefinanzierung stellt einen bisher unerreichten historischen Tiefpunkt f{\"u}r die EZB dar.
Das die EZB Wert darauf legt, den Geldmarktzinssatz genau zu steuern, zeigt, dass auch Keynsianische Elemente in die Geldpolitik der EZB einflie{"ss}en.

\subsubsection{Mindestreservepolitik}
In der Europ{\"a}ischen Geldpolitik ist den Gesch{\"a}ftsbanken  vorgeschrieben bei der Zentralbank einen bestimmten Prozentsatz ihrer Einlagen - den Mindestreservesatz -  als Sichtguthaben vor zu halten. Die Mindestreservepolitik ist daf{\"u}r gedacht, einen stark wirkenden Einfluss auf die Geldsch{\"o}pfung, bzw. das Geldsch{\"o}pfungspotentioal der EZB zu erhalten. Gibt die EZB eine Erh{\"o}hung des Mindestreservesatz vor, nimmt das Geldsch{\"o}pfungspotential zu, und  bei Senkung nimmt es ab. Ein gewollter Nebeneffekt ist, dass bei einer {\"a}nderung des Mindestreservesatzes auch die Liqudit{\"a}tsreserven der Gesch{\"a}ftsbanken sich ver{\"a}ndern.
Die Mindestreservepolitik schafft so einen stabilen zus{\"a}tzlichen Zentralbankgeldbedarf und stellt so die direkte Verbindung zwischen Mindestreserve und Geldsch{\"o}pfung her.
Mindestreserven muss jedes Kreditinstitut im Eurosystem halten, wobei die Mindestreserven nur im Monatsdurchschnitt gehalten werden m{\"u}ssen. Gehaltene verzinste Mindestreserven werden dabei mit dem Zinssatz f{\"u}r das Hauptrefinanzierungsgesch{\"a}ft verzinst.


(S.564)

Geldpolitische Strategien in Europa
::anmerkung:: Geldmenge M3 erkl{\"a}ren

(S.565)
Die geldpolitische Strategie des Eurosystems ist von Rat der Europäischen Zentralbanken entwickelt und am 13.10.1998 der Öffentlichkeit präsentiert worden.
Zentrale Elemente sind seitdem:
\begin{itemize}
    \item{}
    \item{}
    \item{}
\end{itemize}


(S.566)
(S.567)
(S.568)
\end{document}
