\documentclass[
    onecolumn,
    a4paper,
    abstracton,
    parskip=half
    %,draft
    ,final
    ]{scrartcl}

    \usepackage[pdftex
    ,draft
    ]{graphicx}

    \usepackage{booktabs}

    \usepackage[cmex10]{amsmath}

    \interdisplaylinepenalty=2500
    \usepackage{url}

    \usepackage[breaklinks]{hyperref}
    \hyphenation{nothing} % correct bad hyphenation here

    \usepackage{eurosym}

    \usepackage{listings}
    \lstset{basicstyle=\small\ttfamily,breaklines=true}
    \emergencystretch 1000pt

    \usepackage{subcaption}

    \usepackage{mathtools}

    % deutsche Silbentrennung
    \usepackage[ngerman]{babel}

    \usepackage[printonlyused, withpage]{acronym}

    \usepackage[a4paper]{geometry}

    % wegen deutschen Umlauten
    \usepackage[ansinew]{inputenc}

    % fuer Zitate
    \usepackage[round]{natbib}

    \usepackage{setspace}

    \usepackage{units}
    \usepackage{cite}
    %%% Deutsche Verzeichnis-ueberschriften



    %%% Kommentarfunktion %%%
    \usepackage[textwidth=2.2cm
    ,obeyFinal
    ]{todonotes}

\begin{document}

2014, Jahr 7 nach der wohl gr{\"o}{"s}ten Finanzkrise seit dem 2. Weltkrieg. Die Regierungen der gr{\"o}{"s}ten Weltwirtschaftsnationen planen oder f{\"u}hren gro{"s}e Ausgaben zur Bankenrettung oder wieder in Schwung Bringung der Wirtschaften durch, sehr zum Wohlwollen vieler {\"o}konomen. Und doch sei angemerkt, dass es eine nicht geringe Anzahl an kommentaren gab und gibt, die diesen Ma{"s}nahmen kritisch bis ablehnend gegen{\"u}ber stehen. Sie tun es aus verschiedenen Gr{\"u}nden, manche aus praktischen, manche aus analytischen und manche aus ideologischen Gr{\"u}nden.

Der Fokus der Debatte geht zur{\"u}ck auf die von Hayek mit begr{\"u}ndete, sogenannte "`{\"o}stereichische Schule,"' welche nicht selten das Argument mangelnder Effektivit{\"a}t gegen das Eingreifen des Staates in die Wirtschaft geschwungen haben. Von ihrem Standpunkt aus, sollten Individuen ihre eigenen Entscheidungen treffen mit minimaler Beeintr{\"a}chtigung durch den Staat.

Dieser Standpunkt ist derzeit nicht repr{\"a}sentativ f{\"u}r die vorherrschende Hauptstr{\"o}mungen neomonetaristischer und neokeynsianischer Wirtschafts- und Geldpolitik.

Haupts{\"a}chlich handelt es sich dabei um eine These, die mit wenig empirischer Belegbarkeit belastbar ist, allerdings kann diese These helfen die Aufmerksamkeit darauf zu lenken, wo das Eingreifen der Regierungen katastrophal gescheitert ist; jedoch auch ihr Gegenteil: die Beispiele, wo diese Interventionen funktionerten.

\begin{quote}
Ist das alte, liberale Prinzip "`Lasst das Kapital in den H�nden der Einzelnen Fr�chte tragen"' immer noch der richtige Leitsatz,
 oder ist wirklich der Staat kompetenter zu entscheiden, wo und in welcher Form das verf{\"u}gbare Kapital am zweckm{\"a}{"s}igsten zu verwenden ist? \citep[S.23f]{Hayek1969}
\end{quote}

Schaut man sich die Geschichte der Entwicklung der (Finanz-)M�rkte seit dem 16. Jahrhundert an, so kann man wieder und wieder ein, ob nun k�rzlich oder in Vergangenheit eines beobachten: Umregulierte M�rkte boomen und kollabieren schlie�lich. Die aller ersten M�rkte reagierten damals mit erzwungener Standarisierung von Gewichten und Ma�einheiten, eine Regulierung, der sich auch Hayek nicht verschloss.

Ironischerweise pl�dieren gerade die Vertreter einer Nicht-Regulierungs-Sichtweise im Namen des Allgemeinwohls f�r die Maximierung der individuellen Freiheit. 


 















%---------------



 \citep[vgl.][S.23f]{Hayek1969}: "`Jeder Versuch, zwar den laufenden Betrieb vom Wettbewerb bestimmen zu lassen,
 aber die gro{"s}en Entscheidungen {\"u}ber die Vorsorge f{\"u}r die Zukunft einer zentralen Planungsbeh{\"o}rde zu {\"u}bertragen,
muss langfristig zu einer vollst{\"a}ndigen Planwirtschaft f{\"u}hren."'



\end{document}
