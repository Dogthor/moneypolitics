\documentclass[
    onecolumn,
    a4paper,
    abstracton,
    parskip=half
    %,draft
    ,final
    ]{scrartcl}

    \usepackage[pdftex
    ,draft
    ]{graphicx}

    \usepackage{booktabs}

    \usepackage[cmex10]{amsmath}

    \interdisplaylinepenalty=2500
    \usepackage{url}

    \usepackage[breaklinks]{hyperref}
    \hyphenation{nothing} % correct bad hyphenation here

    \usepackage{eurosym}

    \usepackage{listings}
    \lstset{basicstyle=\small\ttfamily,breaklines=true}
    \emergencystretch 1000pt

    \usepackage{subcaption}

    \usepackage{mathtools}

    % deutsche Silbentrennung
    \usepackage[ngerman]{babel}

    \usepackage[printonlyused, withpage]{acronym}

    \usepackage[a4paper]{geometry}

    % wegen deutschen Umlauten
    \usepackage[ansinew]{inputenc}

    % fuer Zitate
    \usepackage[round]{natbib}

    \usepackage{setspace}

    \usepackage{units}
    \usepackage{cite}
    %%% Deutsche Verzeichnis-ueberschriften



    %%% Kommentarfunktion %%%
    \usepackage[textwidth=2.2cm
    ,obeyFinal
    ]{todonotes}

\begin{document}


\citep[vgl.][S.23f]{Hayek1969}: "Ist das alte, liberale Prinzip “Lasst das Kapital in den Händen der einzelnen Früchte tragen” immernoch der richtige Leitsatz,
 oder ist wirklich der Staat kompetenter zu entscheiden, wo und in welcher Form das verfügbare Kapital am zweckmäßigsten zu verwenden ist?"


 \citep[vgl.][S.23f]{Hayek1969}: "Jeder Versuch, zwar den laufenden Betrieb vom Wettbewerb bestimmen zu lassen,
 aber die großen Entscheidungen über die Vorsorge für die Zukunft einer zentralen Planungsbehörde zu übertragen,
muss langfristig zu einer vollständigen Planwirtschaft führen."



\end{document}
