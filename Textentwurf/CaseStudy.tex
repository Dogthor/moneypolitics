\documentclass[
    onecolumn,
    a4paper,
    abstracton,
    parskip=half
    %,draft
    ,final
    ]{scrartcl}

    \usepackage[pdftex
    ,draft
    ]{graphicx}

    \usepackage{booktabs}

    \usepackage[cmex10]{amsmath}

    \interdisplaylinepenalty=2500
    \usepackage{url}

    \usepackage[breaklinks]{hyperref}
    \hyphenation{nothing} % correct bad hyphenation here

    \usepackage{eurosym}

    \usepackage{listings}
    \lstset{basicstyle=\small\ttfamily,breaklines=true}
    \emergencystretch 1000pt

    \usepackage{subcaption}

    \usepackage{mathtools}

    % deutsche Silbentrennung
    \usepackage[ngerman]{babel}

    \usepackage[printonlyused, withpage]{acronym}

    \usepackage[a4paper]{geometry}

    % wegen deutschen Umlauten
    \usepackage[ansinew]{inputenc}

    % fuer Zitate
    \usepackage[round]{natbib}

    \usepackage{setspace}

    \usepackage{units}
    \usepackage{cite}
    %%% Deutsche Verzeichnis-ueberschriften



    %%% Kommentarfunktion %%%
    \usepackage[textwidth=2.2cm
    ,obeyFinal
    ]{todonotes}

\begin{document}

2014, Jahr 7 nach der wohl gr{\"o}{"s}ten Finanzkrise seit dem 2. Weltkrieg. Die Regierungen der gr{\"o}{"s}ten Weltwirtschaftsnationen planen oder f{\"u}hren gro{"s}e Ausgaben zur Bankenrettung oder wieder in Schwung Bringung der Wirtschaften durch, sehr zum Wohlwollen vieler {\"o}konomen. Und doch sei angemerkt, da{"s} es eine nicht geringe Anzahl an kommentaren gab und gibt, die diesen Ma{"s}nahmen kritisch bis ablehnend gegen{\"u}ber stehen. Sie tun es aus verschiedenen Gr{\"u}nden, manche aus praktischen, manche aus analytischen und manche aus ideologischen Gr{\"u}nden.

Der Fokus der Debatte geht zur{\"u}ck auf die von Hayek mit begr{\"u}ndete, sogenannte "`{\"o}stereichische Schule,"' welche nicht selten das Argument mangelnder Effektivit{\"a}t gegen das Eingreifen des Staates in die Wirtschaft geschwungen haben. Von ihrem Standpunkt aus, sollten Individuen ihre eigenen Entscheidungen treffen mit minimaler Beeintr{\"a}chtigung durch den Staat.

Dieser Standpunkt ist derzeit nicht repr{\"a}sentativ f{\"u}r die vorherrschende Hauptstr{\"o}mungen neomonetaristischer und neokeynsianischer Wirtschafts- und Geldpolitik.

Haupts{\"a}chlich handelt es sich dabei um eine These, die mit wenig empirischer Belegbarkeit belastbar ist, allerdings kann diese These helfen die Aufmerksamkeit darauf zu lenken, wo das Eingreifen der Regierungen katastrophal gescheitert ist; jedoch auch ihr Gegenteil: die Beispiele, wo diese Interventionen funktionerten.

\begin{quote}
Ist das alte, liberale Prinzip "`La{"s}t das Kapital in den H{\"a}nden der Einzelnen Fr{\"u}chte tragen"' immer noch der richtige Leitsatz,
 oder ist wirklich der Staat kompetenter zu entscheiden, wo und in welcher Form das verf{\"u}gbare Kapital am zweckm{\"a}{"s}igsten zu verwenden ist? \citep[S.23f]{Hayek1969}
\end{quote}

Schaut man sich die Geschichte der Entwicklung der (Finanz-)M{\"a}rkte seit dem 16. Jahrhundert an, so kann man wieder und wieder ein, ob nun k{\"u}rzlich oder in Vergangenheit eines beobachten: Umregulierte M{\"a}rkte boomen und kollabieren schlie{"s}lich. Die aller ersten M{\"a}rkte reagierten damals mit erzwungener Standarisierung von Gewichten und Ma{"s}einheiten, eine Regulierung, der sich auch Hayek nicht verschlo{"s}.

Ironischerweise pl{\"a}dieren gerade die Vertreter einer Nicht-Regulierungs-Sichtweise im Namen des Allgemeinwohls f{\"u}r die Maximierung der individuellen Freiheit. Die Kosten des Individuellen Scheiterns werden jedoch nicht berechnet. 

Zwei Hauptargumente lassen sich gegen ein nicht-Eingreifen in den Markt zu Felde f{\"u}hren; zum einen existiert bereits ein Ma� an Eingriffen, weit {\"u}ber das von den {\"o}sterreichern vorgeschlagene Mittel hinausgehend, in Form von lang zeitlich etablierten industriellen Normen. Gleichzeitig sehen wir {\"u}ber die letzten 3 Jahrzehnte einen beispiellosen Anstieg gesellschaftlichen und individuellen Wohlstands. Und dar{\"u}ber hinaus, sei erg{\"a}nzend festgestellt, dass die Vorteile des gesamtgesellschaftlichen Wohlstandanstiegs deutlich mehr geteilt werden, als jemals zuvor. Vereinfacht gesagt: Das derzeitige Modell angemessener Eingriffe funktioniert.

Zum anderen scheinen Verfechter einer Nicht-Regulierungs-Sichtweise davon {\"u}berzeugt zu sein, dass individuelle Entscheidungen in der Regel von Ignoranz, Falschinformation oder betr{\"u}gerischer Absicht beeinflusst sind. //Dies nimmt Hayek als Begr{\"u}ndung, dass der Staat nicht handeln d{\"u}rfe //
Ja, Individuen sind in der Lage, Entscheidungen zu treffen, die ihnen selbst, oder anderen Schaden zuf{\"u}gen k{\"o}nnen. Augenscheinlich ist jedoch, dass wir vermeintlich von Fehlern lernen k{\"o}nnen und wir f{\"u}r das n{\"a}chste mal bessere Entscheidungen treffen k{\"o}nnen. Und ja, manche Entscheidungen sind katastrophal, wie jene, die die Umwelt betreffen. Und: es gibt Fehler, von deren Wiedergutmachung unm{\"o}glich ist.

Selbstverst{\"a}ndlich k{\"o}nnen Regulierungen durch Regierungen taktisch unklug, ja, sogar t{\"o}richt sein. Auch sind desastr{\"o}se Resultate m{\"o}glich. Eingriffe k{\"o}nnen korrumpiert sein, oder schlimmer: strategischen Moral Hazard belohnen. Oder gelegentlich schlicht und einfach nicht funktionieren. Doch k{\"o}nnen diese Einw{\"a}nde nicht als generelles Argument gegen Regulierungen gelten. Jedoch ist es ein Argument daf{\"u}r, dass jede Regulierung mit bedacht ausgew{\"a}hlt und sehr sorgf{\"a}ltig geplant werden muss, sowie, dass aus Erfahrungen gelernt werden muss. Die ist, wie auch der Realit{\"a}t mit die man meistern muss, ein vertracktes unterfangen. Nicht-Regulierungs-Bef{\"u}rworter scheinen eine simplere Antwort, das "Nein" zu w{\"a}hlen, eine magische Beschw{\"o}rungsformel um die Realit{\"a}t weniger real werden zu lassen.

Wieder und wieder, Streitpunkt um Streitpunkt, so n{\"a}hert man sich Antwort um Antwort den Fragestellungen unserer Hochkomplexen Welt.

Schauen wir uns die j{\"u}ngere Vergangenheit an, so ist r{\"u}ckblickend die mehrheitliche {\"o}ffentliche Meinung dahingehend best{\"a}tigt worden, dass Eingriffe im gro�en Ma�stab in das Bankensystem sinnvoll waren und sind, nachdem dieses stark beeintr{\"a}chtigt war. Es w{\"a}re katastrophal gewesen, einfach darauf zu warten, dass der Bankensystem sich wieder rekonstituirt und h{\"a}tte noch gr{\"o}�eren gesellschaftlichen Schaden angerichtet, als durch die Finanzkrise schon so angerichtet wurde. Dar{\"u}ber hinaus: Wenn Konsumenten und Unternehmen ihre Ausgaben senken, m{\"u}ssen die Regierungen sich gegen den Trend stemmen und seine Ausgaben erh{\"o}hen. Auf die grundlegende Richtigkeit dieser Logik wies schon Keynes hin.

Der wissenschaftliche Einspruch gegen die Erh{\"o}hung in Deutschland ist vor allem, dass diese Erh{\"o}hung der Ausgaben Unternehmensausgaben verdr{\"a}ngen. Doch es kann nichts verdr{\"a}ngt werden, was bereits weggefallen ist. Selbstverst{\"a}ndlich wird diese Verdr{\"a}ngung von Unternehmensausgaben durch den Staat eine ernsthafte Bedrohung, sobald die Wirtschaft sich erneuert. Und ein inflation{\"a}rer Druck steht als zuk{\"u}nftige Gefahr am Himmel. Doch hei�t dies blo�, dass diese stimulierenden Eingriffe schnell zur{\"u}ck gefahren werden m{\"u}ssen, wenn die Umst{\"a}nde, die zu ihrer Einsetzung f{\"u}hrten, sich ver{\"a}ndern.

Auf Grund der Eile und Gr{\"o}�e der in die Wege geleiteten Ausgaben (begleitet von anderen finanziellen Hilfen f{\"u}r Gesch{\"a}fts- und Kreditbanken), war von vornherein klar: Geld wird verschwendet, manches auch ohne Einsatzzweck verbrannt. Doch das war scheinbar der Preis, den man f{\"u}r die "Kernschmelze des Finanzsektors" bereit war zu zahlen. 

Die Analogie mag vielleicht {\"u}berstrapaziert sein, doch sie trifft immernoch: Wenn das Haus in Flammen steht, sch{\"u}tte Wasser {\"u}berall hin, um die Flammen auszul{\"o}schen. Danach beginnt der schwierige Prozess: Wir m{\"u}ssen den Keller auspumpen und Bausch{\"a}den der Konstruktion reparieren. 


%---------------


 \citep[vgl.][S.23f]{Hayek1969}: "`Jeder Versuch, zwar den laufenden Betrieb vom Wettbewerb bestimmen zu la{"s}en,
 aber die gro{"s}en Entscheidungen {\"u}ber die Vorsorge f{\"u}r die Zukunft einer zentralen Planungsbeh{\"o}rde zu {\"u}bertragen,
mu{"s} langfristig zu einer vollst{\"a}ndigen Planwirtschaft f{\"u}hren."'



\end{document}
