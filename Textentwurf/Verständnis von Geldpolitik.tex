\documentclass[
      onecolumn,
      a4paper,
      abstracton,
      parskip=half
      %,draft
      ,final
      ]{scrartcl}

      \usepackage[pdftex
      ,draft
      ]{graphicx}

      \usepackage{booktabs}

      \usepackage[cmex10]{amsmath}

      \interdisplaylinepenalty=2500
      \usepackage{url}

      \usepackage[breaklinks]{hyperref}
      \hyphenation{nothing} % correct bad hyphenation here

      \usepackage{eurosym}

      \usepackage{listings}
      \lstset{basicstyle=\small\ttfamily,breaklines=true}
      \emergencystretch 1000pt

      \usepackage{subcaption}

      \usepackage{mathtools}

      % deutsche Silbentrennung
      \usepackage[ngerman]{babel}

      \usepackage[printonlyused, withpage]{acronym}

      \usepackage[a4paper]{geometry}

      % wegen deutschen Umlauten
      \usepackage[ansinew]{inputenc}

      % fuer Zitate
      \usepackage[round]{natbib}

      \usepackage{setspace}

      \usepackage{units}
      \usepackage{cite}
      %%% Deutsche Verzeichnis-ueberschriften



      %%% Kommentarfunktion %%%
      \usepackage[textwidth=2.2cm
      ,obeyFinal
      ]{todonotes}
\begin{document}

\subsection{Beschreibung des eigenen Verst{\"a}ndnis von Geldpolitik}

Nach Basseler hat die Geldpolitik die Hauptaufgabe, eine optimale Geldversorgung der Wirtschaft zu gew{\"a}hrleisten \citep[Vgl.][S. 551]{Basseler2010}.
Real wird diese Aufgabe von einer gr{\"o}{"ss}tenteils staatlichen, aber unabh{\"a}ngigen Zentralbank {\"u}bernommen. Dabei herrscht weit verbreitet Konsens dar{\"u}ber, dass Geldpolitik staatliche Aufgabe bleibt, auch wenn Ideen einer dezentralen, dem Wettbewerb unterliegenden Geldversorgung durch private Gesch{\"a}ftsbanken und ein System konkurierender Parallelw{\"a}hrungen kursieren. (Hayek)
"`Zentrale Zielgr{\"o}{"ss}e der Geldpolitik ist die Geldmenge M3. (...) Dabei ist zu beachten, dass das herk{\"o}mmliche Konzept von Banken zum Konzept der Mont{\"a}ren Finanzinstitute (MFIs) erweitert worden ist."`
\citep[vgl.][S. 507]{Basseler2010}

\citep[vgl.][S.508]{Basseler2010} "`MFIs sind also im Wesentlichen:
- Zentralbanken,
- Kreditinstitute und
- Geldmarktfonds.
(...) Innerhalb der Geldmenge M3 spielen Bargeldumlauf und t{\"a}glich f{\"a}llige Einlagen die gr{\"o}{"ss}te Rolle; Einlagen mit vereinbarterter K{\"u}ndigungsfrist (...) sind ebenfalls quantitativ bedeutsam; die {\"u}brigen Komponenten machen insgesamt nur knapp 20 Prozent der Geldmenge M3 aus."`


\subsubsection{ Akteure des Finanzbereiches}

\citep[vgl.][S.511f]{Basseler2010} "`Die Akteure im Finanzbereich werden allgemein Finanzintermedi{\"a}re genannt. [Diese] vermitteln Finanzprodukte zwischen den Anbietern und Nachfragern. (...) Dies sind vor allem Banken und Kapitalanlagegesellschaften, die selbst Finanzprodukte kreieren sowie institutionelle Anleger."`

 "`Es muss im Finanzbereich eine staatliche Institition geben, eine staatlich organisierte Zentralbank, die folgende Aufgaben erf{\"u}llt:
 \begin{itemize}
\item{die Ausgabe der gesetzlichen Zahlungsmittel (Staatliches Emissionsmonopol),}
\item{die Durchf{\"u}hrung einer Geldpolitik mit dem Ziel einer angemessenen Begrenzung der Geldmenge,}
\item{die Organisation eines reibungslosen Zahlungs- und Kreditverkehrs als >>Bank der Banken<<,}
\item{die Wahrung der Geldwertstabilit{\"a}t,}
\item{die Bereitstellung einer ausreichenden Menge an Geld in Krisenzeiten"'}
\end{itemize}



## 16.4.2. Gesch{\"a}ftsbanken (Kreditinstitute)
\citep[vgl.][S.512-13]{Basseler2010} "`Gesch{\"a}ftsbanken (oder auch Kreditinstitute) sind die zentralen Akteure im Finanzbereich einer Volkswirtschaft. Die erste zentrale Funktion von Gesch{\"a}ftsbanken (kurz: Banken) ist die Abwicklung des **Zahlungsverkehrs** einer Volkswirtschaft. (...) Die zweite zentrale Funktion von Banken ist die Organisation und Durchf{\"u}hrung des **Kreditverkehrs** einer Volkswirtschaft. "`
"`(...)die Organisation des Kreditverkehrs ist ein klassisches Gesch{\"a}ft der banken: Sie beschaffen Geld, sie verleihen Geld und sie versuchen, dieses Geld mit Gewinn wieder zur{\"u}ckzubekommen.(...)Die Kreditgew{\"a}hrung war neben der Abwicklung des Zahlungsverkehrs die klassische Aufgabe der Gesch{\"a}ftsbank."`
 "`Daneben gibt es weitere Aufgaben der Banken, vor allem die Verm{\"o}gensverwaltung der Kunden, die Ausgabe und den Handel mit Wertpapieren, die Beratung und Unterst{\"u}tzung bei Unternehmenszusammenschl{\"u}ssen oder die Unterst{\"u}tzung von Unternehmen bei ihrer Kapitalaufnahme, etwa bei B{\"o}rseng{\"a}ngen. Dies wird zusammenfassend **Investmentbanking** bezeichnet. (...) In diesem Segment des Bankengesch{\"a}fts  werden auf Zertifikate entwickelt und verkauft oder Fonds aufgelegt und Fondsanteile verkauft. (...)In Kontinentaleuropa ist (...)das Universalbankensystem etabliert(...). Sparkassen {\"u}bernehmen (...)die Abwicklung des Zahlungsverkehrs und des >>kleinen<< Kreditverkehrs, kleine Privatbanken {\"u}bernehmen eher die Funktionen des Investmentbankings und gro{"ss}e Universalbanken {\"u}bernehmen alle Gesch{\"a}ftssparten."`

\citep[vgl.][S.515]{Basseler2010} "`Das Eigenkapital [einer Bank] setzt sich konkret zusammen aus dem Grundkapital, den Kapitalr{\"u}cklagen und den Gewinnr{\"u}cklagen (einbehaltene Gewinne) sowie einer stillen Einlage des Finanzmarktstabilisierungsfonds (...). Grunds{\"a}tzlich ist das Eigenkapital der Banken von zentraler Bedeutung. Es ist letztlich das Kapital, das die Bank zum Ausgleich von Verlusten aus ihrem Kredit- und Investmentgesch{\"a}ft einsetzen kann.  Daher sind in der Bankenaufsicht bestimmte Mindestanforderungen an die H{\"o}he des haftenden Eigenkapitals (...)vorgesehen"`

\citep[vgl.][S.512]{Basseler2010}  "`Mit dem (...) 01.01.1999 ist (...)die Europ{\"a}ische Zentralbank (EZB) die zentrale Institution - also die Zentralbank - f{\"u}r die Festlegung und Ausf{\"u}hrung der Geldpolik. Daneben existiert weiterhin die Deutsche Bundesbank, die als Zentralbank der Bundesrepublik Deutschland Teil des Europ{\"a}ischen Systems der Zentralbanken ist. Sie ist (...) ausf{\"u}hrendes organ der geldpolitischen Entscheidungen der (...)EZB. "`



\end{document}
