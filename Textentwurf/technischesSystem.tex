\documentclass[
    onecolumn,
    a4paper,
    abstracton,
    parskip=half
    %,draft
    ,final
    ]{scrartcl}

    \usepackage[pdftex
    ,draft
    ]{graphicx}

    \usepackage{booktabs}

    \usepackage[cmex10]{amsmath}

    \interdisplaylinepenalty=2500
    \usepackage{url}

    \usepackage[breaklinks]{hyperref}
    \hyphenation{nothing} % correct bad hyphenation here

    \usepackage{eurosym}

    \usepackage{listings}
    \lstset{basicstyle=\small\ttfamily,breaklines=true}
    \emergencystretch 1000pt

    \usepackage{subcaption}

    \usepackage{mathtools}

    % deutsche Silbentrennung
    \usepackage[ngerman]{babel}

    \usepackage[printonlyused, withpage]{acronym}

    \usepackage[a4paper]{geometry}

    % wegen deutschen Umlauten
    \usepackage[ansinew]{inputenc}

    % fuer Zitate
    \usepackage[round]{natbib}

    \usepackage{setspace}

    \usepackage{units}
    \usepackage{cite}
    %%% Deutsche Verzeichnis-ueberschriften



    %%% Kommentarfunktion %%%
    \usepackage[textwidth=2.2cm
    ,obeyFinal
    ]{todonotes}

\begin{document}

\section{Technisches System}
  \label{sec2:technischesSystem}



\subsection{Organisationen der Europ{\"a}ischen Geldpolitk}
%%Autor:Gregor May %%

In der Europ{\"a}ischen Union (EU) wird die Geldpolitik durch die Europ{\"a}ische Zentralbank (EZB) und das Europ{\"a}ische System der Zentralbanken (ESZB) organisiert. Dabei umfasst das ESZB alle 28 nationalen Zentralbanken (NZBen) der Mitgliedsstaaten der EU sowie die Europ{\"a}ische Zentralbank. Zuletzt wurden Estland und Lettland in die Eurozone aufgenommen.\footnote[98]{Die Eurozone besteht derzeit aus 18 EU-Staaten und hat daher auch den Beinamen "`Euro-18"' erhalten.} Sonderstatus im ESZB haben dabei die sogenannten "`Outs"', jene Mitgliedsstaaten der EU, die den Euro nicht eingef{\"u}hrt haben.
Dies umfasst derzeit: D{\"a}nemark, Gro{"s}britanien, Schweden sowie die meisten neuen EU-Mitgliedsstaaten nach 2001. Sie sind vom Entscheidungsprozess der ESZB ausgeschlossen und vollziehen eine eigenst{\"a}ndige nationale Geldpolitik.
Formal unterscheidet der EG-Vertrag\footnote[25]{EGV, Art. 105-109 d} zwischen EZB und ESZB, faktisch entscheidet jedoch nur eine Institution: die EZB mit ihren Beschlussorganen\footnote[99]{\citep[vgl.][S.553]{Basseler2010}}
%%Umstellen%%Mit Organisation von Geldarten und Geldsch{\"o}pfung durch die EZB und das ESZB sind diese bei einer supranationalen Institution innerhalb des Eurosystems monopolisiert. %%Umstellen%%

\subsection{Die Europ{\"a}ische Zentralbank}
 %% Autor: Gregor May %%
Als Beschlussorgane leiten der EZB-Rat und das Direktorium die Europ{\"a}ischen Zentralbank. 
%%Umstellen%%Das Direktorium der EZB setzt sich aus dem Pr{\"a}sidenten und dem Vizepr{\"a}sidenten der EZB, sowie weiteren vier Mitgliedern zusammen, die von den Regierungen der Mitgliedsstaaten auf Ebene von Staats- und Regierungschefs einvernehmlich ernannt werden. Dem EU-Rat steht hierbei ein Empfehlungsrecht zu. %%Umstellen%%
Erweitert besteht beim Europ{\"a}ischem Parlament und beim EZB-Rat ein Anh{\"o}rungsrecht.\footnote[46]{\citep[vgl.][S.553]{Basseler2010}}

Der EZB-Rat besteht aus dem Direktorium und den Pr{\"a}sidenten aller nationalen Zentralbanken, die den Euro gemeinsam eingef{\"u}hrt haben. Innerhalb der Europ{\"a}ischen Zentralbank liegt die exekutive Gewalt beim Direktorium, welches "`f{\"u}r die Durchf{\"u}hrung der Geldpolitik nach den Leitlinien und Beschl{\"u}ssen des EZB-Rates verantwortlich ist.\footnote[47]{\citep[S.553]{Basseler2010}}"'

Das Direktorium der EZB ist gegen{\"u}ber den nationalen Zentralbanken des Eurosystems weisungsbefugt. Somit entsteht ein duales System, bestehend aus dem Exekutivorgan der EZB in Form des Direktoriums und ein Beschlussorgan in Form des EZB-Rates.

Beschl{\"u}sse der gemeinschaftlichen europ{\"a}ischen Geldpolitik des Euroraumes werden vom EZB-Rat erarbeitet und erlassen um die Ausgabe von M{\"u}nzen und Banknoten zu regeln oder um die vom ESZB {\"u}bertragenen Aufgaben zu erf{\"u}llen.\footnote[100]{\citep[vgl.][S.553]{Basseler2010}}
Derzeit umfasst der EZB-Rat 18 L{\"a}nder und das Direktorium. Abgestimmt wird mit einfacher Mehrheit der Anwesenden, bei Stimmengleichheit entscheidet die Stimme des Pr{\"a}sidenten.


\subsection{Ziele und Aufgaben von ESZB und EZB}  %% Autor: Gregor May %%
Die EG-Vertr{\"a}ge legen als vorrangiges Ziel des ESZB und damit der EZB "`die Gew{\"a}hrleistung der Preisstabilit{\"a}t(\ldots). (\ldots)Soweit dies ohne Beeintr{\"a}chtigung des Zieles der Preisstabilit{\"a}t m{\"o}glich ist, unterst{\"u}tzt das ESZB die allgemeine Wirtschaftspolitik der Gemeinschaft, um die Verwirklichung der in Artikel 2 festgelegten Ziele der Gemeinschaft beizutragen. Das ESZB handelt im Einklang mit dem Grundsatz einer offenen Marktwirtschaft und mit freien Wettbewerb \ldots"'\footnote[48]{\citep[vgl.][S.554]{Basseler2010}}%%Umstellen%%

Hier wird von der EZB eine Priorisierung der Preisstabilit{\"a}t gegen{\"u}ber anderen Zielen wie Vollbesch{\"a}ftigung und Wachstum festgeschrieben.

Ideologische Grundlage f{\"u}r diese Priorisierung ist die von monetaristischen Str{\"o}mungen ausgehende Vorstellung, dass eine Zentralbank zu allererst die Verantwortung der Preisstabilit{\"a}tssicherung besitzt. Andere Akteure, wie Gewerkschaften und Tarifparteien, sind f{\"u}r die Vollbesch{\"a}ftigung verantwortlich. Wirtschaftswachstum ergibt sich aus technischem Fortschritt und Bev{\"o}lkerungswachstum.\footnote[50]{ebd.}

Die Aufgaben des ESZB werden in Art. 105, Abs. 2 EGV wie folgt festgelegt:
\begin{itemize}
 \item{die Geldpolitik der Gemeinschaft festzulegen und umsetzen,} 
  \item{Divisengesch{\"a}fte im Einklang mit Artikel 111 durchzuf{\"u}hren,}
 \item{die offiziellen W{\"a}hrungsreserven der Mitgliedsstaaten zu halten und zu verwalten,}
 \item{das reibungslose Funktionieren des Zahlungssystems zu f{\"o}rdern}\footnote[50]{\citep[vgl.][S.555]{Basseler2010}}
\end{itemize}

%%Umstellen%% 
Die Aufgaben des ESZB werden in Art. 105, Abs. 2 EGV wie folgt festgelegt: die Geldpolitik der Gemeinschaft festzulegen und umsetzen, Divisengesch{\"a}fte im Einklang mit Artikel 111\footnote[1999]{Dieser Artikel legt fest, dass alle Entscheidungen {\"u}ber Wechselkurssysteme dem Ministerrat vorbehalten sind und dementsprechend nicht durch die EZB gelenkt werden kann.} durchzuf{\"u}hren, die offiziellen W{\"a}hrungsreserven der Mitgliedstaaten zu halten und zu verwalten, das reibungslose Funktionieren des Zahlungssystems zu f{\"o}rdern\footnote[50]{\citep[vgl.][S.555]{Basseler2010}}

Dieser Artikel legt fest, dass alle Entscheidungen {\"u}ber die Wechselkurssysteme dem Ministerrat vorbehalten sind.

%%Umstellen%%  Nicht mehr als Liste, sondern als Satz. oder Folgende Auflistung:
Die Aufgaben des ESZB werden in Art. 105, Abs. 2 EGV wie folgt festgelegt:
\begin{itemize}
 \item{Festzulegung und Umsetzung der gemeinschaftlichen Geldpolitik} 
  \item{Divisengesch{\"a}fte im Einklang mit Artikel 111\footnote[1999]{Dieser Artikel legt fest, dass alle Entscheidungen {\"u}ber Wechselkurssysteme dem Ministerrat vorbehalten sind und dementsprechend nicht durch die EZB gelenkt werden kann.} durchzuf{\"u}hren}
 \item{Haltung und Verwaltung der offiziellen W{\"a}hrungsreserven der Mitgliedstaaten}
 \item{F{\"o}rderung der reibungslose Funktion des Zahlungssystems.}\footnote[50]{\citep[vgl.][S.555]{Basseler2010}}
\end{itemize}
%%Umstellen%%

\subsubsection{Unabh{\"a}ngigkeit der EZB}  %% Autor: Gregor May %%

Gew{\"a}hrleistung der Preisstabilit{\"a}t ist als vorrangiges Ziel der EZB festgeschrieben. Da die EZB eine gesamteurop{\"a}ische Geldpolitik organisieren soll, steht eine Unabh{\"a}ngigkeit von nationalen Interessen einzelner Regierungen als unabdingbar im Raum. Einfachheitshalber sei festgestellt, dass die EZB auf verschiedene Arten als unabh{\"a}ngig gelten darf.\footnote[51]{\citep[vgl.][S.555-557]{Basseler2010}}

Zum Ersten ist die EZB funktional relativ unabh{\"a}ngig, da sie Weisungen nicht entgegennehmen darf.\footnote[34]{vgl. Artikel 107, EGV (Unabh{\"a}ngigkeit der EZB)}

Die Unabh{\"a}ngigkeit, keinerlei Kontrollen durch Regierungen und Parlamente zu unterliegen, ist im europ{\"a}ischen Raum einzigartig. Sie wird nur durch die Verpflichtung beschr{\"a}nkt eine allgemeine Wirtschaftspolitik der Gemeinschaft zu unterst{\"u}tzen, soferndas Ziel der Preisstabilit{\"a}t nicht beeintr{\"a}chtigt wird.

Zum Zweiten ist die EZB dar{\"u}ber hinaus personell unabh{\"a}ngig, allein durch die Ernennung von Pr{\"a}sidenten der Nationalbanken k{\"o}nnen einzelne Regierungen Einfluss aus{\"u}ben.

Zum Dritten ist die EZB finanziell unabh{\"a}ngig --- sie verf{\"u}gt {\"u}ber eigene Einnahmen und einen eigenen Haushalt --- und besitzt die Kontrolle {\"u}ber alle Instrumente der Geldpolitik.

\subsection{Instrumente der Europ{\"a}ischen Geldpolitik}  %% Autor: Gregor May %%

Im weiteren Verlauf wollen wir kurz die geldpolitischen Instrumente erl{\"a}ufern, die der Europ{\"a}ischen Zentralbank zur Verf{\"u}gung stehen.



\subsubsection{Offenmarktpolitik der EZB}  %% Autor: Gregor May %%

Zentrales Instrument f{\"u}r die Geldversorgung einer Wirtschaft ist die Offenmarktpolitik. Man versteht hierunter den An- und Verkauf von Wertpapieren gegen Zentralbankgeld durch die Zentralbank.
Die Zentralbank kann durch Offenmarktk{\"a}ufe bzw. -verk{\"a}ufe die Zentralbankgeldsch{\"o}pfung steuern. 

F{\"u}r die EZB im ESZB sind nur finanziell solide MFIs, sprich Finanzinstitute, die in das Mindestreservesystem einbezogen sind, als Gesch{\"a}ftspartner der EZB zugelassen. Es wird allgemein zwischen der expansiven (Kauf von Wertpapieren) und kontraktiven Offenmarktpolitik (Verkauf von Wertpapieren) unterschieden.\footnote[54]{\citep[S.556]{Basseler2010}}

Nach einem Kauf von Wertpapieren hat sich der Bestand an Zentralbankgeld der Gesch{\"a}ftsbanken erh{\"o}ht. Diese erh{\"o}hte Geldbasis erm{\"o}glicht nun den Gesch{\"a}ftsbanken Kredite an Nichtbanken weiterzugeben.\footnote[55]{\citep[S.557]{Basseler2010}}


Die Zentralbank kann Gesch{\"a}ftsbanken nicht zur Offenmarktpolitik zwingen, sondern muss entsprechende Anreize bieten. Bei expansiver Offenmarktpolitik bestehen diese in niedrigen Zinss{\"a}tzen f{\"u}r die Zentralbankgeldkreditgew{\"a}hrung unterhalb des Geldmarktzinses, bei kontraktiver Offenmarktpolitik in Zinss{\"a}tzen {\"u}ber dem Geldmarktzins.
Zeitlich befristete Offenmarktgesch{\"a}fte werden auch Wertpapierpensionsgesch{\"a}fte oder Repo-Gesch{\"a}fte genannt\footnote[36]{Repo-Gesch{\"a}fte leitet sich vom englischen "`Repurchase"', zu deutsch R{\"u}ckkauf, ab}. Der Zinssatz wird hier Pensionssatz genannt\footnote[37]{Entsprechend hei{"s}t der Pensionssatz auch Repo-Rate}.
Durch Repo-Gesch{\"a}ften l{\"a}sst sich die Zentralbankgeldmenge durch die Zentralbank recht einfach steuern. Wenn zeitlich begrenzte Gesch{\"a}fte auslaufen und nicht erneuert werden, treten automatisch kontraktive Effekte auf.


Das Bankensystem im Eurogebiet ist auf die Bereitstellung von Zentralbankgeld durch die EZB angewiesen, weshalb die Offenmarktpolitik der EZB am Interbankengeldmarkt\footnote[53]{\citep[vgl.][S.558f]{Basseler2010}} ansetzt. Dementsprechend greift die EZB auf folgende Instrumente zur Steuerung der Offenmarktpolitik zur{\"u}ck:
\begin{enumerate}
 \item{Hauptrefinanzierungsinstrument}
 \item{l{\"a}ngerfristige Refinanzierungsgesch{\"a}fte}
 \item{Feinsteuerungsoption}
 \item{Strukturelle Operationen}
 \end{enumerate}



Das Hauptrefinanzierungsinstrument und das l{\"a}ngerfristige Refinanzierungsgesch{\"a}ft sind regelm{\"a}{"s}ig stattfindende, dem Interbankenmarkt Liquidit{\"a}t zuf{\"u}hrende Repo-Gesch{\"a}fte.
F{\"u}r das Hauptrefinanzierungsinstrument sind die w{\"o}chentlichen Transaktionen auf eine Woche G{\"u}ltigkeit begrenzt. Das Hauptrefinanzierungsinstrument steuert die Zinss{\"a}tze (Zentraler Leitzins der EZB) sowie die Liquidit{\"a}t am europ{\"a}ischen Geldmarkt. Die Refinanzierungsgesch{\"a}fte finden mit monatlichem Abstand statt und haben  Laufzeiten von mehreren Monaten bis hin zu einem Jahr.

Sollte es zu unerwarteten marktm{\"a}{"s}igen Liquidit{\"a}tsschwankungen auf die Zinss{\"a}tze kommen, stehen immer noch die Feinsteuerungsoperationen zur Verf{\"u}gung. Diese werden entweder mit befristeten Transaktionen oder in Form von definitiven  K{\"a}ufen oder Verk{\"a}ufen von Wertpapieren ausgef{\"u}hrt.

Der EZB stehen als letztes Instrument der Offenmarktpolitik die Strukturellen Operationen zur Verf{\"u}gung. Sie haben das Ziel die grundlegenden Liquidit{\"a}tspositionen des Finanzsektors zu beeinflussen. Dies passiert {\"u}ber die Emission von Schuldverschreibungen, definitive K{\"a}ufen oder Verk{\"a}ufe und befristete Transaktionen.\footnote[58]{\citep[S.560]{Basseler2010}}

\subsubsection{St{\"a}ndige Fazilit{\"a}ten}  %% Autor: Gregor May %%

Will die EZB eine genaue Steuerung des Geldmarktzinssatzes, so greift sie auf das Instrument der st{\"a}ndigen Fazilit{\"a}t zur{\"u}ck. Hierunter versteht man die Bereitstellung bzw. Absch{\"o}pfung von Liquidit{\"a}t jeweils bis zum n{\"a}chsten Gesch{\"a}ftstag in Form von Tageskrediten oder t{\"a}glichen Anlagen {\"u}bersch{\"u}ssiger Liqudit{\"a}ten. Im Unterschied zur Offenmarktpolitik erfolgt die Inanspruchnahme der st{\"a}ndigen Fazilit{\"a}ten auf Initiative der Banken und ist grunds{\"a}tzlich unbeschr{\"a}nkt m{\"o}glich. Bislang werden sie jedoch nur in geringem Umfang genutzt, da die Konditionen im Vergleich zu denen am Interbankenmarkt ung{\"u}nstiger sind.\footnote[59]{\citep[vgl.][S.560ff]{Basseler2010}}

Gesch{\"a}ftsbanken k{\"o}nnen zur Deckung eines vor{\"u}bergehenden Liquidit{\"a}tsbedarfs unbegrenzt die Spitzenrefinanzierungsfazilit{\"a}t in Anspruch nehmen, so sie von von der EZB als Gesch{\"a}ftspartner zugelassen sind. Dieser Bedarf wird mit einem im Voraus bekanntgegebenen Kreditzinssatz verzinst, welcher die Obergrenze des allgemeinen Tagesgeldsatzes am Geldmarkt ist.

Gesch{\"a}ftsbanken mit {\"u}bersch{\"u}ssiger Liquidit{\"a}t  k{\"o}nnen auf Einlagefazilit{\"a}ten zur{\"u}ckgreifen und die Liquidit{\"a}t bis zum n{\"a}chsten Gesch{\"a}ftstag bei den nationalen Zentralbanken anlegen. Diese Einlage wird mit einem im Voraus bekanntgegebenen Zinssatz verzinst, welcher die Untergrenze des allgemeinen Tagesgeldsatzes am Geldmarkt ist.

Gew{\"o}hnlicherweise bewegt sich der Leitzins der EZB zwischen dem Zinssatz f{\"u}r die Einlagefazilit{\"a}t und der Spitzenrefinanzierungsfazilit{\"a}t. Ein Zinssatz von weniger als 1 Prozent f{\"u}r die Hauptrefinanzierung stellt einen bisher unerreichten historischen Tiefpunkt f{\"u}r die EZB dar.
Dass die EZB Wert darauf legt, den Geldmarktzinssatz genau zu steuern, zeigt, dass auch keynesianische Elemente in die Geldpolitik der EZB einflie{"s}en.\footnote[59]{\citep[vgl.][S.562f]{Basseler2010}}

\subsubsection{Mindestreservepolitik}   %% Autor: Gregor May %%
In der Europ{\"a}ischen Geldpolitik ist den Gesch{\"a}ftsbanken vorgeschrieben bei der Zentralbank einen bestimmten Prozentsatz ihrer Einlagen, den Mindestreservesatz, als Sichtguthaben vor zu halten. Die Mindestreservepolitik\citep[vgl.][S.562f]{Basseler2010} soll einen stark wirkenden Einfluss auf die Geldsch{\"o}pfung bzw. das Geldsch{\"o}pfungspotential der EZB erhalten. Gibt die EZB eine Erh{\"o}hung des Mindestreservesatz vor, nimmt das Geldsch{\"o}pfungspotential zu, bei Senkung nimmt es ab. 
Ein gewollter Nebeneffekt ist, dass bei einer {\"A}nderung des Mindestreservesatzes auch die Liqudit{\"a}tsreserven der Gesch{\"a}ftsbanken sich ver{\"a}ndern.
Die Mindestreservepolitik schafft so einen stabilen zus{\"a}tzlichen Zentralbankgeldbedarf. Dies stellt die direkte Verbindung zwischen Mindestreserve und Geldsch{\"o}pfung her.

Mindestreserven m�ssen durch die Kreditinstitute nur im Monatsdurchschnitt gehalten werden. Der Zinssatz f{\"u}r das Hauptrefinanzierungsgesch{\"a}ft wird f�r angelegte Mindestreserven verwendet.





\subsubsection{Geldpolitische Strategien in Europa}%% Autor: Gregor May %%
::anmerkung:: Geldmenge M3 erkl{\"a}ren




Die geldpolitische Strategie des Eurosystems wurde vom Rat der Europ{\"a}ischen Zentralbanken entwickelt und am 13.10.1998 der {\"O}ffentlichkeit pr{\"a}sentiert worden.
Zentrale Elemente sind seitdem:
\begin{itemize}
 \item Das Hauptziel der Preisstabilit{\"a}t, definiert als Anstieg des Harmonisierten Verbraucherpreisindex (HVPI) f{\"u}r den Euroraum von unter 2 Prozent.\footnote[78]{Seit Einf{\"u}hrung des Euros ist diese Marke  jedes Jahr {\"u}berschritten worden, gerade in der Finanzkrise mit mehr als 1 Prozent j{\"a}hrlich.} Allerdings muss die Preisstabilit{\"a}t nur mittelfristig gew{\"a}hrleistet werden.
 %\cite{EZB 2004, S.52}
 \item Eine Geldmengenpolitik mit der Verk{\"u}ndung eines Referenzwertes f{\"u}r das Wachstum der Geldmenge M3.
 \item Beobachtung und Einsch{\"a}tzung der k{\"u}nftigen Preisentwicklung sowie der Preisstabilit{\"a}t des Euroraumes insgesamt.
\end{itemize}

Das Ziel der Preisstabili{\"a}t ist 2003 dahingehend pr{\"a}zisiert worden, dass mittelfristig eine Preissteigerungsrate unter, aber ann{\"a}hernd, 2 Prozent gegen{\"u}ber dem Vorjahr eingehalten werden muss. Dadurch wird versucht, deflation{\"a}re Geldpolitik seitens der Nationalen Zentralbanken zu unterbinden.\footnote[101]{\citep[vgl.][S.564-568]{Basseler2010}}


Analyse und Beurteilung zuk{\"u}nftiger Preisentwicklungen nimmt eine zentrale Rolle in der Arbeit der EZB ein. Diese Arbeit wird als  "`Zwei-S{\"a}ulen-Strategie"' umschrieben\footnote[95]{\citep[S.568]{Basseler2010}}.



%%Umformulieren%%



Sie umfasst eine Analyse 

  realwirtschaftlicher Bestimmungsfaktoren, kurz- und mittelfristiger Preisentwicklungen, und

dar�ber hinaus eine monet{\"a}re Analyse der Entwicklung von Geldmengen, 


vor allem der Geldmenge M3, als mittel- bis langfristiger Faktor der Preisentwicklung.


%%Umformulieren%%





W{\"a}hrend sich die wirtschaftliche Analyse mit dem Zusammenspiel von Angebot- und Nachfrageentwicklung an den G{\"u}ter-, Dienstleistungs- und Faktorm{\"a}rkten zur Vermeidung von Inflation sehr keynesianisch gepr{\"a}gt ist, so wird die Geldmengenanalyse aus monetaristischer Sichtweise betrieben\footnote[97]{\citep[S.568]{Basseler2010}}. Somit l{\"a}sst sich die europ{\"a}ische Geldpolitik als ein Kompromiss, eine Mischung aus unterschiedlichen geldpolitischen Ansichten und geldtheorietischen Positionen verstehen. Eine vereinigte Geldpolitik Europas.\footnote[99]{\citep[vgl.][S.558f]{Basseler2010}}

\clearpage






\end{document}
