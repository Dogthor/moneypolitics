Das Leben des 1671 in Schottland geborenen John Law liest sich wie ein M{\"a}rchen [1]. Ein Draufg{\"a}nger im Kasino und mit den Frauen. In England zum Tode verurteilt, weil er als Sieger aus einem Duell hervorging, floh er nach Europa und bekam in den dortigen Kasinos und der fortschrittlicheren Finanzpolitik anderer L{\"a}nder, wie z.B. die Niederlande [2] ein f{\"u}r diese Zeit, neues Verst{\"a}ndnis von Geld: "Ungenutztes Geld war nichts - nichts als das Potential f{\"u}r Aktion." [3] Law erkannte rasch, dass mehr Bargeld zu mehr Handelsaktivit{\"a}t f{\"u}hrt und unterbreitete seine Ideen u.a. in Schottland und Turin. Die Staaten k{\"o}nnen Noten auf sich selbst ausstellen,d.h. Kredite gew{\"a}hren im Tausch mit der F{\"a}higkeit, in Zukunft Geld aufzubringen, mit Hilfe von Steuern. Dieses Geld wird in England noch heute fiat money [4] genannt und ist heute in Europa und auch den USA[5] gel{\"a}ufig. Durch Freunde bekam Law Kontakt zum Regenten von Frankreich[6], der ihm vollstes Vertrauen entgegen brachte. Die franz{\"o}sische Staatskasse, die zu der Zeit 3 milliarden Livres [7] an Schulden angeh{\"a}uft hat, ben{\"o}tigte neue Ideen und Law war der Mann der sie hatte. Er gr{\"u}ndete 1716 die Banque National, die erste Bank Frankreichs, ausgestattet mit 6 Millionen Livres Aktienkapital. Bareinzahlungen waren in Form von M{\"u}nzen auf Konten m{\"o}glich, auch {\"u}berweisungen durch Schecks auf andere Konten. Was aber neu und besonders war stellte "raison d'etre" [8] dar, d.h. die Ausgabe von Papiergeld. Das war die grandiose Idee und sie funktionierte. Die Leute hatten mehr Geld zur Verf{\"u}gung und gaben es auch aus, die G{\"u}ternachfrage stieg an und damit auch die G{\"u}terproduktion. Und einen weiteren Plan setzte Law um, die Gr{\"u}ndung der Mississippi-Gesellschaft. Philippe {\"u}bereignete ihm dazu Louisana und mit der Ausgabe von Aktien sollte damit eine Expedition [9] finanziert werden. Law und seine Familie bewegten sich in den h{\"o}chsten Kreisen, ja befanden sich f{\"o}rmlich an der Spitze, der High Societie. Der p{\"a}bstliche Nimbus, der bewegt war zur Geburtstagsfeier von Laws Tochter eingeladen zu sein, gab der Tochter einen Kuss und ihr {\"a}lterer Bruder ging mit Ludwig 15. jagen.
Alles war perfekt oder doch nicht? Das "`System"', wie Law es nannte, brach in sich zusammen. Die Gewinne wurden nicht wie versprochen f{\"u}r eine Expedition ausgegeben, sondern zur Begleichung der imensen Staatsschuld. Bei einer Abwertung der Noten verloren die Menschen endg{\"u}ltig das Vertrauen in diese.
Wie schon erw{\"a}hnt funktioniert die heutige Geldpolitik wie bei dem von Law durchgef{\"u}hrten Feldexperiment, doch Sicherheitsmechanismen verhindern meist eine derartige Eskalation.
Wir befassen uns in dieser Hausarbeit mit der Ausarbeitung dieser Sicherheitsmechanismen durch verschiedene Denkans{\"a}tze, Funktionen die die Instabilit{\"a}t des Systems vertr{\"a}glicher f{\"u}r seine Benutzer gemacht haben. Dabei gehen wir auf die Denkanst{\"o}fle der Str{\"o}mungen des Keynesianismus, der Monetaristen und der Gruppe der neuen politischen ÷konomie, der ÷sterreichischen Schule ein.

*** Hier kommen noch Kommentare f{\"u}r die case-study ***

_____________________________________________________________________________________________________________
[1]Nachzulesen z.B. : 2.Kapitel Schumpeters Reithosen: Die genialsten Wirtschaftstheorien und ihre verr{\"u}ckten Erfinder von Paul Strathern (Autor), Rita Seuss ({\"u}bersetzer), Sonja Schuhmacher ({\"u}bersetzer)
[2]Belebung der Schifffart dank Windm{\"u}hlen, erster B{\"o}rsencrash auf dem Tulpenmarkt
[3]Law Schumpeters Reithosen
[4]fiduzi{\"a}r: nicht durch Gold gedeckt, lat.: fiducia: Vertrauen
[5]"in god we trust" steht u.a. auf der 1-Dollar-Note und Gottesvertrauen braucht man auch
[6]Philippe , Herzog von OrlÈans, nachdem der Sonnenk{\"o}nig Ludwig 14. 1715 starb
[7]Livre frz.: Pfund war vom 9. bis zum 18. Jh. die franz{\"o}sische W{\"a}hrung, 1795 abgel{\"o}st vom Franc
[8]gedeckt durch die 6 Millionen Aktienkapital, Versprechen(!) auf Auszahlung, allerdings Barreserven von 350 tausend Livres
[9] ebenda
