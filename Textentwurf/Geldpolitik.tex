Nach Basseler hat die Geldpolitik die Hauptaufgabe, eine optimale Geldversorgung der Wirtschaft zu gewährleisten \citep*[vgl.][S. 551]{Basseler2010}.
Real wird diese Aufgabe von einer größtenteils staatlichen, aber unabhängigen Zentralbank übernommen. Dabei herrscht weit verbreitet Konsens darüber, dass Geldpolitik staatliche Aufgabe bleibt, auch wenn Ideen einer dezentralen, dem Wettbewerb unterliegenden Geldversorgung durch private Geschäftsbanken und ein System konkurierender Parallelwährungen kursieren. (Hayek)
"`Zentrale Zielgröße der Geldpolitik ist die Geldmenge M3. (...) Dabei ist zu beachten, dass das herkömmliche Konzept von Banken zum Konzept der Montären Finanzinstitute (MFIs) erweitert worden ist."`
\citep*[vgl.][S. 507]{Basseler2010}

\citep*[vgl.][S.508]{Basseler2010} "`MFIs sind also im Wesentlichen:
- Zentralbanken,
- Kreditinstitute und
- Geldmarktfonds.
(...) Innerhalb der Geldmenge M3 spielen Bargeldumlauf und täglich fällige Einlagen die größte Rolle; Einlagen mit vereinbarterter Kündigungsfrist (...) sind ebenfalls quantitativ bedeutsam; die übrigen Komponenten machen insgesamt nur knapp 20 Prozent der Geldmenge M3 aus."`


\subsubsection{ Akteure des Finanzbereiches}
%% Autor: Gregor May %%
\citep*[vgl.][S.511f]{Basseler2010} "`Die Akteure im Finanzbereich werden allgemein Finanzintermediäre genannt. [Diese] vermitteln Finanzprodukte zwischen den Anbietern und Nachfragern. (...) Dies sind vor allem Banken und Kapitalanlagegesellschaften, die selbst Finanzprodukte kreieren sowie institutionelle Anleger."`

 "`Es muss im Finanzbereich eine staatliche Institition geben, eine staatlich organisierte Zentralbank, die folgende Aufgaben erfüllt:
 \begin{itemize}
\item{die Ausgabe der gesetzlichen Zahlungsmittel (Staatliches Emissionsmonopol),}
\item{die Durchführung einer Geldpolitik mit dem Ziel einer angemessenen Begrenzung der Geldmenge,}
\item{die Organisation eines reibungslosen Zahlungs- und Kreditverkehrs als >>Bank der Banken<<,}
\item{die Wahrung der Geldwertstabilität,}
\item{die Bereitstellung einer ausreichenden Menge an Geld in Krisenzeiten"'}
\end{itemize}


\citep*[vgl.][S.512-13]{Basseler2010} "`Geschäftsbanken (oder auch Kreditinstitute) sind die zentralen Akteure im Finanzbereich einer Volkswirtschaft. Die erste zentrale Funktion von Geschäftsbanken (kurz: Banken) ist die Abwicklung des **Zahlungsverkehrs** einer Volkswirtschaft. (...) Die zweite zentrale Funktion von Banken ist die Organisation und Durchführung des **Kreditverkehrs** einer Volkswirtschaft. "`
"`(...)die Organisation des Kreditverkehrs ist ein klassisches Geschäft der banken: Sie beschaffen Geld, sie verleihen Geld und sie versuchen, dieses Geld mit Gewinn wieder zurückzubekommen.(...)Die Kreditgewährung war neben der Abwicklung des Zahlungsverkehrs die klassische Aufgabe der Geschäftsbank."`
 "`Daneben gibt es weitere Aufgaben der Banken, vor allem die Vermögensverwaltung der Kunden, die Ausgabe und den Handel mit Wertpapieren, die Beratung und Unterstützung bei Unternehmenszusammenschlüssen oder die Unterstützung von Unternehmen bei ihrer Kapitalaufnahme, etwa bei Börsengängen. Dies wird zusammenfassend **Investmentbanking** bezeichnet. (...) In diesem Segment des Bankengeschäfts  werden auf Zertifikate entwickelt und verkauft oder Fonds aufgelegt und Fondsanteile verkauft. (...)In Kontinentaleuropa ist (...)das Universalbankensystem etabliert(...). Sparkassen übernehmen (...)die Abwicklung des Zahlungsverkehrs und des >>kleinen<< Kreditverkehrs, kleine Privatbanken übernehmen eher die Funktionen des Investmentbankings und große Universalbanken übernehmen alle Geschäftssparten."`

\citep*[vgl.][S.515]{Basseler2010} "`Das Eigenkapital [einer Bank] setzt sich konkret zusammen aus dem Grundkapital, den Kapitalrücklagen und den Gewinnrücklagen (einbehaltene Gewinne) sowie einer stillen Einlage des Finanzmarktstabilisierungsfonds (...). Grundsätzlich ist das Eigenkapital der Banken von zentraler Bedeutung. Es ist letztlich das Kapital, das die Bank zum Ausgleich von Verlusten aus ihrem Kredit- und Investmentgeschäft einsetzen kann.  Daher sind in der Bankenaufsicht bestimmte Mindestanforderungen an die Höhe des haftenden Eigenkapitals (...)vorgesehen"`

\citep*[vgl.][S.512]{Basseler2010}  "`Mit dem (...) 01.01.1999 ist (...)die Europäische Zentralbank (EZB) die zentrale Institution - also die Zentralbank - für die Festlegung und Ausführung der Geldpolik. Daneben existiert weiterhin die Deutsche Bundesbank, die als Zentralbank der Bundesrepublik Deutschland Teil des Europäischen Systems der Zentralbanken ist. Sie ist (...) ausführendes organ der geldpolitischen Entscheidungen der (...)EZB. "`
\clearpage
