1.1 Benennung der Fragestellung
%% Autor: Marius Hanniske %%

Am Anfang steht die Forschung. Dabei sind zwei Motive herausgearbeitet, die die Triebfedern der wissenschaftlichen Forschung darstellen:
1.) die Neugierde nach Wissen über die Welt. Sie ist eine regelmäßige Motivation zur Forschung, wie auch John Law versucht hat seine Ideen umzusetzen um herauszufinden ob sie funktionieren.
Eine 2.) unabhängig davon treibende Kraft ist die unvollständige Information, die dafür sorgt, dass die Menschen nur eine vage oder gar keine Kenntnis von der Zukunft besitzen. Die Angst der Menschen vor dem Unbekannten veranlasst sie dazu diese Unkenntnis zu beseitigen. Dieser Drang zur Veränderung der Situation schwindet jedoch, wenn die Zukunft vielversprechend aussieht.[1] Nicht zu vernachlässigen sind die Möglichkeiten einer Veränderung, als Reaktion auf die enttäuschenden Ergebnisse wissenschaftlicher Theorien. Was tun, wenn sich eine Diskrepanz zwischen der Theorie und Realität aufzeigt? Die erste Möglichkeit: Evolution: Anpassung der Theorie an die Realität. Schwierig, wenn die Theorie fest gefahren ist und auf Probleme mit den immer gleichen Argumentationen antworten will. Dann bleibt nur die zweite Möglichkeit der Revolution: bei der eine völlig neue Konstruktion der Analytik angefertigt wird, die die Realität besser wiedergibt als ihre Vorgängertheorie. Nicht zu vergessen das Falsifizierbarkeitskriterium: Widerlegung einer Theorie hat ein stärkeres Gewicht als ihre Bestätigung.[2] Verdeutlichen wir uns das an den ökonomischen Theorien der Klassiker und Neoklassiker. Deren Theorien über ein Jahrhundert das ökonomische Geschehen geprägt haben und auch noch heute in einem gewissen Rahmen die von uns zu betrachtenden Strömungen agitieren. Die Klassiker berufen sich auf die Kräfte des freien Marktes. Den Anfang machte dabei Adam Smith mit seinem 1776 veröffentlichten Werk "Wohlstand der Nationen". Die freie Marktwirtschaft wird erklärt von einer unsichtbaren Hand[3] oder der Vorstellung einer großen Flexibilität von Zinsen, Preisen und Löhnen und einer raschen Anpassung der Wirtschaftssubjekte an veränderte Bedingungen. D.h. sie unterstellen den Individuen ein rationales Verhalten, aus einer Vielzahl von Angeboten das Beste herauszufiltern. In einem Ungleichgewicht reagieren die Preise schneller als realisierte Angebots- und Nachfragemenge und führen sehr rasch zu einem Gleichgewicht zurück.[4] Mit den Argumentationen der Realpreise[5] befassten sich vornehmlich die Neoklassiker.[6] Die Preise wiederum können bequemer zugeteilt, identifiziert und bewertet werden wenn sie eine (be)rechenbare Einheit erhalten. Diese Einheit, ob nun US-Dollar, Yen oder Euro stellt das Geld ganz allgemein dar. Dabei lässt sich das Geld in drei Funktionen unterteilen: Tauschmittel, Recheneinheit und Wertaufbewahrungsfunktion.[7] Die Klassiker haben die Wertaufbewahrungsfunktion des Geldes jedoch vernachlässigt, weil es der ökonomischen Rationalität widerspricht[8] Das Geld als Tauschmittel und Recheneinheit macht aus einer geschlossenen Gesellschaft[9] eine Tauschwirtschaft. Doch in Anbetracht seiner Wertaufberahrungsfunktion hat Geld eine bestimmte Wirkung, in deren Sinn die Geldmenge ein analytisches Konzept darstellt mit dem die Wirkung des Geldes als Zielgröße der Geldpolitik herangezogen wird.[10] Die Forderung der Klassiker war individuelle Selbstständigkeit und Freiheit die ihnen mit dem marktwirtschaftlichen Wettbewerbsmechanismus am besten geeignet erschien um die Interessen der Produzenten mit denen der Konsumenten in Einklang zu bringen. Dabei ordneten sie dem Staat eine untergeordnete Rolle zu. Seine Aufgabe bestand "lediglich" aus der Verwaltung und der Aufrechterhaltung der Rechtsordnung, für die innere und äußere Sicherheit, das Verkehrswesen und das Bildungs- und Gesundheitswesen zu sorgen.[11]


\citep[S.11]{Basseler2010}
____________________________________________________________________________________
[1] \citep[S.11]{bombach1981theorie}
[2] \citep[S.161]{bombach1981theorie}
[3] \citep[S.137]{bombach1981theorie}
[4] \citep[S.11]{Basseler2010} S.291
[5] Die Preise von Gütern kommen nie unabhängig von Preisen anderer Güter zustande
[6] \citep[S.138]{bombach1981theorie}
[7] \citep[S.11]{Basseler2010} S.417
[8] \citep[S.53]{bombach1981theorie}
[9] In dem Fall ist mit "geschlossener Gesellschaft" eine nur aus Selbstversorgungseinheiten bestehende Wirtschaft, in der kein Gütertausch stattfindet, gemeint.
[10] \citep[S.11]{Basseler2010} S.421
[11] \citep[S.11]{Basseler2010} S.60/61
