Auf den Hintergrund der Weltwirtschaftskrise der Jahre 1929 bis 1932 entwickelte
John Maynard Keynes seine Konzeption zu seinem Hauptwerk "The General Theory of
Employment, Intrest and Money". Nahezu s{\"a}mtliche Investitionst{\"a}tigkeiten kamen
in dieser Krise zum Erliegen. Das f{\"u}hrte zu einer bis dahin nicht gekannten
Massenarbeitslosigkeit. [1] Fehler die gemacht wurden[2] sind eindeutig den falschen
Annahmen der Klassiker zuzuschreiben:[3] "das sich bei freier Konkurrenz
einpendelnde Preis-, Lohn- und Zinsniveau f{\"u}hre stets zu Vollbesch{\"a}ftigung der
Produktionsfaktoren." Das Saysche Theorem besagt, dass das Angebot seine Nachfrage
schafft, indem alle produzierten G{\"u}ter mit dem im Produktionsprozess verdienten
Einkommen aufgekauft werden. Nach Keynes nimmt jedoch der Hang zum Verbrauch bei
zunehmenden Einkommen relativ ab, sodass Say's Theorem mehr oder weniger unerf{\"u}llt
bleibt. Es wird gespart, wodurch ein Nachfrager{\"u}ckgang entsteht. Die sinkende
Nachfrage l{\"a}sst die Absatzerwartungen zur{\"u}ck gehen und dies kann dann zu
Arbeitslosigkeit f{\"u}hren.[4] Keynes bem{\"a}ngelte an der klassischen Theorie, dass sie die
Vollbesch{\"a}ftigung aller Produktionsfaktoren voraussetze, was ja offensichtlich in
der Weltwirtschaftskrise nicht zutraf. Den Hauptgrund f{\"u}r die hohe Arbeitslosigkeit
in der Volkswirtschaft sieht Keynes in der unzureichenden Nachfrage die sich aus
{\"u}bersch{\"u}ssiger Ersparnis ergibt. Somit wird die Nachfrage zur strategischen Gr{\"o}{"s}e,
die durch staatliche Nachfrageimpulse stimuliert werden soll, wenn die
Wirtschaftst{\"a}tigkeit zu erlahmen droht. Obwohl Keynes in bestimmten Situationen die
Investition positiv vom Zins beeinflusst sieht, h{\"a}lt er den Einfluss des Zinssatzes
bei weitem nicht f{\"u}r ausreichend um eine optimale Investitionsrate zu erzielen.[5]
Keynes argumentiert, dass die Investitionsnachfrage praktisch nicht auf eine
{\"a}nderung des Zinssatzes reagiert und wenn es doch so w{\"a}re, w{\"u}rde das Systemgleichgewicht
ein Zinsniveau, dass n{\"o}tige Investitionen zur Vollbesch{\"a}ftigung lohnend machen w{\"u}rde,
nicht zu.[6]
Man kann also zusammenfassen, dass in Keynes Wirtschaftspolitik die Besch{\"a}ftigung
die prim{\"a}re Zielgr{\"o}{"s}e ist, die angestrebt werden soll und demnach die Fiskalpolitik
eine starke Wirkung hat, st{\"a}rker als die Geldpolitik, die nur indirekt wirke.[]
Das wird auch nochmal in dem Folgenden Zitat von John Maynard Keynes deutlich:
"Die Bedeutung des Geldes liegt allein in seiner Kaufkraft. Eine Ver{\"a}nderung in der
M{\"u}nzeinheit(...), hat (...)keine Auswirkungen."[8]




_________________________________________________________________________________
[1] \footnote[601]{\citep[S.203]{peters2000}
[2] \footnote[602]{z.B.  verg{\"o}flerte die Landwirtschaft w{\"a}hrend des 1. Weltkriegs seine Kapazit{\"a}ten nach dem Abbruch der internationalen Handelsbeziehungen und litt nach Beendigung des
Krieges an {\"U}berproduktion die sie nicht abbaute. So fielen in den Jahren vor der
Krise die Preise f{\"u}r Anbauprodukte ins Bodenlose. Weiterer Preis- und Lohnverfall
waren die Folge und damit ein Anstieg der Arbeitslosigkeit.
Vgl. \citep[S.14ff]{bombach1981theorie}}

[3] \footnote[603]{ Keynes - Das Ende des laissez faire 2011 Hauptschrift II S.36
[4] \footnote[604]{\citep[S.203]{peters2000}
[5] \footnote[605]{\citep[S.208]{peters2000}
[6] \footnote[606]{ \citep[S.174]{bombach1981theorie}}
[7] \footnote[607]{ \citep[S.181]{bombach1981theorie}}
[8] \footnote[608]{J.M. Keynes - Ein Traktat {\"u}ber W{\"a}hrungsreform S.1
